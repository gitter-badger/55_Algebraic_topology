\documentclass[12pt]{article}
\usepackage{pmmeta}
\pmcanonicalname{AlexanderGrothendiecksMathematicalgenealogy}
\pmcreated{2013-03-22 18:25:22}
\pmmodified{2013-03-22 18:25:22}
\pmowner{bci1}{20947}
\pmmodifier{bci1}{20947}
\pmtitle{Alexander Grothendieck's mathematical `genealogy'}
\pmrecord{17}{41076}
\pmprivacy{1}
\pmauthor{bci1}{20947}
\pmtype{Feature}
\pmcomment{trigger rebuild}
\pmclassification{msc}{55-00}
\pmclassification{msc}{18-00}
\pmclassification{msc}{01A61}
\pmsynonym{heraldics and heredity}{AlexanderGrothendiecksMathematicalgenealogy}
%\pmkeywords{Alexander Grothendieck' s mathematical `genealogy'}
\pmrelated{AlexanderGrothendieckABiographyOf}
\pmrelated{AlexSMathematicalHeritageEsquisseDunProgramme}
\pmdefines{genealogic tree}

% this is the default PlanetMath preamble.  as your knowledge
% of TeX increases, you will probably want to edit this, but
% it should be fine as is for beginners.

% almost certainly you want these
\usepackage{amssymb}
\usepackage{amsmath}
\usepackage{amsfonts}

% used for TeXing text within eps files
%\usepackage{psfrag}
% need this for including graphics (\includegraphics)
%\usepackage{graphicx}
% for neatly defining theorems and propositions
%\usepackage{amsthm}
% making logically defined graphics
%%%\usepackage{xypic}

% there are many more packages, add them here as you need them

% define commands here
\usepackage{amsmath, amssymb, amsfonts, amsthm, amscd, latexsym}
%%\usepackage{xypic}
\usepackage[mathscr]{eucal}

\setlength{\textwidth}{6.5in}
%\setlength{\textwidth}{16cm}
\setlength{\textheight}{9.0in}
%\setlength{\textheight}{24cm}

\hoffset=-.75in     %%ps format
%\hoffset=-1.0in     %%hp format
\voffset=-.4in

\theoremstyle{plain}
\newtheorem{lemma}{Lemma}[section]
\newtheorem{proposition}{Proposition}[section]
\newtheorem{theorem}{Theorem}[section]
\newtheorem{corollary}{Corollary}[section]

\theoremstyle{definition}
\newtheorem{definition}{Definition}[section]
\newtheorem{example}{Example}[section]
%\theoremstyle{remark}
\newtheorem{remark}{Remark}[section]
\newtheorem*{notation}{Notation}
\newtheorem*{claim}{Claim}

\renewcommand{\thefootnote}{\ensuremath{\fnsymbol{footnote%%@
}}}
\numberwithin{equation}{section}

\newcommand{\Ad}{{\rm Ad}}
\newcommand{\Aut}{{\rm Aut}}
\newcommand{\Cl}{{\rm Cl}}
\newcommand{\Co}{{\rm Co}}
\newcommand{\DES}{{\rm DES}}
\newcommand{\Diff}{{\rm Diff}}
\newcommand{\Dom}{{\rm Dom}}
\newcommand{\Hol}{{\rm Hol}}
\newcommand{\Mon}{{\rm Mon}}
\newcommand{\Hom}{{\rm Hom}}
\newcommand{\Ker}{{\rm Ker}}
\newcommand{\Ind}{{\rm Ind}}
\newcommand{\IM}{{\rm Im}}
\newcommand{\Is}{{\rm Is}}
\newcommand{\ID}{{\rm id}}
\newcommand{\GL}{{\rm GL}}
\newcommand{\Iso}{{\rm Iso}}
\newcommand{\Sem}{{\rm Sem}}
\newcommand{\St}{{\rm St}}
\newcommand{\Sym}{{\rm Sym}}
\newcommand{\SU}{{\rm SU}}
\newcommand{\Tor}{{\rm Tor}}
\newcommand{\U}{{\rm U}}

\newcommand{\A}{\mathcal A}
\newcommand{\Ce}{\mathcal C}
\newcommand{\D}{\mathcal D}
\newcommand{\E}{\mathcal E}
\newcommand{\F}{\mathcal F}
\newcommand{\G}{\mathcal G}
\newcommand{\Q}{\mathcal Q}
\newcommand{\R}{\mathcal R}
\newcommand{\cS}{\mathcal S}
\newcommand{\cU}{\mathcal U}
\newcommand{\W}{\mathcal W}

\newcommand{\bA}{\mathbb{A}}
\newcommand{\bB}{\mathbb{B}}
\newcommand{\bC}{\mathbb{C}}
\newcommand{\bD}{\mathbb{D}}
\newcommand{\bE}{\mathbb{E}}
\newcommand{\bF}{\mathbb{F}}
\newcommand{\bG}{\mathbb{G}}
\newcommand{\bK}{\mathbb{K}}
\newcommand{\bM}{\mathbb{M}}
\newcommand{\bN}{\mathbb{N}}
\newcommand{\bO}{\mathbb{O}}
\newcommand{\bP}{\mathbb{P}}
\newcommand{\bR}{\mathbb{R}}
\newcommand{\bV}{\mathbb{V}}
\newcommand{\bZ}{\mathbb{Z}}

\newcommand{\bfE}{\mathbf{E}}
\newcommand{\bfX}{\mathbf{X}}
\newcommand{\bfY}{\mathbf{Y}}
\newcommand{\bfZ}{\mathbf{Z}}

\renewcommand{\O}{\Omega}
\renewcommand{\o}{\omega}
\newcommand{\vp}{\varphi}
\newcommand{\vep}{\varepsilon}

\newcommand{\diag}{{\rm diag}}
\newcommand{\grp}{{\mathbb G}}
\newcommand{\dgrp}{{\mathbb D}}
\newcommand{\desp}{{\mathbb D^{\rm{es}}}}
\newcommand{\Geod}{{\rm Geod}}
\newcommand{\geod}{{\rm geod}}
\newcommand{\hgr}{{\mathbb H}}
\newcommand{\mgr}{{\mathbb M}}
\newcommand{\ob}{{\rm Ob}}
\newcommand{\obg}{{\rm Ob(\mathbb G)}}
\newcommand{\obgp}{{\rm Ob(\mathbb G')}}
\newcommand{\obh}{{\rm Ob(\mathbb H)}}
\newcommand{\Osmooth}{{\Omega^{\infty}(X,*)}}
\newcommand{\ghomotop}{{\rho_2^{\square}}}
\newcommand{\gcalp}{{\mathbb G(\mathcal P)}}

\newcommand{\rf}{{R_{\mathcal F}}}
\newcommand{\glob}{{\rm glob}}
\newcommand{\loc}{{\rm loc}}
\newcommand{\TOP}{{\rm TOP}}

\newcommand{\wti}{\widetilde}
\newcommand{\what}{\widehat}

\renewcommand{\a}{\alpha}
\newcommand{\be}{\beta}
\newcommand{\ga}{\gamma}
\newcommand{\Ga}{\Gamma}
\newcommand{\de}{\delta}
\newcommand{\del}{\partial}
\newcommand{\ka}{\kappa}
\newcommand{\si}{\sigma}
\newcommand{\ta}{\tau}
\newcommand{\med}{\medbreak}
\newcommand{\medn}{\medbreak \noindent}
\newcommand{\bign}{\bigbreak \noindent}
\newcommand{\lra}{{\longrightarrow}}
\newcommand{\ra}{{\rightarrow}}
\newcommand{\rat}{{\rightarrowtail}}
\newcommand{\oset}[1]{\overset {#1}{\ra}}
\newcommand{\osetl}[1]{\overset {#1}{\lra}}
\newcommand{\hr}{{\hookrightarrow}}
\begin{document}
\section{Alexander Grothendieck} 
(n\'ee `` Alexander Raddatz '' in Berlin, Germany;
French adopted name: Alexandre Grothendieck,
with the last name being his mother's maiden name.)

\bigbreak
\subsection{Mathematical `Genealogy'}

%%%%%%%%%%%%%

Ph.D. Universit\'e Henri Poincar\'e Nancy 1    \\
Dissertation: ``Produits tensoriels topologiques et \'espaces nucl\'eaires.'' \\
Mathematics Subject Classification: 14 = Algebraic Geometry \\
Advisor No. 1: Laurent Schwartz \\
Advisor No. 2: Jean Dieudonn\'e \\

\subsection{Alexander Grothendieck's Students' List:}
(incomplete)\\
Name--- School--- Year ---                 --------Descendants----- \\
Pierre Berthelot                                     1 \\
Carlos Contou-Carrare Universit\' Montpellier II 1983 2 \\
{\em Pierre Deligne} Universit\'e Libre de Bruxelles 1968 62 \\
Michel Demazure Universit\'e de Paris 1965 15 \\
Luc Illusie  1971 22 \\
Yves Ladegaillerie Universit\'e Montpellier II 1976  \\
William Messing Princeton University 1971 5 \\
Michel Raynaud    7 \\
Hoang Sinh Universit\'e Denis Diderot - Paris VII 1975  \\
Jean-Louis Verdier  1967 28 

%%%%%%%%%%

\bigbreak
\textbf{Alexander Grothendieck's Doctoral Advisor No.1--Laurent Schwartz: \\
 A Concise Biography and `Genealogy':}\\

%%%%%%%%

Ph.D. Universit\'e Louis Pasteur-- Strasbourg I    1943 \\ 
Dissertation: ``Sommes de Fonctions Exponentielles R\'eelles'' \\
Advisor: {\em Georges Valiron} \\

\subsection{Student(s):} 

Table \\
Name--- School--- Year--------                                 Descendants \\

Bernard Beauzamy Universit\'e Denis Diderot - Paris VII 1976        25 \\
Louis Boutet de Monvel    34 \\
Georges Glaeser Universit\'e Henri Poincar\'e Nancy 1 1957 3 \\
{\bf Alexander Grothendieck}  Universit\'e Henri Poincar\'e Nancy 1   152 \\
Jacques-Louis Lions Universit\'e Henri Poincar\'e Nancy 1 1954      1,202 \\
Bernard Malgrange Universit\'e Henri Poincar\'e Nancy 1 1955 21 \\
André Martineau    37   \\
Bernard Maurey Universit\'e Denis Diderot - Paris VII 1973 4 \\
M. Narasimhan     \\
Gilles Pisier Universit\'e Denis Diderot - Paris VII 1977 23 \\
Erik Thomas    14 \\
J. François Treves Universit\'e Paris IV-Sorbonne   34 \\
Andre Unterberger Universit\'e Denis Diderot - Paris VII   1 \\
Ali Ustunel Universit\'e Paris VI - Pierre et Marie Curie 1981 4 \\
\med
{\em Note:} According to the on-line databases, Laurent Schwartz has 14 students and 1,544 descendants.\\

Laurent Schwartz \\

(1915 - 2002)\\
 
List of References (5 books/articles) \\
  
{\bf Honours awarded to Laurent Schwartz} \\
 
Speaker at International Congress 1950  \\
Fields Medal 1950  \\
British Mathematical Colloquium plenary speaker 1954 . \\

References for Laurent Schwartz\\
--------------------------------------------

Obituary in The Times [available on the Web] \\
\subsection{Articles:}
\bigbreak
H. Bohr, The work of L. Schwartz, Proceedings of the International Congress of Mathematicians, Harvard, 1950 (Providence, RI, 1952). \\
B. L. Malgrange, Schwartz et la th\'eorie des distributions, in Colloquium in honor of Laurent Schwartz 1 (Palaiseau, 1983), Astérisque No. 131 (1985), 25-33. \\
L. Schwartz, Notice sur les travaux scientifiques de Laurent Schwartz, in Mathematical analysis and applications A (New York-London, 1981), 1-25. \\
L. Schwartz, Calcul infinit\'esimal stochastique, in Analyse Math\'ematique et Applications (1988). \\

Source: Encyclopaedia Britannica.


$$<a href="http://www.genealogy.math.ndsu.nodak.edu">$$
$$<img src="http://www.genealogy.math.ndsu.nodak.edu/img/treebutton.gif"$$
$$width="80" height="86" alt="The Mathematics Genealogy Project"$$
$$style="border: 0" /></a>$$
%%%%%
%%%%%
\end{document}
