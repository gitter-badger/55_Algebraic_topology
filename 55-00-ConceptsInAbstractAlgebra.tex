\documentclass[12pt]{article}
\usepackage{pmmeta}
\pmcanonicalname{ConceptsInAbstractAlgebra}
\pmcreated{2013-03-22 14:42:38}
\pmmodified{2013-03-22 14:42:38}
\pmowner{matte}{1858}
\pmmodifier{matte}{1858}
\pmtitle{concepts in abstract algebra}
\pmrecord{26}{36330}
\pmprivacy{1}
\pmauthor{matte}{1858}
\pmtype{Topic}
\pmcomment{trigger rebuild}
\pmclassification{msc}{55-00}
\pmclassification{msc}{18-00}
\pmclassification{msc}{16-00}
\pmclassification{msc}{13-00}
\pmclassification{msc}{20-00}
\pmclassification{msc}{15-00}
\pmsynonym{classes of algebras}{ConceptsInAbstractAlgebra}
\pmsynonym{examples of algebras}{ConceptsInAbstractAlgebra}
%\pmkeywords{specific and general algebras with linked definitions}

% this is the default PlanetMath preamble.  as your knowledge
% of TeX increases, you will probably want to edit this, but
% it should be fine as is for beginners.

% almost certainly you want these
\usepackage{amssymb}
\usepackage{amsmath}
\usepackage{amsfonts}
\usepackage{amsthm}

\usepackage{mathrsfs}

% used for TeXing text within eps files
%\usepackage{psfrag}
% need this for including graphics (\includegraphics)
%\usepackage{graphicx}
% for neatly defining theorems and propositions
%
% making logically defined graphics
%%%\usepackage{xypic}

% there are many more packages, add them here as you need them

% define commands here

\newcommand{\sR}[0]{\mathbb{R}}
\newcommand{\sC}[0]{\mathbb{C}}
\newcommand{\sN}[0]{\mathbb{N}}
\newcommand{\sZ}[0]{\mathbb{Z}}

 \usepackage{bbm}
 \newcommand{\Z}{\mathbbmss{Z}}
 \newcommand{\C}{\mathbbmss{C}}
 \newcommand{\R}{\mathbbmss{R}}
 \newcommand{\Q}{\mathbbmss{Q}}



\newcommand*{\norm}[1]{\lVert #1 \rVert}
\newcommand*{\abs}[1]{| #1 |}



\newtheorem{thm}{Theorem}
\newtheorem{defn}{Definition}
\newtheorem{prop}{Proposition}
\newtheorem{lemma}{Lemma}
\newtheorem{cor}{Corollary}
\begin{document}
The aim of this entry is to present a list of the key \PMlinkescapetext{objects and} operators used in abstract algebra. Each entry in the list \PMlinkescapetext{links} (or will \PMlinkescapetext{link} in the future) to the corresponding PlanetMath entry where the object is presented in greater detail. For convenience, this list also presents the encouraged notation to use (at PlanetMath) for these objects.

\begin{itemize}
\item $G$ group, subgroup
\item normal subgroup
\item cyclic group
\item group algebra
\item Galois group
\item Polish group
\item \PMlinkname{G-Set}{G-Set}
\item groupoid group
\item $\mathcal{G}$ groupoid
\item semigroup
\item monoid
\item generator, generate
\item $[a,b]$ \PMlinkname{commutator}{DerivedSubgroup}
\item $\langle g \rangle$ cyclic group generated by an element
\item $R$ ring, subring
\item polynomial ring
\item $I$, $\mathfrak{a}$ ideal
\item $R/I$ quotient ring
\item $S^{-1}R$ localization of $R$ at $S$
\item $D$ integral domain
\item division ring
\item ring group
\item $F$, $K$ field
\item $N_G(H)$ normalizer of a subgroup
\item $C(a)$ centralizer of an element
\item $Z(G)$ center of a group (or centre of a group)
\item $H \triangleleft G$ normal subgroup
\item $H \operatorname{char} G$ characteristic subgroup
\item $G/H$ quotient group
\item $\langle S^G\rangle$ normal closure
\item $aH, \,Ha$ \PMlinkname{left coset and right coset}{Coset} respectively
\item element, unit, unity, inverse, identity
\item nilpotent
\item idempotent
\item $M$ module, submodule
\item homomorphism, homomorphy
\item isomorphism, isomorphy, isomorphic
\item monomorphism, epimorphism
\item endomorphism
\item automorphism
\end{itemize}

\textbf{General Algebras and Algebroids}

\begin{itemize}
\item Universal algebras
\item superalgerbas
\item F-algebras
\item double algebras
\item Quantum Operator algebras
\item general algebras
\item higher dimensional algebras
\item logic algebras
\item quantum logic algebras
\end{itemize}
%%%%%
%%%%%
\end{document}
