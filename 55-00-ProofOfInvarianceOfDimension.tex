\documentclass[12pt]{article}
\usepackage{pmmeta}
\pmcanonicalname{ProofOfInvarianceOfDimension}
\pmcreated{2013-03-22 15:56:39}
\pmmodified{2013-03-22 15:56:39}
\pmowner{Algeboy}{12884}
\pmmodifier{Algeboy}{12884}
\pmtitle{proof of invariance of dimension}
\pmrecord{8}{37954}
\pmprivacy{1}
\pmauthor{Algeboy}{12884}
\pmtype{Proof}
\pmcomment{trigger rebuild}
\pmclassification{msc}{55-00}

\endmetadata

\usepackage{latexsym}
\usepackage{amssymb}
\usepackage{amsmath}
\usepackage{amsfonts}
\usepackage{amsthm}

%%\usepackage{xypic}

%-----------------------------------------------------

%       Standard theoremlike environments.

%       Stolen directly from AMSLaTeX sample

%-----------------------------------------------------

%% \theoremstyle{plain} %% This is the default

\newtheorem{thm}{Theorem}

\newtheorem{coro}[thm]{Corollary}

\newtheorem{lem}[thm]{Lemma}

\newtheorem{lemma}[thm]{Lemma}

\newtheorem{prop}[thm]{Proposition}

\newtheorem{conjecture}[thm]{Conjecture}

\newtheorem{conj}[thm]{Conjecture}

\newtheorem{defn}[thm]{Definition}

\newtheorem{remark}[thm]{Remark}

\newtheorem{ex}[thm]{Example}



%\countstyle[equation]{thm}



%--------------------------------------------------

%       Item references.

%--------------------------------------------------


\newcommand{\exref}[1]{Example-\ref{#1}}

\newcommand{\thmref}[1]{Theorem-\ref{#1}}

\newcommand{\defref}[1]{Definition-\ref{#1}}

\newcommand{\eqnref}[1]{(\ref{#1})}

\newcommand{\secref}[1]{Section-\ref{#1}}

\newcommand{\lemref}[1]{Lemma-\ref{#1}}

\newcommand{\propref}[1]{Prop\-o\-si\-tion-\ref{#1}}

\newcommand{\corref}[1]{Cor\-ol\-lary-\ref{#1}}

\newcommand{\figref}[1]{Fig\-ure-\ref{#1}}

\newcommand{\conjref}[1]{Conjecture-\ref{#1}}


% Normal subgroup or equal.

\providecommand{\normaleq}{\unlhd}

% Normal subgroup.

\providecommand{\normal}{\lhd}

\providecommand{\rnormal}{\rhd}
% Divides, does not divide.

\providecommand{\divides}{\mid}

\providecommand{\ndivides}{\nmid}


\providecommand{\union}{\cup}

\providecommand{\bigunion}{\bigcup}

\providecommand{\intersect}{\cap}

\providecommand{\bigintersect}{\bigcap}










\begin{document}
An application of the invariance of dimension theorem shows that $\mathbb{R}^n$ is homeomorphic to $\mathbb{R}^m$ if and only if $m=n$.  Already this is a difficult question.  (We will assume $n\leq m$ throughout this article.)

Simple arguments suffice for small dimensions.
\begin{itemize}
\item If $n=0$ cardinality is sufficient: there can be no bijection between $\mathbb{R}^0=\{0\}$ and $\mathbb{R}^n$, $m>0$, as the latter is uncountable.
\item
If $n=1$, then suppose $f:\mathbb{R}\rightarrow\mathbb{R}^m$ is a homeomorphism with $m>1$.  Then certainly the following restriction of $f$
\[f:\mathbb{R}-\{0\}\rightarrow \mathbb{R}^m-\{f(0)\}\]
is also a homeomorphism.  Yet, as $m>1$, $\mathbb{R}^m-\{f(0)\}$ is (path) connected but $\mathbb{R}-\{0\}$ is not connected.  Thus this restriction of
$f$ cannot be a homeomorphism so indeed the original $f$ could not be a homeomorphism.
\end{itemize}
Unfortunately neither of these two arguments extends well to the cases where $n,m>1$.  Indeed even the case for $n=1$ requires a reasonable amount of work to fill in the details.  However, the latter approach does provide the necessary hint for a full solution.

To solve the problem outright depends on algebraic invariants from homology, a
surprisingly big hammer for such a basic topological question.  But the conceptual steps are still basic, and we will attempt to highlight them in our exposition of the proof.

Let $U$ and $V$ be non-empty open subsets of $\mathbb{R}^n$ and $\mathbb{R}^m$ respectively.  Assume that $f:U\rightarrow V$ is a homeomorphism.  

Choose a point $x\in U$ (akin to the point we removed when $n=1$.)  Then consider the relative homology groups $H_i(U,U-\{x\})$, $i\in \mathbb{N}$.  As $U$ is open we may apply the Excision Theorem (axiom) to claim $H_i(U,U-\{x\})\cong H_i(\mathbb{R}^n,\mathbb{R}^n-\{x\})$ -- basically, to look at a punctured open disk it to look at a punctured $\mathbb{R}^n$.  Now we look at the induced long exact sequence from the relative pair $(\mathbb{R}^n,\mathbb{R}^n-\{x\})$ and find $H_i(\mathbb{R}^n,\mathbb{R}^n-\{x\})$ is isomorphic to the reduced homology $\tilde{H}_i(\mathbb{R}^n-\{x\})$.  But $\mathbb{R}^n-\{x\}$ contracts to the sphere $S^{n-1}$ -- and homologoy preserves homotopy type -- so we now have $H_i(U,U-\{x\})\cong H_i(S^{n-1})$.  (Puncture a disk, and it deflates to a sphere of lower dimension.)  

Now it is an exercise in homology to prove that $H_i(S^{n-1})=0$ if $i\neq 0,n-1$ and $\mathbb{Z}$ otherwise.  In particular we are using the fact that the invariance of dimension of spheres is (more) easily established by the homology groups.  

We now repeat the process with $V$.  If $U$ and $V$ are indeed homeomorphic, then this process will result in isomorphic homology groups for every $i\in\mathbb{N}$.  In particular, 
\[\mathbb{Z}\cong H_{m-1}(S^{m-1})\cong H_{m-1}(V,V-\{f(x)\})\cong H_{m-1}(U,U-\{x\})\cong H_{m-1}(S^{n-1}).\]
Thus either $m=1$ which implies $n=0,1$ as $n\leq m$, or $m=n$.  If $n=0$ we have already seen $m=n$.  So the result stands for all $m,n$.


For a detailed accounting of this theorem together with the necessary lemmas refer to:

\noindent Allen Hatcher, \emph{Algebraic Topology}, Cambridge University Press, Cambridge, 2002.  Available on-line at: \PMlinkexternal{http://www.math.cornell.edu/~hatcher/AT/ATpage.html}{http://www.math.cornell.edu/~hatcher/AT/ATpage.html}
%%%%%
%%%%%
\end{document}
