\documentclass[12pt]{article}
\usepackage{pmmeta}
\pmcanonicalname{SorgenfreyLine}
\pmcreated{2013-03-22 13:03:45}
\pmmodified{2013-03-22 13:03:45}
\pmowner{yark}{2760}
\pmmodifier{yark}{2760}
\pmtitle{Sorgenfrey line}
\pmrecord{9}{33469}
\pmprivacy{1}
\pmauthor{yark}{2760}
\pmtype{Example}
\pmcomment{trigger rebuild}
\pmclassification{msc}{55-00}
\pmclassification{msc}{54-00}
\pmclassification{msc}{22-00}
\pmsynonym{Sorgenfrey topology}{SorgenfreyLine}
\pmdefines{lower limit topology}

\usepackage{amssymb}
\usepackage{amsmath}
\usepackage{amsfonts}

\def\sse{\subseteq}
\def\bigtimes{\mathop{\mbox{\Huge $\times$}}}
\def\impl{\Rightarrow}
\def\R{\mathbb{R}}
\begin{document}
The \emph{Sorgenfrey line} is a nonstandard topology on the real line $\R$.
Its topology is defined by the following base of half open intervals
\[
  \mathcal{B} = \{ {[a,b)} \mid a,b\in\R, a<b \}.
\]
Another name is \emph{lower limit topology}, since a sequence $x_\alpha$
converges only if it converges in the standard topology and its limit is
a limit from above (which, in this case, means that at most finitely many
points of the sequence lie below the limit). For example, the sequence
$(1/n)$ converges to $0$, while $(-1/n)$ does not.

This topology is finer than the standard topology on $\R$.
The Sorgenfrey line is first countable and separable, but is not second countable.
It is therefore not metrizable.

\begin{thebibliography}{9}
\bibitem{sorgenfrey}
 R.~H.~Sorgenfrey,
 {\it On the topological product of paracompact spaces},
 Bulletin of the American Mathematical Society 53 (1947) 631--632.
 (This paper is
 \PMlinkexternal{available on-line}{http://projecteuclid.org/euclid.bams/1183510809}
 from Project Euclid.)
\end{thebibliography}
%%%%%
%%%%%
\end{document}
