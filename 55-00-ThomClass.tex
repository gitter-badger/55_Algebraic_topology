\documentclass[12pt]{article}
\usepackage{pmmeta}
\pmcanonicalname{ThomClass}
\pmcreated{2013-03-22 15:40:48}
\pmmodified{2013-03-22 15:40:48}
\pmowner{antonio}{1116}
\pmmodifier{antonio}{1116}
\pmtitle{Thom class}
\pmrecord{5}{37621}
\pmprivacy{1}
\pmauthor{antonio}{1116}
\pmtype{Definition}
\pmcomment{trigger rebuild}
\pmclassification{msc}{55-00}
\pmrelated{Orientation2}
\pmdefines{orientability with respect to a generalized homology theory}

% used for TeXing text within eps files
%\usepackage{psfrag}
% need this for including graphics (\includegraphics)
%\usepackage{graphicx}
% for neatly defining theorems and propositions
%\usepackage{amsthm}
% making logically defined graphics
%%%\usepackage{xypic}

\usepackage{theorem}
\usepackage{amsmath}
\usepackage{amsfonts}
\usepackage{amssymb}
\newcommand{\limv}[2]{\lim\limits_{#1\rightarrow #2}}
\newcommand{\eb}{\mathbf{e}} % Standard basis
\newcommand{\comp}{\circ} % Function composition
\newcommand{\R}{{\mathbb R}} % The reals
\newcommand{\reals}{{\mathbb R}} % The reals
\newcommand{\integs}{{\mathbb Z}} % The integers
\newcommand{\cpxs}{{\mathbb C}} % The "complexes" :)
\newcommand{\setc}[2]{\left\{#1:\: #2\right\}}
\newcommand{\set}[1]{{\left\{#1\right\}}}
\newcommand{\cycle}[1]{\left(#1\right)}
\newcommand{\tuple}[1]{\left(#1\right)}
\newcommand{\Partial}[2]{\frac{\partial #1}{\partial #2}}
\newcommand{\PartialSl}[2]{\partial #1/\partial #2}
\newcommand{\funcsig}[2]{#1\rightarrow #2}
\newcommand{\funcdef}[3]{#1:\funcsig{#2}{#3}}
\newcommand{\supp}{\mathop{\mathrm{Supp}}} % Support of a function
\newcommand{\sgn}{\mathop{\mathrm{sgn}}} % Sign function
\newcommand{\tr}[1]{#1^\mathrm{tr}} % Transpose of a matrix
\newcommand{\inprod}[2]{\left<#1,#2\right>} % Inner product
\newenvironment{smallbmatrix}{\left[\begin{smallmatrix}}{\end{smallmatrix}\right]}
\newcommand{\maps}[2]{\mathop{\mathrm{Maps}}\left(#1,#2\right)}
\newcommand{\intoc}[2]{\left(#1,#2\right]}
\newcommand{\intco}[2]{\left[#1,#2\right)}
\newcommand{\intoo}[2]{\left(#1,#2\right)}
\newcommand{\intcc}[2]{\left[#1,#2\right]}
\newcommand{\transv}{\pitchfork}
\newcommand{\pair}[2]{\left\langle#1,#2\right\rangle}
\newcommand{\norm}[1]{\left\|#1\right\|}
\newcommand{\sqnorm}[1]{\left\|#1\right\|^2}
\newcommand{\bdry}{\partial}
\newcommand{\inv}[1]{#1^{-1}}
\newcommand{\tensor}{\otimes}
\newcommand{\bigtensor}{\bigotimes}
\newcommand{\im}{\operatorname{im}}
\newcommand{\coker}{\operatorname{im}}
\newcommand{\map}{\operatorname{Map}}
\newcommand{\crit}{\operatorname{Crit}}
\newcommand{\Th}{\operatorname{Th}}
\theorembodyfont{\upshape}
\newtheorem{thm}{Theorem}
\newtheorem{dthm}[thm]{Desired Theorem}
\newtheorem{cor}[thm]{Corollary}
\newtheorem{dcor}[thm]{Desired Corollary}
\newtheorem{lem}[thm]{Lemma}
\newtheorem{prop}[thm]{Proposition}
\newtheorem{defn}{Definition}
\newtheorem{rmk}{Remark}
\newtheorem{exm}{Example}
\newcommand{\cross}{\times}
\newcommand{\del}{\nabla}
\newcommand{\homeo}{\cong}
\newcommand{\isom}{\cong}
\newcommand{\htpyeq}{\backsimeq}
\newcommand{\codim}{\operatorname{codim}}
\newcommand{\projp}{{\mathbb R}P}

% open cells (not very nice...)
\newcommand{\oce}{\smash{\overset{\circ}e}} 
\newcommand{\ocD}{\smash{\overset{\circ}D}} 

\newcommand{\susp}{\Sigma}
\newcommand{\restr}[2]{{#1}|_{#2}}

\renewcommand{\hom}{\mathop{\mathrm{Hom}}} % Homomorphisms functor
\newcommand{\rp}{\reals P} % real projective space
\newcommand{\cp}{\cpxs P} % complex projective space
\newcommand{\zmod}[1]{\integs / #1\integs} % Z/nZ
\newcommand{\pt}{\mathrm{pt}}
\begin{document}
Let $h^*$ be a \PMlinkescapetext{multiplicative} generalized cohomology theory (for example, let $h^*=H^*$, singular cohomology with integer coefficients). Let $\xi\to X$ be a vector bundle of dimension $d$ over a topological space $X$. Assume for convenience that $\xi$ has a Riemannian metric, so that we may speak of its associated sphere and disk bundles, $S(\xi)$ and $D(\xi)$ respectively. 

\newcommand{\rh}{\tilde{h}}

Let $x\in X$, and consider the fibers $S(\xi_x)$ and $D(\xi_x)$. Since $D(\xi_x)/S(\xi_x)$ is homeomorphic to the $d$-sphere, the Eilenberg-Steenrod axioms for $h^*$ imply that $h^{*+d}(D(\xi_x),S(\xi_x))$ is isomorphic to the coefficient group $h^*(\pt)$ of $h^*$. In fact, $h^*(D(\xi_x),S(\xi_x))$ is a free module of rank one over the ring $h^*(\pt)$.

\begin{defn}
An element $\tau\in h^*(D(\xi),S(\xi))$ is said to be a \emph{Thom class} for $\xi$ if, for every $x\in X$, the restriction of $\tau$ to $h^*(D(\xi_x),S(\xi_x))$ is an $h^*(\pt)$-module generator.
\end{defn}

Note that $\tau$ lies necessarily in $h^d(D(\xi),S(\xi))$.

\begin{defn}
If a Thom class for $\xi$ exists, $\xi$ is said to be \emph{orientable} with respect to the cohomology theory $h^*$. 
\end{defn}

\begin{rmk}
Notice that we may consider $\tau$ as an element of the reduced $h^*$-cohomology group $\rh^*(X^\xi)$, where $X^\xi$ is the Thom space $D(\xi)/S(\xi)$ of $\xi$. As is the case in the definition of the Thom space, the Thom class may be defined without reference to associated disk and sphere bundles, and hence to a Riemannian metric on $\xi$. For example, the pair $(\xi,\xi-X)$ (where $X$ is included in $\xi$ as the zero section) is homotopy equivalent to $(D(\xi),S(\xi))$.
\end{rmk}

\begin{rmk}
If $h^*$ is singular cohomology with integer coefficients, then $\xi$ has a Thom class if and only if it is an orientable vector bundle in the ordinary sense, and the choices of Thom class are in one-to-one correspondence with the orientations.
\end{rmk}
%%%%%
%%%%%
\end{document}
