\documentclass[12pt]{article}
\usepackage{pmmeta}
\pmcanonicalname{ThomIsomorphismTheorem}
\pmcreated{2013-03-22 15:40:52}
\pmmodified{2013-03-22 15:40:52}
\pmowner{antonio}{1116}
\pmmodifier{antonio}{1116}
\pmtitle{Thom isomorphism theorem}
\pmrecord{6}{37622}
\pmprivacy{1}
\pmauthor{antonio}{1116}
\pmtype{Theorem}
\pmcomment{trigger rebuild}
\pmclassification{msc}{55-00}

\endmetadata

% used for TeXing text within eps files
%\usepackage{psfrag}
% need this for including graphics (\includegraphics)
%\usepackage{graphicx}
% for neatly defining theorems and propositions
%\usepackage{amsthm}
% making logically defined graphics
%%%\usepackage{xypic}

\usepackage{theorem}
\usepackage{amsmath}
\usepackage{amsfonts}
\usepackage{amssymb}
\newcommand{\limv}[2]{\lim\limits_{#1\rightarrow #2}}
\newcommand{\eb}{\mathbf{e}} % Standard basis
\newcommand{\comp}{\circ} % Function composition
\newcommand{\R}{{\mathbb R}} % The reals
\newcommand{\reals}{{\mathbb R}} % The reals
\newcommand{\integs}{{\mathbb Z}} % The integers
\newcommand{\cpxs}{{\mathbb C}} % The "complexes" :)
\newcommand{\setc}[2]{\left\{#1:\: #2\right\}}
\newcommand{\set}[1]{{\left\{#1\right\}}}
\newcommand{\cycle}[1]{\left(#1\right)}
\newcommand{\tuple}[1]{\left(#1\right)}
\newcommand{\Partial}[2]{\frac{\partial #1}{\partial #2}}
\newcommand{\PartialSl}[2]{\partial #1/\partial #2}
\newcommand{\funcsig}[2]{#1\rightarrow #2}
\newcommand{\funcdef}[3]{#1:\funcsig{#2}{#3}}
\newcommand{\supp}{\mathop{\mathrm{Supp}}} % Support of a function
\newcommand{\sgn}{\mathop{\mathrm{sgn}}} % Sign function
\newcommand{\tr}[1]{#1^\mathrm{tr}} % Transpose of a matrix
\newcommand{\inprod}[2]{\left<#1,#2\right>} % Inner product
\newenvironment{smallbmatrix}{\left[\begin{smallmatrix}}{\end{smallmatrix}\right]}
\newcommand{\maps}[2]{\mathop{\mathrm{Maps}}\left(#1,#2\right)}
\newcommand{\intoc}[2]{\left(#1,#2\right]}
\newcommand{\intco}[2]{\left[#1,#2\right)}
\newcommand{\intoo}[2]{\left(#1,#2\right)}
\newcommand{\intcc}[2]{\left[#1,#2\right]}
\newcommand{\transv}{\pitchfork}
\newcommand{\pair}[2]{\left\langle#1,#2\right\rangle}
\newcommand{\norm}[1]{\left\|#1\right\|}
\newcommand{\sqnorm}[1]{\left\|#1\right\|^2}
\newcommand{\bdry}{\partial}
\newcommand{\inv}[1]{#1^{-1}}
\newcommand{\tensor}{\otimes}
\newcommand{\bigtensor}{\bigotimes}
\newcommand{\im}{\operatorname{im}}
\newcommand{\coker}{\operatorname{im}}
\newcommand{\map}{\operatorname{Map}}
\newcommand{\crit}{\operatorname{Crit}}
\newcommand{\Th}{\operatorname{Th}}
\theorembodyfont{\upshape}
\newtheorem{thm}{Theorem}
\newtheorem{dthm}[thm]{Desired Theorem}
\newtheorem{cor}[thm]{Corollary}
\newtheorem{dcor}[thm]{Desired Corollary}
\newtheorem{lem}[thm]{Lemma}
\newtheorem{prop}[thm]{Proposition}
\newtheorem{defn}{Definition}
\newtheorem{rmk}{Remark}
\newtheorem{exm}{Example}
\newcommand{\cross}{\times}
\newcommand{\del}{\nabla}
\newcommand{\homeo}{\cong}
\newcommand{\isom}{\cong}
\newcommand{\htpyeq}{\backsimeq}
\newcommand{\codim}{\operatorname{codim}}
\newcommand{\projp}{{\mathbb R}P}

% open cells (not very nice...)
\newcommand{\oce}{\smash{\overset{\circ}e}} 
\newcommand{\ocD}{\smash{\overset{\circ}D}} 

\newcommand{\susp}{\Sigma}
\newcommand{\restr}[2]{{#1}|_{#2}}

\renewcommand{\hom}{\mathop{\mathrm{Hom}}} % Homomorphisms functor
\newcommand{\rp}{\reals P} % real projective space
\newcommand{\cp}{\cpxs P} % complex projective space
\newcommand{\zmod}[1]{\integs / #1\integs} % Z/nZ
\newcommand{\pt}{\mathrm{pt}}
\begin{document}
Let $\xi\to X$ be a $d$-dimensional vector bundle over a topological space $X$, and let $h^*$ be a multiplicative generalized cohomology theory, such as ordinary cohomology. Let $\tau\in h^d(D(\xi),S(\xi))$ be a Thom class for $\xi$, where $D(\xi)$ and $S(\xi)$ are the associated disk and sphere bundles of $\xi$. 

\newcommand{\rh}{\tilde{h}}

Since $h^*$ is a multiplicative theory, there is a generalized cup product map $$h^*(D(\xi))\tensor_{h^*}h^*(D(\xi),S(\xi))\to h^*(D(\xi),S(\xi)),$$ where the tensor product is over the coefficient ring $h^*(\pt)$ of the theory. Using the isomorphism $p^*:h^*(X)\isom h^*(D(\xi))$ induced by the homotopy equivalence $p:D(\xi)\to X$, we obtain a homomorphism 
$$T: h^n(X)\to h^{n+d}(D(\xi),S(\xi))\isom \rh^{n+d}(X^\xi) $$
taking $\alpha$ to $p^*(\alpha)\cdot\tau$. Here $X^\xi$ stands for the Thom space $D(\xi)/S(\xi)$ of $\xi$.

\medskip
\textbf{Thom isomorphism theorem}\quad $T$ is an isomorphism $h^*(X)\isom \rh^{*+d}(X^\xi)$ of graded modules over $h^*(\pt)$.
\medskip

\begin{rmk} When $\xi$ is a trivial bundle of dimension $1$, this generalizes the suspension isomorphism. In fact, a typical proof of this theorem for compact $X$ proceeds by induction over the number of open sets in a trivialization of $\xi$, using the suspension isomorphism as the base case and  the Mayer-Vietoris sequence to carry out the inductive step.
\end{rmk}

\begin{rmk}
There is also a homology Thom isomorphism $\rh_{*+d}(X^\xi)\isom h_*(X)$, in which the map is given by cap product with the Thom class rather than cup product.
\end{rmk}
%%%%%
%%%%%
\end{document}
