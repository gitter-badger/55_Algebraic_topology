\documentclass[12pt]{article}
\usepackage{pmmeta}
\pmcanonicalname{TopicEntryOnTopology}
\pmcreated{2013-03-22 17:59:57}
\pmmodified{2013-03-22 17:59:57}
\pmowner{rm50}{10146}
\pmmodifier{rm50}{10146}
\pmtitle{topic entry on topology}
\pmrecord{13}{40512}
\pmprivacy{1}
\pmauthor{rm50}{10146}
\pmtype{Topic}
\pmcomment{trigger rebuild}
\pmclassification{msc}{55-00}
\pmclassification{msc}{54-00}
\pmrelated{OverviewOfTheContentOfPlanetMath}

% this is the default PlanetMath preamble.  as your knowledge
% of TeX increases, you will probably want to edit this, but
% it should be fine as is for beginners.

% almost certainly you want these
\usepackage{amssymb}
\usepackage{amsmath}
\usepackage{amsfonts}

% used for TeXing text within eps files
%\usepackage{psfrag}
% need this for including graphics (\includegraphics)
%\usepackage{graphicx}
% for neatly defining theorems and propositions
%\usepackage{amsthm}
% making logically defined graphics
%%%\usepackage{xypic}

% there are many more packages, add them here as you need them

% define commands here

\begin{document}
\PMlinkescapeword{branch}
Topology is a branch of mathematics that arose from the study of geometry. Topology can be viewed as the study of spaces, and   more specifically as the study of those properties of spaces that are invariant under smooth deformation. It has been said that a topologist is a person who cannot tell a doughnut from a coffee cup, as both spaces have a single ``hole'', and each can be continuously deformed into the other.

A \PMlinkname{topological space}{TopologicalSpace} is a set of points together with a definition of which sets are open sets. This definition is called a \PMlinkname{topology}{TopologicalSpace} on the space. For example, any metric space can be seen as a topological space, where the prototypical open set is an open ball of any radius around a center.

There are several major branches of topology:
\begin{itemize}
\item \emph{\PMlinkescapetext{Point-set topology}} is the most basic study of topology. It investigates concepts such as connectedness, compactness, properties of different topologies, and topologies on subspaces and product spaces. See the articles list of common topologies and index of properties of topological spaces for more information.

\item \emph{\PMlinkname{Algebraic topology}{AlgebraicTopology}} studies topological spaces by attaching algebraic invariants to them. Key concepts here are those of homotopy, homology, and cohomology.

\item \emph{Geometric topology} studies manifolds (which are spaces that are locally Euclidean) and their embeddings into Euclidean space, including knot theory.

\item \emph{Differential topology} studies properties of manifolds with differentiable structures and differentiable maps on those manifolds.

\end{itemize}

Other general topological topics include
\begin{itemize}
\item Topics on low dimensional topology
\item Graph theory
\item Noncommutative topology
\item Generalizations of topological spaces
\end{itemize}

See also the attached bibliography for topology.
%%%%%
%%%%%
\end{document}
