\documentclass[12pt]{article}
\usepackage{pmmeta}
\pmcanonicalname{PoincareDuality}
\pmcreated{2013-03-22 13:11:36}
\pmmodified{2013-03-22 13:11:36}
\pmowner{mathcam}{2727}
\pmmodifier{mathcam}{2727}
\pmtitle{Poincar\'e duality}
\pmrecord{10}{33652}
\pmprivacy{1}
\pmauthor{mathcam}{2727}
\pmtype{Theorem}
\pmcomment{trigger rebuild}
\pmclassification{msc}{55M05}
\pmsynonym{Poincar\'e isomorphism}{PoincareDuality}
\pmrelated{DualityInMathematics}

\endmetadata

\usepackage{amssymb}
\usepackage{amsmath}
\usepackage{amsfonts}

\def\Z{\mathbb{Z}}
\begin{document}
If $M$ is a compact, oriented, $n$-dimensional manifold, then there is a canonical (though non-\PMlinkname{natural}{NaturalTransformation}) isomorphism
$$D:H^q(M,\Z)\to H_{n-q}(M,\Z)$$ (where $H^k(M,\Z)$ is the $k$th homology group of $M$ with integer coefficients and $H_k(M,\Z)$ the $k$th \PMlinkname{cohomology}{DeRhamCohomology} group) for all $q$, which is given by cap product with a generator of $H_n(M,\Z)$
(a choice of a generator here corresponds to an orientation).  This isomorphism exists with
coefficients in $\mathbb{Z}/2\mathbb{Z}$ regardless of orientation.

This isomorphism gives a nice interpretation to cup product.  If $X,Y$ are transverse submanifolds of $M$, then $X\cap Y$ is also a submanifold.  All of these submanifolds represent homology classes of $M$ in the appropriate dimensions, and $$D^{-1}([X])\cup D^{-1}([Y])=D^{-1}([X\cap Y]),$$ where $\cup$ is cup product, and $\cap$ in intersection, not cap product.
%%%%%
%%%%%
\end{document}
