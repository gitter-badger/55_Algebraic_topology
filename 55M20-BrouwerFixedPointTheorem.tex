\documentclass[12pt]{article}
\usepackage{pmmeta}
\pmcanonicalname{BrouwerFixedPointTheorem}
\pmcreated{2013-03-22 12:44:34}
\pmmodified{2013-03-22 12:44:34}
\pmowner{mathcam}{2727}
\pmmodifier{mathcam}{2727}
\pmtitle{Brouwer fixed point theorem}
\pmrecord{7}{33046}
\pmprivacy{1}
\pmauthor{mathcam}{2727}
\pmtype{Theorem}
\pmcomment{trigger rebuild}
\pmclassification{msc}{55M20}
\pmclassification{msc}{54H25}
\pmclassification{msc}{47H10}
%\pmkeywords{fixed point}
%\pmkeywords{nonconstructive}
\pmrelated{FixedPoint}
\pmrelated{SchauderFixedPointTheorem}
\pmrelated{TychonoffFixedPointTheorem}
\pmrelated{KKMlemma}
\pmrelated{KKMLemma}

\endmetadata

% this is the default PlanetMath preamble.  as your knowledge
% of TeX increases, you will probably want to edit this, but
% it should be fine as is for beginners.

% almost certainly you want these
\usepackage{amssymb}
\usepackage{amsmath}
\usepackage{amsfonts}

% used for TeXing text within eps files
%\usepackage{psfrag}
% need this for including graphics (\includegraphics)
%\usepackage{graphicx}
% for neatly defining theorems and propositions
%\usepackage{amsthm}
% making logically defined graphics
%%%\usepackage{xypic}

% there are many more packages, add them here as you need them

% define commands here

\newcommand{\Prob}[2]{\mathbb{P}_{#1}\left\{#2\right\}}
\newcommand{\norm}[1]{\left\|#1\right\|}

% Some sets
\newcommand{\Nats}{\mathbb{N}}
\newcommand{\Ints}{\mathbb{Z}}
\newcommand{\Reals}{\mathbb{R}}
\newcommand{\Complex}{\mathbb{C}}
\begin{document}
{\bf Theorem}
Let $\textbf{B}=\{x\in\Reals^n: \norm{x}\le 1\}$ be the closed unit ball in 
$\Reals^n$.  Any continuous function $f: \textbf{B}\to\textbf{B}$ has a fixed point.

\subsection*{Notes}

\begin{description}

\item[Shape is not important]
The theorem also applies to anything homeomorphic to a closed disk, of course.  In particular, we can replace $\textbf{B}$ in the formulation with a square or 
a triangle.

\item[Compactness counts (a)]
The theorem is not true if we drop a point from the interior of $\textbf{B}$.  For example, the map $f(\vec{x})=\frac{1}{2}\vec{x}$ has the single fixed point at $0$; dropping it from the domain yields a map with no \PMlinkname{fixed points}{FixedPoint}.

\item[Compactness counts (b)]
The theorem is not true for an open disk.  For instance, the map $f(\vec{x})=\frac{1}{2}\vec{x}+(\frac{1}{2},0,\ldots,0)$ has its single fixed point on the boundary of $\textbf{B}$.

\end{description}
%%%%%
%%%%%
\end{document}
