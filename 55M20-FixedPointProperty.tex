\documentclass[12pt]{article}
\usepackage{pmmeta}
\pmcanonicalname{FixedPointProperty}
\pmcreated{2013-03-22 13:56:32}
\pmmodified{2013-03-22 13:56:32}
\pmowner{yark}{2760}
\pmmodifier{yark}{2760}
\pmtitle{fixed point property}
\pmrecord{20}{34704}
\pmprivacy{1}
\pmauthor{yark}{2760}
\pmtype{Definition}
\pmcomment{trigger rebuild}
\pmclassification{msc}{55M20}
\pmclassification{msc}{54H25}
\pmclassification{msc}{47H10}
\pmsynonym{fixed-point property}{FixedPointProperty}
\pmrelated{FixedPoint}

\endmetadata

\usepackage{amssymb}
\usepackage{amsmath}
\usepackage{amsfonts}

\newcommand{\sC}{\mathbb{C}}
\newcommand{\sN}{\mathbb{N}}
\newcommand{\sR}{\mathbb{R}}
\newcommand{\sZ}{\mathbb{Z}}
\newcommand{\RP}{\mathbb{RP}}
\begin{document}
\PMlinkescapeword{between}
\PMlinkescapeword{square}
\PMlinkescapeword{states}
\PMlinkescapeword{unit}

Let $X$ be a topological space.
If every continuous function $f\colon X\to X$ has a 
\PMlinkname{fixed point}{FixedPoint}, 
then $X$ is said to have the \emph{fixed point property}.

The fixed point property is obviously preserved under homeomorphisms.
If $h\colon X\to Y$ is a homeomorphism between topological
spaces $X$ and $Y$, and $X$ has the fixed point property,
and $f\colon Y\to Y$ is continuous,
then $h^{-1}\circ f\circ h$ has a fixed point $x\in X$,
and $h(x)$ is a fixed point of $f$.

\section*{Examples}
\begin{enumerate}
\item A space with only one point has the fixed point property.
\item A closed interval $[a,b]$ of $\sR$ has the fixed point property.
\PMlinkname{This can be seen using the mean value theorem.}{BrouwerFixedPointInOneDimension}
\item The extended real numbers have the fixed point property,
as they are homeomorphic to $[0,1]$.
\item The topologist's sine curve has the fixed point property.
\item The real numbers $\sR$ do not have the fixed point property.
For example, the map $x\mapsto x+1$ on $\sR$ has no fixed point.
\item An open interval $(a,b)$ of $\sR$ does not have the fixed point property.
This follows since any such interval is homeomorphic to $\sR$.
Similarly, an open ball in $\sR^n$ does not have the fixed point property.
\item Brouwer's Fixed Point Theorem states that in $\sR^n$,
the closed unit ball with the subspace topology has the fixed point property.
(Equivalently, $[0,1]^n$ has the fixed point property.)
The Schauder Fixed Point Theorem generalizes this result further.
\item For each $n\in\sN$, the real projective space $\RP^{2n}$ has the fixed point property.
\item Every simply-connected plane continuum has the fixed-point property.
\item The Alexandroff--Urysohn square (also known as the Alexandroff square)
has the fixed point property.
\end{enumerate}

\section*{Properties}
\begin{enumerate}
\item \PMlinkname{Any topological space with the fixed point property is connected}{AnyTopologicalSpaceWithTheFixedPointPropertyIsConnected}
and \PMlinkname{$\operatorname{T}_0$}{T0Space}.
\item Suppose $X$ is a topological space with the
fixed point property, and $Y$ is a retract of $X$. Then
$Y$ has the fixed point property.
\item Suppose $X$ and $Y$ are topological spaces, and $X\times Y$ has the
fixed point property. Then $X$ and $Y$ have the fixed point property.
(Proof: If $f\colon X\to X$ is continuous,
then $(x,y)\mapsto (f(x),y)$ is continuous, so $f$ has a fixed point.)
\end{enumerate}

\begin{thebibliography}{9}
 \bibitem{naber} G.\ L.\ Naber, \emph{Topological methods in Euclidean spaces},
   Cambridge University Press, 1980.
 \bibitem{jameson} G.\ J.\ Jameson, \emph{Topology and Normed Spaces},
   Chapman and Hall, 1974.
 \bibitem{ward} L.\ E.\ Ward, \emph{Topology, An Outline for a First Course},
   Marcel Dekker, Inc., 1972.
 \bibitem{hagopian} Charles Hagopian,
   The Fixed-Point Property for simply-connected plane continua,
   Trans. Amer. Math. Soc. 348 (1996) 4525--4548.
 \end{thebibliography}
%%%%%
%%%%%
\end{document}
