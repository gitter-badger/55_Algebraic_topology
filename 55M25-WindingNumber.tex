\documentclass[12pt]{article}
\usepackage{pmmeta}
\pmcanonicalname{WindingNumber}
\pmcreated{2013-03-22 12:56:06}
\pmmodified{2013-03-22 12:56:06}
\pmowner{CWoo}{3771}
\pmmodifier{CWoo}{3771}
\pmtitle{winding number}
\pmrecord{8}{33291}
\pmprivacy{1}
\pmauthor{CWoo}{3771}
\pmtype{Definition}
\pmcomment{trigger rebuild}
\pmclassification{msc}{55M25}
\pmclassification{msc}{30A99}
\pmsynonym{index}{WindingNumber}

\usepackage{amssymb}
\usepackage{amsmath}
\usepackage{amsfonts}
\begin{document}
\PMlinkescapeword{net}
\PMlinkescapeword{theory}
\PMlinkescapeword{definable}
\PMlinkescapeword{integral}
%30A99
Winding numbers are a basic notion in algebraic topology, and play an
important role in connection with analytic functions of a complex variable.
Intuitively, given a closed curve $t\mapsto S(t)$ in an oriented
Euclidean plane (such as the complex plane $\mathbb{C}$), and a point
$p$ not in the image of $S$, the winding number (or index) of $S$ with respect
to $p$ is the net number of times $S$ surrounds $p$. It is not altogether
easy to make this notion rigorous.

Let us take $\mathbb{C}$ for the plane. We have
a continuous mapping $S:[a,b]\to \mathbb{C}$ where $a$ and $b$ are some
reals with $a<b$ and $S(a)=S(b)$. Denote by $\theta(t)$ the angle from the
positive real axis to the ray from $z_0$ to $S(t)$. As $t$ moves from $a$ to
$b$, we expect $\theta$ to increase or decrease by a multiple of $2\pi$,
namely $2\omega\pi$ where $\omega$ is the winding number. One therefore thinks
of using integration. And indeed, in the theory of
functions of a complex variable, it is proved that the value
$$\frac{1}{2\pi i} \int_S \frac{dz}{z-z_0}$$
is an integer and has the expected properties of a winding number around
$z_0$. To define the winding number in this way, we need to assume
that the closed path $S$ is rectifiable (so that the path
integral is defined). An equivalent condition is that the real and imaginary
parts of the function $S$ are of bounded variation.

But if $S$ is any continuous mapping $[a,b]\to \mathbb{C}$ having
$S(a)=S(b)$, the winding number is still definable, without any integration.
We can break up the domain of $S$ into a finite number of intervals such that
the image of $S$, on any of those intervals, is contained in a disc which
does not contain $z_0$. Then $2\omega\pi$ emerges as a finite sum: the sum
of the angles subtended at $z_0$ by the sides of a polygon.

Let $A$, $B$, and $C$ be any three distinct rays from $z_0$.
The three sets
$$S^{-1}(A)\qquad S^{-1}(B)\qquad S^{-1}(C)$$
are closed in $[a,b]$, and they \emph{determine}
the winding number of $S$ around $z_0$. This result can provide an alternative
definition of winding numbers in $\mathbb{C}$, and a definition in some other
spaces also, but the details are rather subtle.

For one more variation on the theme, let $S$ be any topological space
homeomorphic to a circle, and let
$f:S\to S$ be any continuous mapping. Intuitively we expect that if a point
$x$ travels once around $S$, the point $f(x)$ will travel around $S$ some
integral number of times, say $n$ times. The notion can be made precise.
Moreover, the number $n$ is determined by the three closed sets
$$f^{-1}(a)\qquad f^{-1}(b)\qquad f^{-1}(c)$$
where $a$, $b$, and $c$ are any three distinct points in $S$.
%%%%%
%%%%%
\end{document}
