\documentclass[12pt]{article}
\usepackage{pmmeta}
\pmcanonicalname{LusternikSchnirelmannCategory}
\pmcreated{2013-03-22 15:53:30}
\pmmodified{2013-03-22 15:53:30}
\pmowner{juanman}{12619}
\pmmodifier{juanman}{12619}
\pmtitle{Lusternik-Schnirelmann category}
\pmrecord{13}{37893}
\pmprivacy{1}
\pmauthor{juanman}{12619}
\pmtype{Definition}
\pmcomment{trigger rebuild}
\pmclassification{msc}{55M30}
%\pmkeywords{Contractible open cover}
%\pmkeywords{Morse function}
\pmrelated{Topology}
\pmrelated{RoundComplexity}

% this is the default PlanetMath preamble.  as your knowledge
% of TeX increases, you will probably want to edit this, but
% it should be fine as is for beginners.

% almost certainly you want these
\usepackage{amssymb}
\usepackage{amsmath}
\usepackage{amsfonts}

% used for TeXing text within eps files
%\usepackage{psfrag}
% need this for including graphics (\includegraphics)
%\usepackage{graphicx}
% for neatly defining theorems and propositions
%\usepackage{amsthm}
% making logically defined graphics
%%%\usepackage{xypic}

% there are many more packages, add them here as you need them

% define commands here

\begin{document}
Let $X$ be a topological space.  An important topological invariant of $X$ called Lusternik-Schnirelmann category {\bf cat} is defined as follows:

$${\rm cat}(X)={\rm min}\{\#(C)\colon\mbox{where $C$ are the coverings of $X$ by contractible open sets}\}.$$

If $X$ is a manifold, ${\rm cat}(X)$ coincides with the minimal number of critical points among all smooth scalars maps $X\to\mathbb{R}$. 

This is equivalent to saying that $X$ has a covering $\{U_s\}$ such that
it is posible to factor homotopically each $U_s\stackrel{i}\hookrightarrow X$ through $U_s\stackrel{a}\to *\stackrel{b}\to X$ i.e
$$i\simeq b\circ a.$$

This allows us to define another category, e.g.:\\
\begin{quote}
We can ask about the minimal number of open sets $U_s$ that cover $X$ and 
are homotopically equivalent to $S^1$, say, 
the inclusion  $U_s\stackrel{i}\hookrightarrow X$ and $U_s\stackrel{a}\to S^1\stackrel{b}\to X$ are $i\simeq b\circ a$.
\end{quote}

It is becoming standard to speak of the t-cat of $X$. 
This is related to the round complexity of the space.  


\begin{thebibliography}{1}
\bibitem{cite:RHF}
R.H. Fox, {\it On the Lusternik-Schnirelmann category}, Annals of Math. 42 (1941), 333-370.
\bibitem{cite:FT}
F. Takens, {\it The minimal number of critical points of a function on compact manifolds and the Lusternik-Schnirelmann category}, Invent. math. 6,(1968), 197-244.

\end{thebibliography}

%%%%%
%%%%%
\end{document}
