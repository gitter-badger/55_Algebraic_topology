\documentclass[12pt]{article}
\usepackage{pmmeta}
\pmcanonicalname{StandardNball}
\pmcreated{2013-03-22 16:52:49}
\pmmodified{2013-03-22 16:52:49}
\pmowner{Mathprof}{13753}
\pmmodifier{Mathprof}{13753}
\pmtitle{standard n-ball}
\pmrecord{6}{39133}
\pmprivacy{1}
\pmauthor{Mathprof}{13753}
\pmtype{Definition}
\pmcomment{trigger rebuild}
\pmclassification{msc}{55M99}
\pmsynonym{unit n-ball}{StandardNball}
\pmsynonym{unit ball}{StandardNball}
\pmsynonym{standard $n$-ball}{StandardNball}
\pmdefines{n-cell}

\endmetadata

% this is the default PlanetMath preamble.  as your knowledge
% of TeX increases, you will probably want to edit this, but
% it should be fine as is for beginners.

% almost certainly you want these
\usepackage{amssymb}
\usepackage{amsmath}
\usepackage{amsfonts}

% used for TeXing text within eps files
%\usepackage{psfrag}
% need this for including graphics (\includegraphics)
%\usepackage{graphicx}
% for neatly defining theorems and propositions
%\usepackage{amsthm}
% making logically defined graphics
%%%\usepackage{xypic}

% there are many more packages, add them here as you need them

% define commands here

\begin{document}
The \emph{standard n-ball} is $B^n = \{x \in \mathbb{R}^n | \quad ||x|| \le 1 \}$,
where $||x||^2 = \sum_{i=1}^n {x_i}^2 $.
The standard n-ball is also called the \emph{unit n-ball} or the
\emph{n-disk}.

The notation in the literature for the standard n-ball is not uniform.
Sometimes it is denoted by $V^n$ , $D^n$ ,$E^n$ or ${\overline{E}}^n$.

The boundary of the standard n-ball is $S^{n-1}$, the n-1-sphere.

A space that is homeomorphic to the standard n-ball is called
an \emph{n-cell}.

A 1-cell is called an arc and a 2-cell is called a disk.
%%%%%
%%%%%
\end{document}
