\documentclass[12pt]{article}
\usepackage{pmmeta}
\pmcanonicalname{DeRhamCohomology}
\pmcreated{2013-03-22 14:24:40}
\pmmodified{2013-03-22 14:24:40}
\pmowner{pbruin}{1001}
\pmmodifier{pbruin}{1001}
\pmtitle{de Rham cohomology}
\pmrecord{9}{35913}
\pmprivacy{1}
\pmauthor{pbruin}{1001}
\pmtype{Definition}
\pmcomment{trigger rebuild}
\pmclassification{msc}{55N05}
\pmclassification{msc}{58A12}
\pmdefines{de Rham cohomology group}

\endmetadata

% this is the default PlanetMath preamble.  as your knowledge
% of TeX increases, you will probably want to edit this, but
% it should be fine as is for beginners.

% almost certainly you want these
\usepackage{amssymb}
\usepackage{amsmath}
\usepackage{amsfonts}

% used for TeXing text within eps files
%\usepackage{psfrag}
% need this for including graphics (\includegraphics)
%\usepackage{graphicx}
% for neatly defining theorems and propositions
%\usepackage{amsthm}
% making logically defined graphics
%%%\usepackage{xypic}

% there are many more packages, add them here as you need them

% define commands here
\newcommand{\HdR}{{\rm H}_{\rm dR}}
\newcommand{\im}{\mathop{\mathrm{im}}}
\begin{document}
Let $X$ be a paracompact ${\cal C}^\infty$ differential manifold.  Let
$$
\Omega X=\bigoplus_{i=0}^\infty\Omega^i X
$$
denote the graded-commutative $\mathbb{R}$-algebra of differential forms on $X$.  Together with the exterior derivative
$$
d^i\colon\Omega^i X\to\Omega^{i+1}X\quad(i=0,1,\ldots),
$$
$\Omega X$ forms a chain complex $(\Omega X,d)$ of $\mathbb{R}$-vector spaces.  The \PMlinkescapetext{{\it de Rham cohomology groups}} ${\rm H}_{\rm dR}^i X$ of $X$ are defined as the homology groups of this complex, that is to say
$$
{\rm H}_{\rm dR}^i X:=(\ker d^i)/(\im d^{i-1})\quad(i=0,1,\ldots),
$$
where $\Omega^{-1}X$ is taken to be 0, so $d^{-1}\colon 0\to\Omega^0 X$ is the zero map.  The wedge product in $\Omega X$ induces the structure of a graded-commutative $\mathbb{R}$-algebra on
$$
{\rm H}_{\rm dR}X:=\bigoplus_{i=0}^{\infty}\HdR^i X.
$$

If $X$ and $Y$ are both paracompact ${\cal C}^\infty$ manifolds and $f\colon X\to Y$ is a differentiable map, there is an induced map
$$
f^*\colon \HdR Y\to\HdR X,
$$
defined by
$$
f^*[\omega]:=[f^*\omega]\quad\hbox{for $\omega\in\ker d$}.
$$
Here $[\omega]$ denotes the class of $\omega$ modulo $\im d$, and the second $f^*$ is the map $\Omega Y\to\Omega X$ induced by the functor $\Omega$.  This action on differentiable maps makes the de Rham cohomology into a contravariant functor from the category of paracompact ${\cal C}^\infty$ manifolds to the category of graded-commutative $\mathbb{R}$-algebras.  It turns out to be homotopy invariant; this implies that homotopy equivalent manifolds have isomorphic de Rham cohomology.
%%%%%
%%%%%
\end{document}
