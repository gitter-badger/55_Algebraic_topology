\documentclass[12pt]{article}
\usepackage{pmmeta}
\pmcanonicalname{HomologyOfmathbbRP3}
\pmcreated{2013-03-22 13:50:23}
\pmmodified{2013-03-22 13:50:23}
\pmowner{mathcam}{2727}
\pmmodifier{mathcam}{2727}
\pmtitle{homology of $\mathbb{RP}^3$.}
\pmrecord{8}{34576}
\pmprivacy{1}
\pmauthor{mathcam}{2727}
\pmtype{Example}
\pmcomment{trigger rebuild}
\pmclassification{msc}{55N10}
\pmrelated{HomologicalComplexOfTopologicalVectorSpaces}

% this is the default PlanetMath preamble.  as your knowledge
% of TeX increases, you will probably want to edit this, but
% it should be fine as is for beginners.

% almost certainly you want these
\usepackage{amssymb}
\usepackage{amsmath}
\usepackage{amsfonts}
\usepackage{amsthm}
\usepackage[all]{xy}

% used for TeXing text within eps files
%\usepackage{psfrag}
% need this for including graphics (\includegraphics)
%\usepackage{graphicx}
% for neatly defining theorems and propositions
%\usepackage{amsthm}
% making logically defined graphics
%%%\usepackage{xypic}

% there are many more packages, add them here as you need them

% define commands here

\newcommand{\mc}{\mathcal}
\newcommand{\mb}{\mathbb}
\newcommand{\mf}{\mathfrak}
\newcommand{\ol}{\overline}
\newcommand{\ra}{\rightarrow}
\newcommand{\la}{\leftarrow}
\newcommand{\La}{\Leftarrow}
\newcommand{\Ra}{\Rightarrow}
\newcommand{\nor}{\vartriangleleft}
\newcommand{\Gal}{\text{Gal}}
\newcommand{\GL}{\text{GL}}
\newcommand{\Z}{\mb{Z}}
\newcommand{\R}{\mb{R}}
\newcommand{\Q}{\mb{Q}}
\newcommand{\C}{\mb{C}}
\newcommand{\<}{\langle}
\renewcommand{\>}{\rangle}
\begin{document}
We need for this problem knowledge of the homology groups of $S^2$ and $\mb{RP}^2$.  We will simply assume the former:
\begin{align*}
H_k(S^2; \Z)&= 
\begin{cases}
\Z&k=0,2\\
0&else
\end{cases}
\end{align*}

Now, for $\mb{RP}^2$, we can argue without Mayer-Vietoris. $X=\mb{RP}^2$ is connected, so $H_0(X;\Z)=\Z$.  $X$ is non-orientable, so $H_2(X;\Z)$ is 0.  Last, $H_1(X;\Z)$ is the abelianization of the already abelian fundamental group $\pi_1(X)=\Z/2\Z$, so we have:
\begin{align*}
H_k(\mb{RP}^2; \Z)&=
\begin{cases}
\Z&k=0\\
\Z/2\Z&k=1\\
0&k\geq 2
\end{cases}\\
\end{align*}

Now that we have the homology of $\mb{RP}^2$, we can compute the 
homology of $\mb{RP}^3$ from Mayer-Vietoris.  Let $X=\mb{RP}^3$,
$V=\mb{RP}^3\backslash\{pt\}\sim\mb{RP}^2$ (by vieweing $\mb{RP}^3$ as a CW-complex), $U\sim D^3\sim\{pt\}$, and $U\cap V\sim S^2$, where $\sim$ denotes equivalence through a deformation retract.  Then the Mayer-Vietoris sequence gives

$$\xymatrix{
&0\ar[r]& H_3(X;\Z)\ar `r_l[ll] `l^d[dll] [dll]\\
H_2(S^2;\Z)\ar[r] &  H_2(pt;\Z)\oplus H_2(\mb{RP}^2;\Z)\ar[r] & H_2(X;\Z)\ar `r_l[ll] `l^d[dll] [dll]\\
H_1(S^2;\Z)\ar[r] & H_1(pt;\Z)\oplus H_1(\mb{RP}^2;\Z)\ar[r] & H_1(X;\Z)\ar `r_l[ll] `l^d[dll] [dll]\\
H_0(S^2;\Z) \ar[r] &  H_0(pt;\Z)\oplus H_0(\mb{RP}^2;\Z)  \ar[r] & H_0(X;\Z)\ar[r] &0}$$

From here, we substitute in the \PMlinkescapetext{information} from above, and use the fact that the $k$-th homology group of an $n$-dimensional object is 0 if $k>n$, and begin to compute using the \PMlinkescapetext{theory} of short exact sequences.  Since we have as a \PMlinkescapetext{subsequence} the short exact sequence $0\ra H_3(X;\Z)\ra\Z\ra0$, we can conclude $H_3(X;\Z)=\Z$.  Since we have as a \PMlinkescapetext{subsequence} the short exact sequence $0\ra H_2(X;\Z)\ra 0$, we can conclude $H_2(X;\Z)=0.$  Since the bottom \PMlinkescapetext{sequence} splits, we get $0\ra\Z/2\Z\ra H_1(X;\Z)\ra 0$, so that $H_1(X;\Z)=\Z/2\Z$.  We thus conclude that

\begin{align*}
H_k(\mb{RP}^3; \Z)&=
\begin{cases}
\Z&k=0\\
\Z/2\Z&k=1\\
0&k=2\\
\Z&k=3\\
0&k>3\\
\end{cases}
\end{align*}
%%%%%
%%%%%
\end{document}
