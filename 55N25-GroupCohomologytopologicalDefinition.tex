\documentclass[12pt]{article}
\usepackage{pmmeta}
\pmcanonicalname{GroupCohomologytopologicalDefinition}
\pmcreated{2013-03-22 14:32:24}
\pmmodified{2013-03-22 14:32:24}
\pmowner{whm22}{2009}
\pmmodifier{whm22}{2009}
\pmtitle{group cohomology (topological definition)}
\pmrecord{18}{36085}
\pmprivacy{1}
\pmauthor{whm22}{2009}
\pmtype{Definition}
\pmcomment{trigger rebuild}
\pmclassification{msc}{55N25}
%\pmkeywords{cohomology}
%\pmkeywords{topological group}
%\pmkeywords{classifying space}
\pmrelated{CohomologyGroupTheorem}
\pmdefines{group cohomology}
\pmdefines{classifying spaces}

\endmetadata

% this is the default PlanetMath preamble.  as your knowledge
% of TeX increases, you will probably want to edit this, but
% it should be fine as is for beginners.

% almost certainly you want these
\usepackage{amssymb}
\usepackage{amsmath}
\usepackage{amsfonts}

% used for TeXing text within eps files
%\usepackage{psfrag}
% need this for including graphics (\includegraphics)
%\usepackage{graphicx}
% for neatly defining theorems and propositions
%\usepackage{amsthm}
% making logically defined graphics
%%%\usepackage{xypic}

% there are many more packages, add them here as you need them

% define commands here
\begin{document}
Let $G$ be a topological group.  Suppose some contractible space $X$ admits a fixed point free action of $G$, so that the quotient map $p:X \rightarrow X/G$ is a fibre map.  Then $X/G$, denoted $BG$ is called the classifying space of $G$.  Classifying spaces always exist and are unique up to homotopy.  Further, if $G$ has the structure of a CW- complex, we can choose $BG$ to have one too.

The group (co)homology of $G$ is defined to be the  (co)homology of $BG$.  From the long-exact sequence associated to the fibre map, $p$, we know that $\pi_{n} (G) = \pi_{n+1}(BG)$ for $n \geq 0$.  In particular the fundamental group of $BG$ is $\pi_0(G)$, which inherits a group structure as a quotient of $G$. Let $H$ denote $\pi_0(G)$.  Then $H$ acts freely on the cells of $BG^{*}$, the universal over of $BG$.  Hence the cellular resolution for $BG^{*}$, denoted, $C_*(BG^{*})$, is a sequence of free $ZH$- modules and $ZH$- linear maps.  Taking coefficients in some $ZH$- module $A$, we have 

$$
H^n(G; A)=H^n(C_{*}(BG^{*}); A) \,\,\rm{ and }\,\, H_n(G; A)=H_n(C_{*}(BG^{*}); A)
$$

In particular, when $G$ is discrete, $p$ must be the covering map associated to a universal cover.  Hence $X=BG^{*}$ and $C_*(BG^{*})$ is exact, as $X$ is contractible and hence has trivial homology.  Note in this case $H=G$.  So for a discrete group $G$, we have, 

$$
H^n(G; A)=Ext^n_{ZG}(Z, A) \,\,\rm{  and  }\,\, H_n(G; A)= Tor^n_{ZG}(Z, A)
$$

Also, as passing to the universal cover preserves $\pi_{n}$ for  $n > 1$, we know that $\pi_n (BG)=0$ for $n > 1$.  $BG$ is always connected and for a discrete group $\pi_0(G)=G$ so we have $BG = k(G,1)$, the Eilenberg - Maclane space.

As an example take $G = SU_1$.  Note topologically, $SU_1=S^1 = k(Z, 1)$.  As $\pi_n (G) = \pi_{n+1}(BG)$ for  $n \geq 0$, we know that $BSU_1= k(Z, 2)= CP^\infty$.

More explicitly, we may identify $SU_1$ with the unit complex numbers.  This acts freely on the infinite complex sphere (which is contractible) leaving a quotient of $CP^\infty$.  

Hence $H^n(SU_1,Z)=Z$ if $2$ divides $n$ and $0$ otherwise.

Similiarly $BC_2= RP^\infty$ and $BSU_2 = HP^\infty$, as $C_2$ and $SU_2$ are isomorphic to U(R) and U(H) respectively.    So $H^n(C_2, Z_2)=Z_2$ for all $n$ and $H^n(SU_2,Z)=Z$ if $4$ divides $n$ and $0$ otherwise.
%%%%%
%%%%%
\end{document}
