\documentclass[12pt]{article}
\usepackage{pmmeta}
\pmcanonicalname{ExampleOfCohomologyAndMayerVietorisSequence}
\pmcreated{2013-03-22 19:13:27}
\pmmodified{2013-03-22 19:13:27}
\pmowner{joking}{16130}
\pmmodifier{joking}{16130}
\pmtitle{example of cohomology and Mayer-Vietoris sequence}
\pmrecord{7}{42145}
\pmprivacy{1}
\pmauthor{joking}{16130}
\pmtype{Example}
\pmcomment{trigger rebuild}
\pmclassification{msc}{55N33}

\endmetadata

% this is the default PlanetMath preamble.  as your knowledge
% of TeX increases, you will probably want to edit this, but
% it should be fine as is for beginners.

% almost certainly you want these
\usepackage{amssymb}
\usepackage{amsmath}
\usepackage{amsfonts}

% used for TeXing text within eps files
%\usepackage{psfrag}
% need this for including graphics (\includegraphics)
%\usepackage{graphicx}
% for neatly defining theorems and propositions
%\usepackage{amsthm}
% making logically defined graphics
%%%\usepackage{xypic}

% there are many more packages, add them here as you need them

% define commands here

\begin{document}
Consider n-dimensional sphere $S^n = \{v\in R^{n+1}| \quad |v|=1\}$.

Let $A = \{ (x_0,...,x_n)\in S^n| \quad x_0>-1/2 \}$ and
$B = \{ (x_0,...,x_n)\in S^n| \quad x_0<1/2 \}$.

Of course both $A$ and $B$ are open (in $S^n$) and their union is $S^n$. Furthermore, it can be easily seen, that their intersection can be contracted into "big circle", i.e. $A\cap B$ has homotopy type of $S^{n-1}$. Also both $A$ and $B$ are contractible (they are  homeomorphic to $R^n$ via stereographic projection). So, write part of a Meyer-Vietoris sequence (for the cohomology $H^m(X)=H^m(X,G)$, where $G$ is a fixed Abelian group):

$\cdots \rightarrow H^m(A) \oplus H^m(B) \rightarrow H^m(A\cap B)\rightarrow H^{m+1}(S^n)\rightarrow H^{m+1}(A) \oplus H^{m+1}(B)\rightarrow \cdots$

Since both $A$ and $B$ are contractible and $A\cap B$ is homotopic to $S^{n-1}$,  we have the following short exact sequence:

$0 \rightarrow H^m(S^{n-1}) \rightarrow  H^{m+1}(S^n) \rightarrow  0$

which shows that $H^m(S^{n-1})$ is isomorphic to $H^m(S^n)$ for every  $n>0$ and $m>0$. So, in order to calculate cohomology groups of spheres, we only need to know the cohomology groups  of $S^1$. And those can be also calculated, if we once again apply previous schema. Note, that in the case of $S^1$ we have that $A\cap B$ has the homotopy type of a discrete space with two points. Therefore all their cohomology groups are trivial, except for $H^0$ (which can be easily calculated to be equal to $H^0(*) \oplus H^0(*)$, where * is a one-pointed space). 

This schema can be used for other spaces like the torus (which can be also calculated from Kunneth's formula).
%%%%%
%%%%%
\end{document}
