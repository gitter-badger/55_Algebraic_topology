\documentclass[12pt]{article}
\usepackage{pmmeta}
\pmcanonicalname{TangentialCauchyRiemannComplexOfSmoothForms}
\pmcreated{2013-03-22 18:24:18}
\pmmodified{2013-03-22 18:24:18}
\pmowner{bci1}{20947}
\pmmodifier{bci1}{20947}
\pmtitle{tangential Cauchy-Riemann complex of smooth forms}
\pmrecord{18}{41052}
\pmprivacy{1}
\pmauthor{bci1}{20947}
\pmtype{Definition}
\pmcomment{trigger rebuild}
\pmclassification{msc}{55N33}
\pmclassification{msc}{13D25}
\pmclassification{msc}{55R65}
\pmclassification{msc}{22A22}
\pmclassification{msc}{32V05}
\pmclassification{msc}{32V99}
\pmsynonym{tangential Cauchy-Riemann complexes}{TangentialCauchyRiemannComplexOfSmoothForms}
%\pmkeywords{tangential Cauchy-Riemann complexes of smooth forms}
\pmrelated{CRSubmanifold}
\pmrelated{GenericManifold}
\pmrelated{CohomologyGroupTheorem}
\pmrelated{CohomologicalComplexOfTopologicalVectorSpaces}
\pmrelated{GrassmanHopfAlgebrasAndTheirDualCoAlgebras}
\pmrelated{CategoricalSequence}
\pmrelated{ExactSequence2}
\pmrelated{HomologicalComplexOfTopologicalVectorSpaces}
\pmrelated{CauchyRiemannEquations}
\pmrelated{Ca}
\pmdefines{CR complex}
\pmdefines{tangential Cauchy-Riemann complex of $C^{\infty)$-smooth forms}

\endmetadata

% this is the default PlanetMath preamble.  as your knowledge
% of TeX increases, you will probably want to edit this, but
% it should be fine as is for beginners.

% almost certainly you want these
\usepackage{amssymb}
\usepackage{amsmath}
\usepackage{amsfonts}

% used for TeXing text within eps files
%\usepackage{psfrag}
% need this for including graphics (\includegraphics)
%\usepackage{graphicx}
% for neatly defining theorems and propositions
%\usepackage{amsthm}
% making logically defined graphics
%%%\usepackage{xypic}

% there are many more packages, add them here as you need them

% define commands here
\usepackage{amsmath, amssymb, amsfonts, amsthm, amscd, latexsym}
%%\usepackage{xypic}
\usepackage[mathscr]{eucal}

\setlength{\textwidth}{6.5in}
%\setlength{\textwidth}{16cm}
\setlength{\textheight}{9.0in}
%\setlength{\textheight}{24cm}

\hoffset=-.75in     %%ps format
%\hoffset=-1.0in     %%hp format
\voffset=-.4in

\theoremstyle{plain}
\newtheorem{lemma}{Lemma}[section]
\newtheorem{proposition}{Proposition}[section]
\newtheorem{theorem}{Theorem}[section]
\newtheorem{corollary}{Corollary}[section]

\theoremstyle{definition}
\newtheorem{definition}{Definition}[section]
\newtheorem{example}{Example}[section]
%\theoremstyle{remark}
\newtheorem{remark}{Remark}[section]
\newtheorem*{notation}{Notation}
\newtheorem*{claim}{Claim}

\renewcommand{\thefootnote}{\ensuremath{\fnsymbol{footnote%%@
}}}
\numberwithin{equation}{section}

\newcommand{\Ad}{{\rm Ad}}
\newcommand{\Aut}{{\rm Aut}}
\newcommand{\Cl}{{\rm Cl}}
\newcommand{\Co}{{\rm Co}}
\newcommand{\DES}{{\rm DES}}
\newcommand{\Diff}{{\rm Diff}}
\newcommand{\Dom}{{\rm Dom}}
\newcommand{\Hol}{{\rm Hol}}
\newcommand{\Mon}{{\rm Mon}}
\newcommand{\Hom}{{\rm Hom}}
\newcommand{\Ker}{{\rm Ker}}
\newcommand{\Ind}{{\rm Ind}}
\newcommand{\IM}{{\rm Im}}
\newcommand{\Is}{{\rm Is}}
\newcommand{\ID}{{\rm id}}
\newcommand{\GL}{{\rm GL}}
\newcommand{\Iso}{{\rm Iso}}
\newcommand{\Sem}{{\rm Sem}}
\newcommand{\St}{{\rm St}}
\newcommand{\Sym}{{\rm Sym}}
\newcommand{\SU}{{\rm SU}}
\newcommand{\Tor}{{\rm Tor}}
\newcommand{\U}{{\rm U}}

\newcommand{\A}{\mathcal A}
\newcommand{\Ce}{\mathcal C}
\newcommand{\D}{\mathcal D}
\newcommand{\E}{\mathcal E}
\newcommand{\F}{\mathcal F}
\newcommand{\G}{\mathcal G}
\newcommand{\Q}{\mathcal Q}
\newcommand{\R}{\mathcal R}
\newcommand{\cS}{\mathcal S}
\newcommand{\cU}{\mathcal U}
\newcommand{\W}{\mathcal W}

\newcommand{\bA}{\mathbb{A}}
\newcommand{\bB}{\mathbb{B}}
\newcommand{\bC}{\mathbb{C}}
\newcommand{\bD}{\mathbb{D}}
\newcommand{\bE}{\mathbb{E}}
\newcommand{\bF}{\mathbb{F}}
\newcommand{\bG}{\mathbb{G}}
\newcommand{\bK}{\mathbb{K}}
\newcommand{\bM}{\mathbb{M}}
\newcommand{\bN}{\mathbb{N}}
\newcommand{\bO}{\mathbb{O}}
\newcommand{\bP}{\mathbb{P}}
\newcommand{\bR}{\mathbb{R}}
\newcommand{\bV}{\mathbb{V}}
\newcommand{\bZ}{\mathbb{Z}}

\newcommand{\bfE}{\mathbf{E}}
\newcommand{\bfX}{\mathbf{X}}
\newcommand{\bfY}{\mathbf{Y}}
\newcommand{\bfZ}{\mathbf{Z}}

\renewcommand{\O}{\Omega}
\renewcommand{\o}{\omega}
\newcommand{\vp}{\varphi}
\newcommand{\vep}{\varepsilon}

\newcommand{\diag}{{\rm diag}}
\newcommand{\grp}{{\mathbb G}}
\newcommand{\dgrp}{{\mathbb D}}
\newcommand{\desp}{{\mathbb D^{\rm{es}}}}
\newcommand{\Geod}{{\rm Geod}}
\newcommand{\geod}{{\rm geod}}
\newcommand{\hgr}{{\mathbb H}}
\newcommand{\mgr}{{\mathbb M}}
\newcommand{\ob}{{\rm Ob}}
\newcommand{\obg}{{\rm Ob(\mathbb G)}}
\newcommand{\obgp}{{\rm Ob(\mathbb G')}}
\newcommand{\obh}{{\rm Ob(\mathbb H)}}
\newcommand{\Osmooth}{{\Omega^{\infty}(X,*)}}
\newcommand{\ghomotop}{{\rho_2^{\square}}}
\newcommand{\gcalp}{{\mathbb G(\mathcal P)}}

\newcommand{\rf}{{R_{\mathcal F}}}
\newcommand{\glob}{{\rm glob}}
\newcommand{\loc}{{\rm loc}}
\newcommand{\TOP}{{\rm TOP}}

\newcommand{\wti}{\widetilde}
\newcommand{\what}{\widehat}

\renewcommand{\a}{\alpha}
\newcommand{\be}{\beta}
\newcommand{\ga}{\gamma}
\newcommand{\Ga}{\Gamma}
\newcommand{\de}{\delta}
\newcommand{\del}{\partial}
\newcommand{\ka}{\kappa}
\newcommand{\si}{\sigma}
\newcommand{\ta}{\tau}
\newcommand{\med}{\medbreak}
\newcommand{\medn}{\medbreak \noindent}
\newcommand{\bign}{\bigbreak \noindent}
\newcommand{\lra}{{\longrightarrow}}
\newcommand{\ra}{{\rightarrow}}
\newcommand{\rat}{{\rightarrowtail}}
\newcommand{\oset}[1]{\overset {#1}{\ra}}
\newcommand{\osetl}[1]{\overset {#1}{\lra}}
\newcommand{\hr}{{\hookrightarrow}}
\begin{document}
\subsection{Tangential Cauchy-Riemann complexes}

\subsubsection{Introduction:} \emph{Cauchy-Riemann ($CR$) manifolds and generic submanifolds}

 Let $X$ be a complex manifold of complex dimension $n$. If $M$ is a $\mathcal{C}^{\infty}$-smooth
real submanifold of real codimension $k$ in $X$, let us denote by $T_{\tau}^{\mathbb{C}} (M)$ the
\emph{tangential complex space at $\tau \in M$}. Such a manifold $M$ can be locally represented in the form:
$ M = { z \in \Omega | \rho_1(z)=...= \rho_k(z)=0}$, where all $\rho_i , 1 \leq i \leq k$ are real
$\mathcal{C}^{\infty}$--functions in an open subset $\Omega$ of X. The submanifold $M$ is called \emph{$CR$} if the
number $dim_{\mathbb{C}} T_{\tau}^{\mathbb{C}} (M)$ is independent of the point $\tau \in M$. A submanifold $M_g$
is called \emph{CR generic} if $dim_{\mathbb{C}} T_{\tau}^{\mathbb{C}} (M_g)= (n-k)$ for every $\tau \in M$.

\begin{definition}
\emph{Tangential Cauchy-Riemann complexes}

Let us consider $M_g$ to be an oriented $\mathcal{C}^{\infty}$-smooth $CR$ \emph{generic submanifold} of real
codimension $k$ in an $n$-dimensional complex manifold $X$, and let us denote by $\mathsf{S_M}$
the ideal sheaf in the \PMlinkname{Grassmann algebra}{GrassmanHopfAlgebrasAndTheirDualCoAlgebras} ${\E}$ of germs of complex valued $\mathcal{C}^{\infty}$--forms on
$X$, that are \emph{locally generated by functions} (which vanish on $M_g$), and by their
anti-holomorphic differentials. One also has on $X$ the \emph{Dolbeault complexes} for the
sheaves of germs of smooth forms: 

\[\xymatrix{
{\E}^{p,*} : 0 \to {\E}^{p,0}\ar[r]^{~~~~~~~\overline{\partial}} & {\E}^{p,1} \ar[r]^{\overline {\partial}} & \cdots \ar[r]^{\overline {\partial}} & {\E}^{p,n}\ar[r] & 0
}\]

, where ${\E}^{p,j}$ is the sheaf of germs of complex valued $\mathcal{C}^{\infty}$--forms of bidegree $(p,j)$, for $p,j \leq n$. Let us also set $\mathsf{S_M}^{p,j} = \mathsf{S_M} \bigcup {\E}^{p,j} $.
As $\overline{\partial}\mathsf{S_M}^{p,j} \subset \mathsf{S_M}^{p,j+1}$, for each $0 \leq p \leq n$
we now have the categorical sequence of subcomplexes of the complex ${\E}^{p,*}$ written as :

\[\xymatrix{
{\mathsf{S_M}^{p,*}}: 0 \to {\mathsf{S_M}^{p,0}} \ar[r]^{~~~~~~~\overline{\partial}} & {\mathsf{S_M}^{p,1}} \ar[r]^{\overline{\partial}} & \cdots \ar[r]^{\overline{\partial}} & {\mathsf{S_M}^{p,n}}\ar[r] & 0.}
\]

Therefore, we also have the \emph{quotient complexes} ${\E}^{p,*}$ defined by the exact sequences of
fine sheaves complexes:

\[\xymatrix{
{0} \to {\mathsf{S_M}^{p,*}} \ar[r]& {\E}^{p,*} \ar[r]& \cdots \ar[r] & [{\E}^{p,*}]\ar[r] & 0.
}\]

With the \emph{induced differentials} denoted by $\overline{\partial_M}$ we can now write
the quotient complex--which is called the \emph{tangential Cauchy-Riemann complex of $\mathcal{C}^{\infty}$--smooth forms}-- as follows:

\[\xymatrix{
[{\E}^{p,*}]: 0 \to [{E}^{p,0}]~\ar[r]^{~~~~~~~\overline{\partial_M}} & [{\E}^{p,1}] \ar[r]^{\overline{\partial_M}} & \cdots \ar[r]^{\overline{\partial_M}} & [{\E}^{p,n}]\ar[r] & 0.
}\]
\end{definition}

\textbf{Remarks:}
There are two distinct ways of defining the tangential Cauchy-Riemann complex:
\begin{itemize}
\item an extrinsic approach that uses the $\overline{\partial_M}$ of the ambient $C^n$;

\item an intrinsic approach that does not utilize the ambient $C^n$, and thus generalizes to abstract
$CR$ manifolds (\emph{viz.} A. Bogess, 2000).
\end{itemize}

For further, full details the reader is referred to the recent textbook by Burgess (2000) on this subject.

The \emph{cohomology groups} of $[{\E}^{p,*}]$ on $M \bigcap U$, for $U$ being an open subset
of $X$, are then appropriately denoted here as $H_{\infty}^{p,j}(M\bigcap U)$.


\begin{thebibliography}{9}

\bibitem{CLJL2k}
Christine Laurent-Thi\'ebaut and J\''urgen Leiterer: Dolbeault Isomorphism for CR Manifolds ({\em preprint}).
Pr\'epublication de l'Institut Fourier no. 521 (2000).

\bibitem{NMVG87}
M. Nacinovich and G. Valli, Tangential Cauchy-Riemann complexes on distributions, \emph{Ann. Math. Pure Appl.},
146 (1987): 123--169.

\bibitem{AB2k}
A. Boggess, 2000. \emph{$CR$ Manifolds and the Tangential Cauchy-Riemann Complex}, Boca Raton: CRC Press
(\PMlinkexternal{Book Abstract and Contents on line}{http://books.google.com/books?printsec=frontcover&id=UPQKdeVkLCwC};
\PMlinkexternal{see also the PM book reference}{http://planetmath.org/?op=getobj&from=books&id=154}).

\bibitem{SDGT2k6}
Sorin Dragomir and Giuseppe Tomassini, 2006. Differential geometry and analysis on CR manifolds,
{\em Progress in Mathematics}, vol. 246, Birkh\''auser, Basel.
(\PMlinkexternal{avail. review in PDF}{http://www.ams.org/bull/2008-45-01/S0273-0979-07-01160-3/S0273-0979-07-01160-3.pdf})

\end{thebibliography}
%%%%%
%%%%%
\end{document}
