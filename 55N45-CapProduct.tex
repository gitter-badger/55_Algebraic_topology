\documentclass[12pt]{article}
\usepackage{pmmeta}
\pmcanonicalname{CapProduct}
\pmcreated{2013-03-22 16:26:10}
\pmmodified{2013-03-22 16:26:10}
\pmowner{Mazzu}{14365}
\pmmodifier{Mazzu}{14365}
\pmtitle{cap product}
\pmrecord{9}{38589}
\pmprivacy{1}
\pmauthor{Mazzu}{14365}
\pmtype{Definition}
\pmcomment{trigger rebuild}
\pmclassification{msc}{55N45}
\pmdefines{cap product}

% this is the default PlanetMath preamble.  as your knowledge
% of TeX increases, you will probably want to edit this, but
% it should be fine as is for beginners.

% almost certainly you want these
\usepackage{amssymb}
\usepackage{amsmath}
\usepackage{amsfonts}

% used for TeXing text within eps files
%\usepackage{psfrag}
% need this for including graphics (\includegraphics)
%\usepackage{graphicx}
% for neatly defining theorems and propositions
%\usepackage{amsthm}
% making logically defined graphics
%%%\usepackage{xypic}

% there are many more packages, add them here as you need them

% define commands here

\begin{document}
\newcommand{\CAP}{\frown}
\newcommand{\CUP}{\smile}

Let $X$ be a topological space, $(C_*(X),\partial)$ the singular chain complex, and $(C^*(X;\mathbb K),\delta)$ the singular cochain complex in any coefficient group $\mathbb K$. We can define a bilinear pairing operation $$\CAP: C^i(X;\mathbb K)\times C_n(X)\rightarrow C_{n-i}(X),\ \ \ (n\geq i)$$
in the following way: for each cochain $b\in C^i(X;\mathbb K)$ and each chain $\sigma\in C_n(X)$ we define their  \emph{cap product} $b\CAP\sigma$ as the unique  $(n-i)$-singular chain such that $$a(b\CAP\sigma)=(a\CUP b)(\sigma),$$ where $\CUP: C^j(X;\mathbb K)\times C^h(X;\mathbb K)\rightarrow C^{j+h}(X;\mathbb K)$ denotes the cup product. 
Combining the definition of cap product with the standard properties of cup product we obtain that $$\partial (b\CAP\xi)=(\partial b)\CAP\xi + (-1)^{\mathrm{dim}(b)}b\CAP\partial(\xi),$$ thus there is a corresponding operation in cohomology $$\CAP: H^i(X;\mathbb K)\otimes H_n(X)\rightarrow H_{n-i}(X),\ \ \ (n\geq i)$$ that we also call \emph{cap product}.
%%%%%
%%%%%
\end{document}
