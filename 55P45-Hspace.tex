\documentclass[12pt]{article}
\usepackage{pmmeta}
\pmcanonicalname{Hspace}
\pmcreated{2013-03-22 16:18:18}
\pmmodified{2013-03-22 16:18:18}
\pmowner{yark}{2760}
\pmmodifier{yark}{2760}
\pmtitle{H-space}
\pmrecord{4}{38427}
\pmprivacy{1}
\pmauthor{yark}{2760}
\pmtype{Definition}
\pmcomment{trigger rebuild}
\pmclassification{msc}{55P45}
\pmsynonym{Hopf-space}{Hspace}
\pmsynonym{H space}{Hspace}
\pmsynonym{Hopf space}{Hspace}

\endmetadata


\begin{document}
\PMlinkescapeword{basepoint}
\PMlinkescapeword{fixed}
\PMlinkescapeword{homotopies}
\PMlinkescapeword{identity}
\PMlinkescapeword{implies}
\PMlinkescapeword{relative}

A topological space $X$ is said to be an \emph{H-space} (or \emph{Hopf-space})
if there exists
a continuous binary operation $\varphi\colon X\times X\to X$ 
and a point $p\in X$ such that the functions 
$X\to X$ defined by $x\mapsto\varphi(p,x)$ and $x\mapsto\varphi(x,p)$
are both homotopic to the identity map via homotopies that leave $p$ fixed.
The element $p$ is sometimes referred to as an `identity',
although it need not be an identity element in the usual sense.
Note that the definition implies that $\varphi(p,p)=p$.

Topological groups are examples of H-spaces.

If $X$ is an H-space with `identity' $p$,
then the fundamental group $\pi_1(X,p)$ is abelian.
(However, it is possible for the fundamental group to be non-abelian
for other choices of basepoint, if $X$ is not path-connected.)


%%%%%
%%%%%
\end{document}
