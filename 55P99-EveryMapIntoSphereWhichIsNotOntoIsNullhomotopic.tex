\documentclass[12pt]{article}
\usepackage{pmmeta}
\pmcanonicalname{EveryMapIntoSphereWhichIsNotOntoIsNullhomotopic}
\pmcreated{2013-03-22 18:31:41}
\pmmodified{2013-03-22 18:31:41}
\pmowner{joking}{16130}
\pmmodifier{joking}{16130}
\pmtitle{every map into sphere which is not onto is nullhomotopic}
\pmrecord{8}{41233}
\pmprivacy{1}
\pmauthor{joking}{16130}
\pmtype{Theorem}
\pmcomment{trigger rebuild}
\pmclassification{msc}{55P99}

\endmetadata

% this is the default PlanetMath preamble.  as your knowledge
% of TeX increases, you will probably want to edit this, but
% it should be fine as is for beginners.

% almost certainly you want these
\usepackage{amssymb}
\usepackage{amsmath}
\usepackage{amsfonts}

% used for TeXing text within eps files
%\usepackage{psfrag}
% need this for including graphics (\includegraphics)
%\usepackage{graphicx}
% for neatly defining theorems and propositions
%\usepackage{amsthm}
% making logically defined graphics
%%%\usepackage{xypic}

% there are many more packages, add them here as you need them

% define commands here

\begin{document}
\textbf{Proposition}. Let $X$ be a topological space and $f:X\to\mathbb{S}^{n}$ a continous map from $X$ to $n$-dimensional sphere which is not onto. Then $f$ is nullhomotopic.

\textit{Proof}. Assume that there is $y_0\in\mathbb{S}^{n}$ such that $y_0\not\in\mathrm{im}(f)$. It is well known that there is a homeomorphism $\phi:\mathbb{S}^{n}\setminus\{y_{0}\}\to\mathbb{R}^{n}$. Then we have an induced map
$$\phi\circ f:X\to\mathbb{R}^{n}.$$
Since $\mathbb{R}^{n}$ is contractible, then there is $c\in\mathbb{R}^{n}$ such that $\phi\circ f$ is homotopic to the constant map in $c$ (denoted with the same symbol $c$). Let $\psi:\mathbb{R}^{n}\to\mathbb{S}^{n}$ be a map such that $\psi(x)=\phi^{-1}(x)$ (note that $\psi$ is not the inverse of $\phi$ because $\psi$ is not onto) and take any homotopy $H:\mathrm{I}\times X\to\mathbb{R}^{n}$ from $\phi\circ f$ to $c$. Then we have a homotopy $F:\mathrm{I}\times X\to\mathbb{S}^{n}$ defined by the formula $F=\psi\circ H$. It is clear that
$$F(0,x)=\psi (H(0,x))=\psi(\phi(f(x))) = f(x);$$
$$F(1,x)=\psi (H(1,x))=\psi(c)\in\mathbb{S}^{n}.$$
Thus $F$ is a homotopy from $f$ to a constant map. $\square$

\textbf{Corollary}. If $A\subseteq\mathbb{S}^{n}$ is a deformation retract of $\mathbb{S}^{n}$, then $A=\mathbb{S}^{n}$.

\textit{Proof}. If $A\subseteq X$ then by deformation retraction (associated to $A$) we understand a map $R:\mathrm{I}\times X\to X$ such that $R(0,x)=x$ for all $x\in X$, $R(1,a)=a$ for all $a\in A$ and $R(1,x)\in A$ for all $x\in X$. Thus a deformation retract is a subset $A\subseteq X$ such that there is a deformation retraction $R:\mathrm{I}\times X\to X$ associated to $A$.

Assume that $A$ is a deformation retract of $\mathbb{S}^{n}$ and $A\neq\mathbb{S}^{n}$. Let $R:\mathrm{I}\times\mathbb{S}^{n}\to \mathbb{S}^{n}$ be a deformation retraction. Then $r:\mathbb{S}^{n}\to\mathbb{S}^{n}$ such that $r(x)=R(1,x)$ is homotopic to the identity map (by definition of a deformation retract), but on the other hand it is homotopic to a constant map (it follows from the proposition, since $r$ is not onto, because $A$ is a proper subset of $\mathbb{S}^{n}$). Thus the identity map is homotopic to a constant map, so $\mathbb{S}^{n}$ is contractible. Contradiction. $\square$
%%%%%
%%%%%
\end{document}
