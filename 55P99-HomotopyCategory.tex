\documentclass[12pt]{article}
\usepackage{pmmeta}
\pmcanonicalname{HomotopyCategory}
\pmcreated{2013-03-22 18:17:07}
\pmmodified{2013-03-22 18:17:07}
\pmowner{bci1}{20947}
\pmmodifier{bci1}{20947}
\pmtitle{homotopy category}
\pmrecord{44}{40895}
\pmprivacy{1}
\pmauthor{bci1}{20947}
\pmtype{Topic}
\pmcomment{trigger rebuild}
\pmclassification{msc}{55P99}
\pmclassification{msc}{55R10}
\pmclassification{msc}{55R05}
\pmclassification{msc}{55R65}
\pmclassification{msc}{55R37}
\pmsynonym{category of homotopy equivalence classes}{HomotopyCategory}
%\pmkeywords{homotopy equivalence classes of a topological space}
%\pmkeywords{homotopy category}
\pmrelated{FundamentalGroupoidFunctor}
\pmrelated{TopologicalSpace}
\pmrelated{ApproximationTheoremForAnArbitrarySpace}
\pmrelated{FundamentalGroupoid}
\pmrelated{RiemannianMetric}
\pmrelated{CohomologyGroupTheorem}
\pmrelated{OmegaSpectrum}
\pmrelated{CategoryOfGroupoids2}
\pmdefines{fundamental groupoid}
\pmdefines{fundamental group functor}
\pmdefines{homotopy category}
\pmdefines{fundamental groupoid of a topological space}

% this is the default PlanetMath preamble.  as your 

% almost certainly you want these
\usepackage{amssymb}
\usepackage{amsmath}
\usepackage{amsfonts}

% define commands here
\usepackage{amsmath, amssymb, amsfonts, amsthm, amscd, latexsym}
%%\usepackage{xypic}
\usepackage[mathscr]{eucal}
\theoremstyle{plain}
\newtheorem{lemma}{Lemma}[section]
\newtheorem{proposition}{Proposition}[section]
\newtheorem{theorem}{Theorem}[section]
\newtheorem{corollary}{Corollary}[section]
\theoremstyle{definition}
\newtheorem{definition}{Definition}[section]
\newtheorem{example}{Example}[section]
%\theoremstyle{remark}
\newtheorem{remark}{Remark}[section]
\newtheorem*{notation}{Notation}
\newtheorem*{claim}{Claim}

\renewcommand{\thefootnote}{\ensuremath{\fnsymbol{footnote%%@
}}}
\numberwithin{equation}{section}

\newcommand{\Ad}{{\rm Ad}}
\newcommand{\Aut}{{\rm Aut}}
\newcommand{\Cl}{{\rm Cl}}
\newcommand{\Co}{{\rm Co}}
\newcommand{\DES}{{\rm DES}}
\newcommand{\Diff}{{\rm Diff}}
\newcommand{\Dom}{{\rm Dom}}
\newcommand{\Hol}{{\rm Hol}}
\newcommand{\Mon}{{\rm Mon}}
\newcommand{\Hom}{{\rm Hom}}
\newcommand{\Ker}{{\rm Ker}}
\newcommand{\Ind}{{\rm Ind}}
\newcommand{\IM}{{\rm Im}}
\newcommand{\Is}{{\rm Is}}
\newcommand{\ID}{{\rm id}}
\newcommand{\GL}{{\rm GL}}
\newcommand{\Iso}{{\rm Iso}}
\newcommand{\Sem}{{\rm Sem}}
\newcommand{\St}{{\rm St}}
\newcommand{\Sym}{{\rm Sym}}
\newcommand{\SU}{{\rm SU}}
\newcommand{\Tor}{{\rm Tor}}
\newcommand{\U}{{\rm U}}

\newcommand{\A}{\mathcal A}
\newcommand{\Ce}{\mathcal C}
\newcommand{\D}{\mathcal D}
\newcommand{\E}{\mathcal E}
\newcommand{\F}{\mathcal F}
\newcommand{\G}{\mathcal G}
\newcommand{\Q}{\mathcal Q}
\newcommand{\R}{\mathcal R}
\newcommand{\cS}{\mathcal S}
\newcommand{\cU}{\mathcal U}
\newcommand{\W}{\mathcal W}

\newcommand{\bA}{\mathbb{A}}
\newcommand{\bB}{\mathbb{B}}
\newcommand{\bC}{\mathbb{C}}
\newcommand{\bD}{\mathbb{D}}
\newcommand{\bE}{\mathbb{E}}
\newcommand{\bF}{\mathbb{F}}
\newcommand{\bG}{\mathbb{G}}
\newcommand{\bK}{\mathbb{K}}
\newcommand{\bM}{\mathbb{M}}
\newcommand{\bN}{\mathbb{N}}
\newcommand{\bO}{\mathbb{O}}
\newcommand{\bP}{\mathbb{P}}
\newcommand{\bR}{\mathbb{R}}
\newcommand{\bV}{\mathbb{V}}
\newcommand{\bZ}{\mathbb{Z}}

\newcommand{\bfE}{\mathbf{E}}
\newcommand{\bfX}{\mathbf{X}}
\newcommand{\bfY}{\mathbf{Y}}
\newcommand{\bfZ}{\mathbf{Z}}

\renewcommand{\O}{\Omega}
\renewcommand{\o}{\omega}
\newcommand{\vp}{\varphi}
\newcommand{\vep}{\varepsilon}
\newcommand{\diag}{{\rm diag}}
\newcommand{\grp}{{\mathbb G}}
\newcommand{\dgrp}{{\mathbb D}}
\newcommand{\desp}{{\mathbb D^{\rm{es}}}}
\newcommand{\Geod}{{\rm Geod}}
\newcommand{\geod}{{\rm geod}}
\newcommand{\hgr}{{\mathbb H}}
\newcommand{\mgr}{{\mathbb M}}
\newcommand{\ob}{{\rm Ob}}
\newcommand{\obg}{{\rm Ob(\mathbb G)}}
\newcommand{\obgp}{{\rm Ob(\mathbb G')}}
\newcommand{\obh}{{\rm Ob(\mathbb H)}}
\newcommand{\Osmooth}{{\Omega^{\infty}(X,*)}}
\newcommand{\ghomotop}{{\rho_2^{\square}}}
\newcommand{\gcalp}{{\mathbb G(\mathcal P)}}
\newcommand{\rf}{{R_{\mathcal F}}}
\newcommand{\glob}{{\rm glob}}
\newcommand{\loc}{{\rm loc}}
\newcommand{\TOP}{{\rm TOP}}
\newcommand{\wti}{\widetilde}
\newcommand{\what}{\widehat}
\renewcommand{\a}{\alpha}
\newcommand{\be}{\beta}
\newcommand{\ga}{\gamma}
\newcommand{\Ga}{\Gamma}
\newcommand{\de}{\delta}
\newcommand{\del}{\partial}
\newcommand{\ka}{\kappa}
\newcommand{\si}{\sigma}
\newcommand{\ta}{\tau}
\newcommand{\lra}{{\longrightarrow}}
\newcommand{\ra}{{\rightarrow}}
\newcommand{\rat}{{\rightarrowtail}}
\newcommand{\oset}[1]{\overset {#1}{\ra}}
\newcommand{\osetl}[1]{\overset {#1}{\lra}}
\newcommand{\hr}{{\hookrightarrow}}
\begin{document}
\subsection{Homotopy category, fundamental groups and fundamental groupoids}

  Let us consider first the category \textbf{$Top$} whose objects are topological spaces $X$ with a chosen 
basepoint $x \in X$ and whose morphisms are continuous maps $X \to Y$ that associate the basepoint of $Y$ to the
basepoint of $X$. The fundamental group of $X$ specifies a functor $Top \to \textbf{G}$,
with $\textbf{G}$ being the category of groups and group homomorphisms, which is called 
\emph{the fundamental group functor}. 

\subsection{Homotopy category}
  Next, when one has a suitably defined relation of homotopy between morphisms, or maps, in a category \textbf{$U$}, one can define the \emph{homotopy category} $hU$ as the category whose objects are the same as the objects of \textbf{$U$}, but with morphisms being defined by the homotopy classes of maps; this is in fact the homotopy category of \emph{unbased spaces}. 

\subsection{Fundamental groups}
 We can further require that homotopies on \textbf{$Top$} map each basepoint to a corresponding basepoint, thus leading to the definition of the \emph{homotopy category $hTop$ of based spaces}. Therefore, the fundamental group is a \emph{homotopy invariant} functor on \textbf{$Top$}, with the meaning that the latter functor factors through a functor $ hTop \to \textbf{G} $. A homotopy equivalence in \textbf{$U$} is an isomorphism in $hTop$. Thus, based homotopy equivalence induces an isomorphism of fundamental groups. 

\subsection{Fundamental groupoid}
 In the general case when one does not choose a basepoint, a \emph{fundamental groupoid}
$\Pi_1 (X)$ of a topological space $X$ needs to be defined as the category whose objects are
the base points of $X$ and whose morphisms $x \to y$ are the equivalence classes of paths from $x$ to $y$.
\begin{itemize}
\item Explicitly, the objects of $\Pi_1(X)$ are the points of $X$
$$\mathrm{Obj}(\Pi_1(X))=X\,,$$
\item morphisms are homotopy classes of paths ``rel endpoints'' that is
$$\mathrm{Hom}_{\Pi_1(x)}(x,y)=\mathrm{Paths}(x,y)/\sim\, ,$$
where, $\sim$ denotes homotopy rel endpoints, and,
\item composition of morphisms is defined \emph{via} piecing together, or concatenation, of paths.
\end{itemize}

\subsection{Fundamental groupoid functor}

  Therefore, the set of endomorphisms of an object $x$ is precisely the fundamental group
$\pi(X,x)$. One can thus construct the \emph{groupoid of homotopy equivalence classes}; this construction can be then carried out by utilizing functors from the category \textbf{$Top$}, or its subcategory $hU$,
to the \emph{category of groupoids and groupoid homomorphisms}, $Grpd$. One such functor 
which associates to each topological space its fundamental (homotopy) groupoid is appropriately called the
\emph{fundamental groupoid functor}.

\subsection{An example: the category of simplicial, or CW-complexes}
 
  As an important example, one may wish to consider the category of simplicial, or $CW$-complexes and homotopy defined
for $CW$-complexes. Perhaps, the simplest example is that of a one-dimensional $CW$-complex, which is a graph.
As described above, one can define a functor from the category of graphs, \textbf{Grph}, to \textbf{$Grpd$}
and then define the fundamental homotopy groupoids of graphs, hypergraphs, or pseudographs. The case of freely generated
graphs (one-dimensional $CW$-complexes) is particularly simple and can be computed with a digital computer by a finite
algorithm using the finite groupoids associated with such finitely generated $CW$-complexes.

\subsubsection{Remark}
 Related to this concept of homotopy category for unbased topological spaces, one can then prove the 
\emph{approximation theorem for an arbitrary space} by considering a functor 
$$\Gamma : \textbf{hU} \longrightarrow \textbf{hU},$$ and also the construction of an approximation of an arbitrary space $X$ as the 
colimit $\Gamma X$ of a sequence of cellular inclusions of $CW$-complexes $X_1, ..., X_n$ , so 
that one obtains $X \equiv colim  [X_i]$. 

  Furthermore, the homotopy groups of the $CW$-complex $\Gamma X$ are the colimits of the 
homotopy groups of $X_n$,  and $\gamma_{n+1} : \pi_q(X_{n+1})\longmapsto\pi_q (X)$ is a group epimorphism.

\begin{thebibliography} {9}

\bibitem{MJP1999}
May, J.P. 1999, \emph{A Concise Course in Algebraic Topology.}, The University of Chicago Press: Chicago


\bibitem{BR-JG2k4}
R. Brown and G. Janelidze.(2004). Galois theory and a new homotopy double groupoid of a map of spaces.(2004). 
{\em Applied Categorical Structures},\textbf{12}: 63-80. Pdf file in arxiv: math.AT/0208211 


\end{thebibliography}
%%%%%
%%%%%
\end{document}
