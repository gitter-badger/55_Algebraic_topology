\documentclass[12pt]{article}
\usepackage{pmmeta}
\pmcanonicalname{BibliographyForGroupoidsAndAlgebraicTopology}
\pmcreated{2013-03-22 18:15:50}
\pmmodified{2013-03-22 18:15:50}
\pmowner{bci1}{20947}
\pmmodifier{bci1}{20947}
\pmtitle{bibliography for groupoids and algebraic topology}
\pmrecord{41}{40864}
\pmprivacy{1}
\pmauthor{bci1}{20947}
\pmtype{Bibliography}
\pmcomment{trigger rebuild}
\pmclassification{msc}{55Q05}
\pmclassification{msc}{55U40}
\pmclassification{msc}{20L05}
\pmclassification{msc}{58H05}
\pmclassification{msc}{54-00}
\pmclassification{msc}{58H10}
\pmclassification{msc}{22A22}
\pmclassification{msc}{18B40}
\pmclassification{msc}{20-00}
\pmclassification{msc}{55-00}
\pmsynonym{algebraic topology bibliography}{BibliographyForGroupoidsAndAlgebraicTopology}
%\pmkeywords{algebraic topology bibliography}
%\pmkeywords{groupoids and topology}
%\pmkeywords{groupoids and algebraic topology}
\pmrelated{BibliographyOfCategoryTheory}

% this is the default PlanetMath preamble.  as your 
\usepackage{amssymb}
\usepackage{amsmath}
\usepackage{amsfonts}

% used for TeXing text within eps files
%\usepackage{psfrag}
% need this for including graphics (\includegraphics)
%\usepackage{graphicx}
% for neatly defining theorems and propositions
%\usepackage{amsthm}
% making logically defined graphics
%%%\usepackage{xypic}

% define commands here
\usepackage{amsmath, amssymb, amsfonts, amsthm, amscd, latexsym}
%%\usepackage{xypic}
\usepackage[mathscr]{eucal}

\setlength{\textwidth}{6.5in}
%\setlength{\textwidth}{16cm}
\setlength{\textheight}{9.0in}
%\setlength{\textheight}{24cm}

\hoffset=-.75in     %%ps format
%\hoffset=-1.0in     %%hp format
\voffset=-.4in

\theoremstyle{plain}
\newtheorem{lemma}{Lemma}[section]
\newtheorem{proposition}{Proposition}[section]
\newtheorem{theorem}{Theorem}[section]
\newtheorem{corollary}{Corollary}[section]

\theoremstyle{definition}
\newtheorem{definition}{Definition}[section]
\newtheorem{example}{Example}[section]
%\theoremstyle{remark}
\newtheorem{remark}{Remark}[section]
\newtheorem*{notation}{Notation}
\newtheorem*{claim}{Claim}

\renewcommand{\thefootnote}{\ensuremath{\fnsymbol{footnote%%@
}}}
\numberwithin{equation}{section}

\newcommand{\Ad}{{\rm Ad}}
\newcommand{\Aut}{{\rm Aut}}
\newcommand{\Cl}{{\rm Cl}}
\newcommand{\Co}{{\rm Co}}
\newcommand{\DES}{{\rm DES}}
\newcommand{\Diff}{{\rm Diff}}
\newcommand{\Dom}{{\rm Dom}}
\newcommand{\Hol}{{\rm Hol}}
\newcommand{\Mon}{{\rm Mon}}
\newcommand{\Hom}{{\rm Hom}}
\newcommand{\Ker}{{\rm Ker}}
\newcommand{\Ind}{{\rm Ind}}
\newcommand{\IM}{{\rm Im}}
\newcommand{\Is}{{\rm Is}}
\newcommand{\ID}{{\rm id}}
\newcommand{\GL}{{\rm GL}}
\newcommand{\Iso}{{\rm Iso}}
\newcommand{\Sem}{{\rm Sem}}
\newcommand{\St}{{\rm St}}
\newcommand{\Sym}{{\rm Sym}}
\newcommand{\SU}{{\rm SU}}
\newcommand{\Tor}{{\rm Tor}}
\newcommand{\U}{{\rm U}}

\newcommand{\A}{\mathcal A}
\newcommand{\Ce}{\mathcal C}
\newcommand{\D}{\mathcal D}
\newcommand{\E}{\mathcal E}
\newcommand{\F}{\mathcal F}
\newcommand{\G}{\mathcal G}
\newcommand{\Q}{\mathcal Q}
\newcommand{\R}{\mathcal R}
\newcommand{\cS}{\mathcal S}
\newcommand{\cU}{\mathcal U}
\newcommand{\W}{\mathcal W}

\newcommand{\bA}{\mathbb{A}}
\newcommand{\bB}{\mathbb{B}}
\newcommand{\bC}{\mathbb{C}}
\newcommand{\bD}{\mathbb{D}}
\newcommand{\bE}{\mathbb{E}}
\newcommand{\bF}{\mathbb{F}}
\newcommand{\bG}{\mathbb{G}}
\newcommand{\bK}{\mathbb{K}}
\newcommand{\bM}{\mathbb{M}}
\newcommand{\bN}{\mathbb{N}}
\newcommand{\bO}{\mathbb{O}}
\newcommand{\bP}{\mathbb{P}}
\newcommand{\bR}{\mathbb{R}}
\newcommand{\bV}{\mathbb{V}}
\newcommand{\bZ}{\mathbb{Z}}

\newcommand{\bfE}{\mathbf{E}}
\newcommand{\bfX}{\mathbf{X}}
\newcommand{\bfY}{\mathbf{Y}}
\newcommand{\bfZ}{\mathbf{Z}}

\renewcommand{\O}{\Omega}
\renewcommand{\o}{\omega}
\newcommand{\vp}{\varphi}
\newcommand{\vep}{\varepsilon}

\newcommand{\diag}{{\rm diag}}
\newcommand{\grp}{{\mathbb G}}
\newcommand{\dgrp}{{\mathbb D}}
\newcommand{\desp}{{\mathbb D^{\rm{es}}}}
\newcommand{\Geod}{{\rm Geod}}
\newcommand{\geod}{{\rm geod}}
\newcommand{\hgr}{{\mathbb H}}
\newcommand{\mgr}{{\mathbb M}}
\newcommand{\ob}{{\rm Ob}}
\newcommand{\obg}{{\rm Ob(\mathbb G)}}
\newcommand{\obgp}{{\rm Ob(\mathbb G')}}
\newcommand{\obh}{{\rm Ob(\mathbb H)}}
\newcommand{\Osmooth}{{\Omega^{\infty}(X,*)}}
\newcommand{\ghomotop}{{\rho_2^{\square}}}
\newcommand{\gcalp}{{\mathbb G(\mathcal P)}}

\newcommand{\rf}{{R_{\mathcal F}}}
\newcommand{\glob}{{\rm glob}}
\newcommand{\loc}{{\rm loc}}
\newcommand{\TOP}{{\rm TOP}}

\newcommand{\wti}{\widetilde}
\newcommand{\what}{\widehat}

\renewcommand{\a}{\alpha}
\newcommand{\be}{\beta}
\newcommand{\ga}{\gamma}
\newcommand{\Ga}{\Gamma}
\newcommand{\de}{\delta}
\newcommand{\del}{\partial}
\newcommand{\ka}{\kappa}
\newcommand{\si}{\sigma}
\newcommand{\ta}{\tau}
\newcommand{\med}{\medbreak}
\newcommand{\medn}{\medbreak \noindent}
\newcommand{\bign}{\bigbreak \noindent}
\newcommand{\lra}{{\longrightarrow}}
\newcommand{\ra}{{\rightarrow}}
\newcommand{\rat}{{\rightarrowtail}}
\newcommand{\oset}[1]{\overset {#1}{\ra}}
\newcommand{\osetl}[1]{\overset {#1}{\lra}}
\newcommand{\hr}{{\hookrightarrow}}

\begin{document}
The following are recent sources for several areas in abstract algebra, homological algebra, homotopy groups, homotopy groupoids, algebraic topology and higher dimensional algebra (HDA).

\subsection{Algebraic Topology and Groupoids}
\begin{enumerate}
\item Ronald Brown: Topology and Groupoids, BookSurge LLC (2006).
\item Ronald Brown R, P.J. Higgins, and R. Sivera.: \emph{``Non-Abelian algebraic topology"}.
 http://www. bangor.ac.uk/mas010/nonab-a-t.html; http://www.bangor.ac.uk/mas010/nonab-t/partI010604.pdf , 
Springer: in press (2010).
\item R. Brown and J.-L. Loday: Homotopical excision, and Hurewicz theorems, for n-cubes of spaces, Proc. London Math. Soc., 54:(3), 176-192, (1987).
\item R. Brown and J.-L. Loday: Van Kampen Theorems for diagrams of spaces, Topology, 26: 311-337 (1987).
\item R. Brown and G. H. Mosa: Double algebroids and crossed modules of algebroids, University of Wales-Bangor, Maths Preprint, 1986.
\item R. Brown and C.B. Spencer: Double groupoids and crossed modules, Cahiers Top. G\'eom. Diff. 17 (1976), 343-362.
\item Madalina (Ruxi) Buneci.: \emph{Groupoid Representations}., Ed. Mirton: Timisoara (2003).
\item Allain Connes: \emph{Noncommutative Geometry}, Academic Press 1994.
\end{enumerate}

\subsection{Non-Abelian Algebraic Topology and Higher Dimensional Algebra}
\begin{enumerate}
\item Ronald Brown: Non-Abelian Algebraic Topology, vols. I and II. 2010. (in press: Springer): \PMlinkexternal{Nonabelian Algebraic Topology:filtered spaces, crossed complexes, cubical higher homotopy groupoids}{http://www.bangor.ac.uk/~mas010/rbrsbookb-e040310.pdf}

\item \PMlinkexternal{Higher Dimensional Algebra: An Introduction}{http://en.wikipedia.org/wiki/Higher_dimensional_algebra}

\item  \PMlinkexternal{Higher Dimensional Algebra and Algebraic Topology., 282 pages, Feb. 10, 2010}{http://en.wikipedia.org/wiki/User:Bci2/Books/Higher_Dimensional_Algebra}

\end{enumerate}
\subsubsection{General Topology}
\begin{enumerate}
\item N. Bourbaki, \emph{General Topology, Part 1},
Addison-Wesley Publishing Company, 1966
\item \PMlinkescapetext{John G. Hocking, Gail S. Young, \emph{Topology}, Addison-Wesley, Reading, Massachussets, 1961; Dover, New York, 1988; ISBN 0-486-65674, QA611.H68}
\item R.~Engelking, \emph{General topology}, PWN, Warsaw, 1977.
\item R. Brown, \emph{Elements of Modern Topology}, McGraw Hill, Maidenhead, 1968.
\item R. Brown and T.L. Thickstun, \emph{Low-Dimensional Topology}, Volume 1 of the conference on Topology in Low Dimension, Bangor, 1979), London Math. Soc. Lecture Notes No. 48 (1982). (now available from CUP). 
\item R.Brown, \emph{Topology: a geometric account of general topology, homotopy types, and the fundamental groupoid}, 
Ellis Horwood, Chichester (1988) 460 pp. 
\item J.L. Kelley, \emph{General Topology}, 
D. van Nostrand Company, Inc., 1955; Springer-Verlag, New York, ISBN 0-387-90125-6
\item \PMlinkescapetext{William J. Pervin, \emph{Foundations of General Topology}, Academic Press, 
New York, 1964.}
\item William S. Massey, \emph{A Basic Course in Algebraic Topology}, Springer, New York, 1991; ISBN 0-387-97430-X, 3-540-97430-X; QA612.M374
\item C.R.F. Maunder, \emph{Algebraic Topology}, Cambridge University Press, 1980, Dover Edition, 1996; ISBN 0-486-69131-4.
\item Fred H. Croom, \emph{Basic Concepts of Algebraic Topology}, Springer-Verlag, New York, 1978; ISBN 0-387-90288-0
\item Joseph J. Rotman, \emph{An Introduction to Algebraic Topology}, Springer-Verlag, New York, 1988: ISBN 0-387-96678-1
\end{enumerate}

\subsection{Topology in relation to other areas of mathematics*}

\begin{enumerate}
\item Michael Atiyah, \emph{The Geometry and Physics of Knots}, Cambridge University Press, Cambridge, 1990; ISBN 0-521-39521-6, 0-521-39554-2
\item Alfsen, E.M. and F. W. Schultz: \emph{Geometry of State Spaces of
Operator Algebras}, Birkh\"auser, Boston-Basel-Berlin (2003).
\item J. Butterfield and C. J. Isham : A topos perspective on the Kochen-Specker theorem I-IV, Int. J. Theor. Phys, 37 (1998) No 11., 2669-2733 38 (1999) No 3., 827-859, 39 (2000) No 6., 1413-1436, 41 (2002) No 4., 613-639.
\item M. Chaician and A. Demichev: \emph{Introduction to Quantum Groups}, World Scientific (1996).
\end{enumerate}

\subsection{Topological Groups*}

\begin{enumerate}
\item \PMlinkescapetext{Taqdir Husain, \emph{Introduction to Topological Groups}, W. B. Saunders Company, Philadelphia, 1966.}
\item \PMlinkescapetext{Markus Stroppel, \emph{Locally Compact Groups}, EMS Textbooks in Mathematics, European Mathematical Society,
Z\"urich, 2006, ISBN 3-03719-016-7}
\end{enumerate}




{\bf References on the Foundations of Algebra: Groups, Rings, Fields, Modules,... , MSC 20-00, MSC 20A}

\begin{enumerate}
\item David S. Dummit, Richard M. Foote, \emph{Abstract Algebra}, John Wiley and Sons, Inc. 1999
\item I. N. Herstein, \emph{Topics in Algebra}, Xerox College Publishing, 1975.
\begin{quote} A classic undergraduate textbook introducing groups, rings, fields, and linear algebra.
\end{quote}  
\item Serge Lang, \emph{Algebra}, Addison-Wesley Publishing Company, 1997. 
\begin{quote}A textbook covering a broad range of topics in algebra, such as basic homological algebra and basic category theory which are covered at the beginner's level.
\end{quote}  
\item John S. Rose, \emph{A Course on Group Theory}, Dover Publications, New York, 1994.
\item Thomas W. Hungerford, \emph{Algebra}, Springer-Verlag Graduate Texts in Mathematics {\bf 73}, 1974.  
\end{enumerate}


\subsection{Literature on Groupoids and Topology}

\begin{enumerate}
\item R. Brown, K.Hardie, H.Kamps, T. Porter, The homotopy double groupoid of a Hausdorff space, 
\emph{Theory and Applications of Categories}, {\bf 10} (2002) 71-93.
\item R. Brown, I.Icen and O. Mucuk. Holonomy and monodromy groupoids, in \emph{Lie Algebroids}, Banach Center Publications Institute of Mathematics, Polish Academy of Sciences, Warsaw, {\bf 54} (2001) 9-20. 
\item R. Brown. Algebraic homotopy, Supplement III., in \emph{Encyclopaedia of Mathematics}, Managing Editor: M. Hazewinkel Kluwer Academic Publishers (2002) 29-31.
\item R. Brown. Exponential law in topology, Supplement III., in \emph{Encyclopaedia of Mathematics}, Managing Editor: M. Hazewinkel Kluwer Academic Publishers (2002) 142-143.
\item R. Brown \.I. I\c{c}en, and O. Mucuk), Examples and coherence properties of local subgroupoids, \emph{Topology and its Applications} {\bf 127} (2003) 393-408. 
\item R. Brown and I.ICEN. Towards a 2-dimensional notion of holonomy, \emph{Advances in Math}. {\bf 178} (2003) 141-175.
\item R. Brown C.D.WENSLEY), Computation and homotopical applications of induced crossed modules, \emph{J. Symbolic Computation} {\bf 35} (2003) 59-72. 
\item R. Brown, M. BULLEJOS and T.PORTER), Crossed complexes, free crossed resolutions and graph products of groups, \emph{Proceedings Workshop Korea 2000}, Heldermann Verlag. {\bf 27} (2003) 11-26.
\end{enumerate}

\subsection{General Topology}

Alexandroff, Paul. Elementary Concepts of Topology Mineola, NY: Dover, 1961. 

Arkhangelski i, A.V. and Pontrjagin, Lev S., eds. General Topology I: Basic Concepts and Constructions, Dimension Theory New York, NY: Springer-Verlag, 1990. 

Bing, R.H. Elementary Point Set Topology Washington, DC: Mathematical Association of America, 1960. 

Bourbaki, Nicolas. Elements of Mathematics: General Topology New York, NY: Springer-Verlag, 1989. 

Chinn, William G. and Steenrod, Norman E. First Concepts of Topology Washington, DC: Mathematical Association of America, 1966. 

Dugundji, James. Topology Boston, MA: Allyn and Bacon, 1966. 

Fuks, D.B. and Rokhlin, V.A. Beginner's Course in Topology: Geometric Chapters New York, NY: Springer-Verlag, 1984. 

Gamelin, Theodore W. and Greene, Robert E. Introduction to Topology Philadelphia, PA: Saunders College, 1983. 

Gemignani, Michael C. Elementary Topology, Mineola, NY: Dover, 1990. Second Edition. 

Hausdorff, Felix. Set Theory, New York, NY: Chelsea, 1957, 1978. Third Edition. 

Hurewicz, Witold and Wallman, Henry. Dimension Theory Princeton, NJ: Princeton University Press, 1941. 

Kaplansky, Irving. Set Theory and Metric Spaces, New York, NY: Chelsea, 1977. Second Edition. 

Kelley, John L. General Topology, New York, NY: Springer-Verlag, 1975. 

Kuratowski, K. Topology, New York, NY: Academic Press, 1966, 1969. 2 Vols. 

Munkres, James R. Topology: A First Course Englewood Cliffs, NJ: Prentice Hall, 1975. 

Newman, M.H.A. Elements of the Topology of Plane Sets of Points New York, NY: Cambridge University Press, 1964. 

Oxtoby, John C. Measure and Category: A Survey of the Analogies between Topological and Measure Spaces, New York, NY: Springer-Verlag, 1980. Second Edition. 

Sierpinski, W. General Topology Toronto: University of Toronto Press, 1952. 

Simmons, George F. Introduction to Topology and Modern Analysis Melbourne, FL: Robert E. Krieger, 1983. 

Steen, Lynn Arthur and Seebach, J. Arthur, Jr. Counterexamples in Topology, New York, NY: Springer-Verlag, 1978. Second Edition. 

Sutherland, W.A. Introduction to Metric and Topological Spaces New York, NY: Clarendon Press, 1975. 

Willard, Stephen. General Topology Reading, MA: Addison-Wesley, 1970. 


\subsubsection{Geometric Topology and Theoretical Physics}

Atiyah, Michael F. The Geometry and Physics of Knots New York, NY: Cambridge University Press, 1990. 
Burde, Gerhard and Zieschang, Heiner. Knots Hawthorne, NY: Walter de Gruyter, 1985. 

Crowell, Richard H. and Fox, Ralph H. Introduction to Knot Theory New York, NY: Springer-Verlag, 1977. 

Firby, P.A. and Gardiner, C.F. Surface Topology New York, NY: Ellis Horwood, 1982. 

Flegg, H. Graham. From Geometry to Topology Philadelphia, PA: Crane, Russak, 1974. 

Francis, George K. A Topological Picturebook New York, NY: Springer-Verlag, 1987. 

Freedman, M.H. and Luo, Feng. Selected Applications of Geometry to Low-Dimensional Topology Providence, RI: American Mathematical Society, 1990. 

Griffiths, H.B. Surfaces, New York, NY: Cambridge University Press, 1976, 1981. Second Edition. 

Kauffman, Louis H. On Knots Princeton, NJ: Princeton University Press, 1987. 

Moise, Edwin E. Geometric Topology in Dimensions 2 and 3 New York, NY: Springer-Verlag, 1977. 

Montesinos, Jose Maria. Classical Tessellations and Three-Manifolds New York, NY: Springer-Verlag, 1987. 

Reidemeister, K. Knot Theory Moscow, ID: BCS Associates, 1983. 

Rolfsen, Dale. Knots and Links, Boston, MA: Publish or Perish, 1976, 1991. Second Edition. 

Rourke, C.P. and Sanderson, B.J. Introduction to Piecewise-Linear Topology New York, NY: Springer-Verlag, 1972. 

Wall, C.T.C. A Geometric Introduction to Topology Reading, MA: Addison-Wesley, 1972. 

Weeks, Jeffrey R. The Shape of Space: How to Visualize Surfaces and Three-Dimensional Manifolds New York, NY: Marcel Dekker, 1985. 

\subsubsection{Algebraic Topology}

Aleksandrov, P. Combinatorial Topology, Baltimore, MD: Graylock Press, 1956--60. 3 Vols. 

Armstrong, M.A. Basic Topology New York, NY: Springer-Verlag, 1983, 1990. 

Artin, Emil. Introduction to Algebraic Topology Columbus, OH: Charles E. Merrill, 1969. 

Blackett, Donald W. Elementary Topology: A Combinatorial and Algebraic Approach, New York, NY: Academic Press, 1982. 

Giblin, P.J. Graphs, Surfaces and Homology: An Introduction to Algebraic Topology, New York, NY: Chapman and Hall, 1981. Second Edition. 

Gramain, Andre. Topology of Surfaces Moscow, ID: BCS Associates, 1984. 

Greenberg, Marvin Jay and Harper, John R. Lectures on Algebraic Topology, Reading, MA: W.A. Benjamin, 1967, 1981. Second Edition. 

Henle, Michael. A Combinatorial Introduction to Topology New York, NY: W.H. Freeman, 1979. 

Janich, Klaus. Topology New York, NY: Springer-Verlag, 1984. 

Kosniowski, Czes. A First Course in Algebraic Topology New York, NY: Cambridge University Press, 1980. 

Lefschetz, Solomon. Topology, New York, NY: Chelsea, 1956. Second Edition. 

Massey, William S. Algebraic Topology: An Introduction New York, NY: Springer-Verlag, 1977. 

Massey, William S. A Basic Course in Algebraic Topology New York, NY: Springer-Verlag, 1991. 

Milnor, John W. Singular Points of Complex Hypersurfaces Princeton, NJ: Princeton University Press, 1968. 

Milnor, John W. and Stasheff, James D. Characteristic Classes Princeton, NJ: Princeton University Press, 1974. 

Munkres, James R. Elements of Algebraic Topology Reading, MA: Addison-Wesley, 1984. 

Naber, Gregory L. Topological Methods in Euclidean Spaces New York, NY: Cambridge University Press, 1980. 

Pontrjagin, Lev S. Topological Groups, New York, NY: Gordon and Breach, 1966. Second Edition. 

Pontrjagin, Lev S. Foundations of Combinatorial Topology Baltimore, MD: Graylock Press, 1952. 

Seifert, Herbert and Threlfall, W. A Textbook of Topology New York, NY: Academic Press, 1980. 

Spanier, Edwin H. Algebraic Topology New York, NY: McGraw-Hill, 1966. 

Stillwell, John. Classical Topology and Combinatorial Group Theory New York, NY: Springer-Verlag, 1980. 

Vick, James. Homology Theory: An Introduction to Algebraic Topology New York, NY: Academic Press, 1973. 

\subsubsection{Differential Topology}

Almgren, F.J. Plateau's Problem Redwood City, CA: Benjamin Cummings, 1966. 

Auslander, Louis and Mackenzie, Robert E. Introduction to Differentiable Manifolds Mineola, NY: Dover, 1977. 

Brocker, Th. and Janich, Klaus. Introduction to Differential Topology New York, NY: Cambridge University Press, 1982. 

Chillingworth, D.R.J. Differential Topology with a View to Applications Brooklyn, NY: Pitman, 1976. 

Guillemin, Victor and Pollack, Alan. Differential Topology Englewood Cliffs, NJ: Prentice Hall, 1974. 

Hirsch, Morris W. Differential Topology New York, NY: Springer-Verlag, 1976. 

Milnor, John W. Topology from the Differentiable Viewpoint Charlottesville, VA: University Press of Virginia, 1965. 

Milnor, John W. Morse Theory Princeton, NJ: Princeton University Press, 1963. 

Singer, I.M. and Thorpe, John A. Lecture Notes on Elementary Topology and Geometry New York, NY: Springer-Verlag, 1987. 

Warner, Frank W. Foundations of Differentiable Manifolds and Lie Groups New York, NY: Springer-Verlag, 1983. 


%%%%%
%%%%%
\end{document}
