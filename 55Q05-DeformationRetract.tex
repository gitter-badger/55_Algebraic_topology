\documentclass[12pt]{article}
\usepackage{pmmeta}
\pmcanonicalname{DeformationRetract}
\pmcreated{2013-03-22 13:31:44}
\pmmodified{2013-03-22 13:31:44}
\pmowner{mathcam}{2727}
\pmmodifier{mathcam}{2727}
\pmtitle{deformation retract}
\pmrecord{14}{34121}
\pmprivacy{1}
\pmauthor{mathcam}{2727}
\pmtype{Definition}
\pmcomment{trigger rebuild}
\pmclassification{msc}{55Q05}
\pmrelated{Retract}
\pmdefines{strong deformation retract}

% this is the default PlanetMath preamble.  as your knowledge
% of TeX increases, you will probably want to edit this, but
% it should be fine as is for beginners.

% almost certainly you want these
\usepackage{amssymb}
\usepackage{amsmath}
\usepackage{amsfonts}

% used for TeXing text within eps files
%\usepackage{psfrag}
% need this for including graphics (\includegraphics)
%\usepackage{graphicx}
% for neatly defining theorems and propositions
%\usepackage{amsthm}
% making logically defined graphics
%%%\usepackage{xypic}

% there are many more packages, add them here as you need them

% define commands here
\begin{document}
Let $X$ and $Y$ be topological spaces such that $Y\subset X$.
A \emph{deformation retract} of $X$ onto $Y$ is a collection of
mappings $f_t:X\rightarrow X$, $t\in [0,1]$ such that
\begin{enumerate}
\item $f_0 = id_X$, the identity mapping on $X$,
\item $f_1(X) \subseteq Y$,
\item $Y$ is a retract of $X$ via $f_1$ (that is, $f_1$ restricted to $Y$ is the identity on $Y$)
\item the mapping $X\times I\rightarrow X$, $(x,t)\mapsto f_t(x)$ is continuous, where the topology on $X\times I$ is the product topology.
\end{enumerate}

Of course, by condition 3, condition 2 can be improved: $f_1(X)=Y$.

A deformation retract is called a \emph{strong deformation retract} if condition 3 above is replaced by a stronger form: $Y$ is a retract of $X$ via $f_t$ for every $t\in [0,1]$.

\subsubsection*{Properties}
\begin{itemize}
\item Let $X$ and $Y$ be as in the above definition. Then a collection of
mappings $f_t:X\rightarrow X$, $t\in [0,1]$ is 
a deformation retract (of $X$ onto $Y$) if and only if it is a 
\PMlinkname{homotopy}{HomotopyOfMaps} rel Y \PMlinkescapetext{between} $\operatorname{id}_X$ and some retraction $r$ of $X$ onto $Y$. 
\end{itemize}

\subsubsection*{Examples}
\begin{itemize}
\item If $x_0\in \mathbb{R}^n$, then
$f_t(x)=(1-t)x+tx_0$, $x\in \mathbb{R}^n$  shows that $\mathbb{R}^n$ deformation retracts onto $\{x_0\}$.
 Since $\{x_0\} \subset \mathbb{R}^n$,
it follows that deformation retract is not an equivalence relation.
\item The same map as in the previous example can be used to deformation retract any star-shaped set in $\mathbb{R}^n$ onto a point.
\item we obtain a
deformation retraction of
$\mathbb{R}^n\backslash \{0\}$ onto the \PMlinkid{$(n-1)$-sphere }{186} $S^{n-1}$ by setting $$f_t(x)=(1-t)x+t \displaystyle{\frac{x}{||x||}},$$ where $x\in \mathbb{R}^n\backslash \{0\}$, $n>0$,  
\item The \PMlinkid{M\"obius strip}{3278} deformation retracts onto the circle $S^1$.
\item The $2$-torus with one point removed deformation retracts onto
two copies of $S^1$ joined at one point. (The circles can be
chosen to be longitudinal and latitudinal circles of the torus.)
\item The characters E,F,H,K,L,M,N, and T all deformation retract onto the charachter I, while the letter Q deformation retracts onto the letter O.
\end{itemize}
%%%%%
%%%%%
\end{document}
