\documentclass[12pt]{article}
\usepackage{pmmeta}
\pmcanonicalname{DeformationRetractIsTransitive}
\pmcreated{2013-03-22 15:43:59}
\pmmodified{2013-03-22 15:43:59}
\pmowner{mps}{409}
\pmmodifier{mps}{409}
\pmtitle{deformation retract is transitive}
\pmrecord{4}{37683}
\pmprivacy{1}
\pmauthor{mps}{409}
\pmtype{Result}
\pmcomment{trigger rebuild}
\pmclassification{msc}{55Q05}

\endmetadata

% this is the default PlanetMath preamble.  as your knowledge
% of TeX increases, you will probably want to edit this, but
% it should be fine as is for beginners.

% almost certainly you want these
\usepackage{amssymb}
\usepackage{amsmath}
\usepackage{amsfonts}

% used for TeXing text within eps files
%\usepackage{psfrag}
% need this for including graphics (\includegraphics)
%\usepackage{graphicx}
% for neatly defining theorems and propositions
\usepackage{amsthm}
% making logically defined graphics
%%%\usepackage{xypic}

% there are many more packages, add them here as you need them

% define commands here
\newtheorem*{proposition*}{Proposition}
\DeclareMathOperator{\id}{id}
\begin{document}
\PMlinkescapeword{between}
\begin{proposition*}
Let $Z\subset Y\subset X$ be nested topological spaces.  If there exist a 
\PMlinkname{deformation retraction}{DeformationRetraction} of $X$ onto $Y$ and a deformation retraction of $Y$ onto $Z$,
then there also exists a deformation retraction of $X$ onto $Z$.  In other words,
``being a deformation retract of'' is a transitive relation.
\end{proposition*}

\begin{proof}
Since $Y$ is a deformation retract of $X$, there is a homotopy 
$F:I\times X\to X$ between $\id_X$ and a retract $r:X\to Y$ of 
$X$ onto $Y$.  Similarly, there is a homotopy $G:I\times Y\to Y$
between $\id_Y$ and a retract $s:Y\to Z$ of $Y$ onto $Z$.

First notice that since both $r$ and $s$ fix $Z$, the map $sr:X\to Z$
is a retraction.

Now define a map $\widetilde{G}:I\times X\to X$ by 
$\widetilde{G}=iG(\id_I\times r)$, where $i:Y\hookrightarrow X$ is 
inclusion.  Observe that
\begin{itemize}
\item
$\widetilde{G}(0,x)=r(x)$ for any $x\in X$;
\item
$\widetilde{G}(1,x)=sr(x)$ for any $x\in X$; and
\item
$\widetilde{G}(t,a)=a$ for any $a\in Z$.
\end{itemize}
Hence $\widetilde{G}$ is a homotopy between the retractions $r$ and $sr$.

Finally we must 
\PMlinkname{glue together the homotopies}{GluingTogentherContinuousFunctions} $F$ and $\widetilde{G}$ to get a 
homotopy between $\id_X$ and $sr$.  To do this, define a function
$H:I\times X\to X$ by
\[
H(t,x)=\begin{cases}
F(2t,x),                & 0\le t\le\frac{1}{2} \\
\widetilde{G}(2t-1, x), & \frac{1}{2}\le t\le 1.
\end{cases}
\]
Since $F(1,x)=\widetilde{G}(0,x)=r(x)$, the gluing yieds a continuous map.
By construction, 
\begin{itemize}
\item
$H(0,x)=x$ for all $x\in X$;
\item
$H(1,x)=sr(x)$ for all $x\in X$; and
\item
$H(t,a)=a$ for any $a\in Z$.
\end{itemize}
Hence $H$ is a homotopy between the identity map on $X$ and a retraction of $X$ onto $Z$.  We conclude that $H$ is a deformation retraction of $X$ onto $Z$.
\end{proof}
%%%%%
%%%%%
\end{document}
