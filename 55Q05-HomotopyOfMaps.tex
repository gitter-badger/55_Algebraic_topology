\documentclass[12pt]{article}
\usepackage{pmmeta}
\pmcanonicalname{HomotopyOfMaps}
\pmcreated{2013-03-22 12:13:19}
\pmmodified{2013-03-22 12:13:19}
\pmowner{mathcam}{2727}
\pmmodifier{mathcam}{2727}
\pmtitle{homotopy of maps}
\pmrecord{12}{31584}
\pmprivacy{1}
\pmauthor{mathcam}{2727}
\pmtype{Definition}
\pmcomment{trigger rebuild}
\pmclassification{msc}{55Q05}
\pmsynonym{homotopic maps}{HomotopyOfMaps}
\pmrelated{HomotopyOfPaths}
\pmrelated{HomotopyEquivalence}
\pmrelated{ConstantFunction}
\pmrelated{Contractible}
\pmdefines{homotopic}
\pmdefines{nullhomotopic}

% this is the default PlanetMath preamble.  as your knowledge
% of TeX increases, you will probably want to edit this, but
% it should be fine as is for beginners.

% almost certainly you want these
\usepackage{amssymb}
\usepackage{amsmath}
\usepackage{amsfonts}

% used for TeXing text within eps files
%\usepackage{psfrag}
% need this for including graphics (\includegraphics)
%\usepackage{graphicx}
% for neatly defining theorems and propositions
%\usepackage{amsthm}
% making logically defined graphics
%%%%\usepackage{xypic} 

% there are many more packages, add them here as you need them

% define commands here
\begin{document}
Let $X,Y$ be topological spaces, $A$ a closed subspace of $X$ and $f,g:X \to Y$ continuous maps. A \emph{homotopy of maps} is a continuous function $F:X \times [0,1] \to Y$ satisfying
\begin{enumerate}
\item $F(x,0)=f(x)$ for all $x \in X$
\item $F(x,1)=g(x)$ for all $x \in X$
\item $F(x,t)=f(x)=g(x)$ for all $x \in A, t\in [0,1]$.
\end{enumerate}
We say that $f$ is homotopic to $g$ relative to $A$ and denote this by $f \simeq g$ $rel A$. If $A=\emptyset$, this can be written $f \simeq g$. If $g$ is the constant map (i.e. $g(x)=y$  for all $x \in X$), then we say that $f$ is \emph{nullhomotopic}.
%%%%%
%%%%%
%%%%%
\end{document}
