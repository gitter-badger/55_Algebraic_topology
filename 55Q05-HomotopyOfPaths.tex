\documentclass[12pt]{article}
\usepackage{pmmeta}
\pmcanonicalname{HomotopyOfPaths}
\pmcreated{2013-03-22 12:13:16}
\pmmodified{2013-03-22 12:13:16}
\pmowner{RevBobo}{4}
\pmmodifier{RevBobo}{4}
\pmtitle{homotopy of paths}
\pmrecord{8}{31576}
\pmprivacy{1}
\pmauthor{RevBobo}{4}
\pmtype{Definition}
\pmcomment{trigger rebuild}
\pmclassification{msc}{55Q05}
\pmsynonym{homotopic paths}{HomotopyOfPaths}
\pmsynonym{continuous deformation}{HomotopyOfPaths}
\pmsynonym{homotopy}{HomotopyOfPaths}
\pmrelated{HomotopyOfMaps}
\pmrelated{HomotopyWithAContractibleDomain}
\pmrelated{PathConnectnessAsAHomotopyInvariant}

\endmetadata

% this is the default PlanetMath preamble.  as your knowledge
% of TeX increases, you will probably want to edit this, but
% it should be fine as is for beginners.

% almost certainly you want these
\usepackage{amssymb}
\usepackage{amsmath}
\usepackage{amsfonts}

% used for TeXing text within eps files
%\usepackage{psfrag}
% need this for including graphics (\includegraphics)
%\usepackage{graphicx}
% for neatly defining theorems and propositions
%\usepackage{amsthm}
% making logically defined graphics
%%%%\usepackage{xypic} 

% there are many more packages, add them here as you need them

% define commands here
\begin{document}
Let $X$ be a topological space and $p,q$ paths in $X$ with the same initial point $x_{0}$ and terminal point $x_{1}$. If there exists a continuous function $F: I \times I \to X$ such that
\begin{enumerate}
\item $F(s,0)=p(s)$ for all $s \in I$
\item $F(s,1)=q(s)$ for all $s \in I$
\item $F(0,t)=x_{0}$ for all $t \in I$
\item $F(1,t)=x_{1}$ for all $t \in I$
\end{enumerate}

we call $F$ a \emph{homotopy of paths} in $X$ and say $p,q$ are \emph{homotopic paths} in $X$. $F$ is also called a \emph{continuous deformation}.
%%%%%
%%%%%
%%%%%
\end{document}
