\documentclass[12pt]{article}
\usepackage{pmmeta}
\pmcanonicalname{VanKampensTheoremResult}
\pmcreated{2013-03-22 14:09:23}
\pmmodified{2013-03-22 14:09:23}
\pmowner{mathcam}{2727}
\pmmodifier{mathcam}{2727}
\pmtitle{Van Kampen's theorem result}
\pmrecord{6}{35576}
\pmprivacy{1}
\pmauthor{mathcam}{2727}
\pmtype{Result}
\pmcomment{trigger rebuild}
\pmclassification{msc}{55Q05}

\usepackage{amssymb}
\usepackage{amsmath}
\usepackage{amsfonts}

% used for TeXing text within eps files
%\usepackage{psfrag}
% need this for including graphics (\includegraphics)
%\usepackage{graphicx}
% for neatly defining theorems and propositions
%\usepackage{amsthm}
% making logically defined graphics
%%\usepackage{xypic}
\begin{document}
There is a more general version of the theorem of van Kampen which involves 
the fundamental groupoid $\pi_1(X,A)$ on a {\it set $A$} of base points, 
defined as the full subgroupoid of $\pi_1(X)$ on the set $A \cap X$. This 
allows one to compute the fundamental group of the circle $S^1$ and many more 
cases. 

{\bf Theorem} If $X$ is the union of open sets $U,V$ with intersection $W$,
and $A$ meets each path component of  $U,V,W$ then the following induced diagram 
$$\xymatrix{ \pi_1(W,A) \ar [r] \ar [d] & \pi_1(U,A) \ar [d] \\
\pi_1(V,A)  \ar [r] & \pi_1(X,A)} $$
is a pushout in the category of groupoids. 

This may be found in R. Brown's book ``Topology: a geometric account of general topology
and the fundamental groupoid'', Ellis Horwood 1988 (first edition McGraw Hill, 1968).
It has many useful applications, and was the guide for higher dimensional theorems 
involving higher homotopy groupoids.
%%%%%
%%%%%
\end{document}
