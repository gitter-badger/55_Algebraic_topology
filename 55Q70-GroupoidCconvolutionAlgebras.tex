\documentclass[12pt]{article}
\usepackage{pmmeta}
\pmcanonicalname{GroupoidCconvolutionAlgebras}
\pmcreated{2013-03-22 18:13:59}
\pmmodified{2013-03-22 18:13:59}
\pmowner{bci1}{20947}
\pmmodifier{bci1}{20947}
\pmtitle{groupoid C*-convolution algebras}
\pmrecord{63}{40820}
\pmprivacy{1}
\pmauthor{bci1}{20947}
\pmtype{Topic}
\pmcomment{trigger rebuild}
\pmclassification{msc}{55Q70}
\pmclassification{msc}{55Q55}
\pmclassification{msc}{55Q52}
\pmclassification{msc}{81R50}
\pmclassification{msc}{22A22}
\pmclassification{msc}{81R10}
\pmclassification{msc}{20L05}
\pmclassification{msc}{18B40}
\pmsynonym{convolution algebra}{GroupoidCconvolutionAlgebras}
%\pmkeywords{groupoid convolution algebras and quantum state space structures}
%\pmkeywords{classification theory of C*-algebras}
\pmrelated{GroupCAlgebra}
\pmrelated{CAlgebra}
\pmrelated{CAlgebra3}
\pmrelated{Convolution}
\pmrelated{NuclearCAlgebra}
\pmrelated{QuantumGravityTheories}
\pmrelated{Algebras2}
\pmrelated{C_cG}
\pmrelated{HomomorphismsOfCAlgebrasAreContinuous}
\pmrelated{SigmaFiniteBorelMeasureAndRelatedBorelConcepts}
\pmdefines{groupoid C*-convolution algebra}
\pmdefines{groupoid C*-algebra}
\pmdefines{groupoid $C^*$-algebra}
\pmdefines{Haar systems}
\pmdefines{C*-convolution}
\pmdefines{measured groupoid}

\endmetadata

% this is the default PlanetMath preamble.  as your knowledge
% of TeX increases, you will probably want to edit this, but
% it should be fine as is for beginners.

% almost certainly you want these
\usepackage{amssymb}
\usepackage{amsmath}
\usepackage{amsfonts}

% used for TeXing text within eps files
%\usepackage{psfrag}
% need this for including graphics (\includegraphics)
%\usepackage{graphicx}
% for neatly defining theorems and propositions
%\usepackage{amsthm}
% making logically defined graphics
%%%\usepackage{xypic}

% there are many more packages, add them here as you need them

% define commands here
\usepackage{amsmath, amssymb, amsfonts, amsthm, amscd, latexsym, enumerate}
%%\usepackage{xypic}
\usepackage[mathscr]{eucal}

\setlength{\textwidth}{6.5in}
%\setlength{\textwidth}{16cm}
\setlength{\textheight}{9.0in}
%\setlength{\textheight}{24cm}

\hoffset=-.75in     %%ps format
%\hoffset=-1.0in     %%hp format
\voffset=-.4in


\theoremstyle{plain}
\newtheorem{lemma}{Lemma}[section]
\newtheorem{proposition}{Proposition}[section]
\newtheorem{theorem}{Theorem}[section]
\newtheorem{corollary}{Corollary}[section]

\theoremstyle{definition}
\newtheorem{definition}{Definition}[section]
\newtheorem{example}{Example}[section]
%\theoremstyle{remark}
\newtheorem{remark}{Remark}[section]
\newtheorem*{notation}{Notation}
\newtheorem*{claim}{Claim}

\renewcommand{\thefootnote}{\ensuremath{\fnsymbol{footnote}}}
\numberwithin{equation}{section}

\newcommand{\Ad}{{\rm Ad}}
\newcommand{\Aut}{{\rm Aut}}
\newcommand{\Cl}{{\rm Cl}}
\newcommand{\Co}{{\rm Co}}
\newcommand{\DES}{{\rm DES}}
\newcommand{\Diff}{{\rm Diff}}
\newcommand{\Dom}{{\rm Dom}}
\newcommand{\Hol}{{\rm Hol}}
\newcommand{\Mon}{{\rm Mon}}
\newcommand{\Hom}{{\rm Hom}}
\newcommand{\Ker}{{\rm Ker}}
\newcommand{\Ind}{{\rm Ind}}
\newcommand{\IM}{{\rm Im}}
\newcommand{\Is}{{\rm Is}}
\newcommand{\ID}{{\rm id}}
\newcommand{\GL}{{\rm GL}}
\newcommand{\Iso}{{\rm Iso}}
\newcommand{\rO}{{\rm O}}
\newcommand{\Sem}{{\rm Sem}}
\newcommand{\St}{{\rm St}}
\newcommand{\Sym}{{\rm Sym}}
\newcommand{\SU}{{\rm SU}}
\newcommand{\Tor}{{\rm Tor}}
\newcommand{\U}{{\rm U}}

\newcommand{\A}{\mathcal A}
\newcommand{\Ce}{\mathcal C}
\newcommand{\D}{\mathcal D}
\newcommand{\E}{\mathcal E}
\newcommand{\F}{\mathcal F}
\newcommand{\G}{\mathcal G}
\renewcommand{\H}{\mathcal H}
\renewcommand{\cL}{\mathcal L}
\newcommand{\Q}{\mathcal Q}
\newcommand{\R}{\mathcal R}
\newcommand{\cS}{\mathcal S}
\newcommand{\cU}{\mathcal U}
\newcommand{\W}{\mathcal W}

\newcommand{\bA}{\mathbb{A}}
\newcommand{\bB}{\mathbb{B}}
\newcommand{\bC}{\mathbb{C}}
\newcommand{\bD}{\mathbb{D}}
\newcommand{\bE}{\mathbb{E}}
\newcommand{\bF}{\mathbb{F}}
\newcommand{\bG}{\mathbb{G}}
\newcommand{\bK}{\mathbb{K}}
\newcommand{\bM}{\mathbb{M}}
\newcommand{\bN}{\mathbb{N}}
\newcommand{\bO}{\mathbb{O}}
\newcommand{\bP}{\mathbb{P}}
\newcommand{\bR}{\mathbb{R}}
\newcommand{\bV}{\mathbb{V}}
\newcommand{\bZ}{\mathbb{Z}}

\newcommand{\bfE}{\mathbf{E}}
\newcommand{\bfX}{\mathbf{X}}
\newcommand{\bfY}{\mathbf{Y}}
\newcommand{\bfZ}{\mathbf{Z}}

\renewcommand{\O}{\Omega}
\renewcommand{\o}{\omega}
\newcommand{\vp}{\varphi}
\newcommand{\vep}{\varepsilon}

\newcommand{\diag}{{\rm diag}}
\newcommand{\grp}{{\mathsf{G}}}
\newcommand{\dgrp}{{\mathsf{D}}}
\newcommand{\desp}{{\mathsf{D}^{\rm{es}}}}
\newcommand{\Geod}{{\rm Geod}}
\newcommand{\geod}{{\rm geod}}
\newcommand{\hgr}{{\mathsf{H}}}
\newcommand{\mgr}{{\mathsf{M}}}
\newcommand{\ob}{{\rm Ob}}
\newcommand{\obg}{{\rm Ob(\mathsf{G)}}}
\newcommand{\obgp}{{\rm Ob(\mathsf{G}')}}
\newcommand{\obh}{{\rm Ob(\mathsf{H})}}
\newcommand{\Osmooth}{{\Omega^{\infty}(X,*)}}
\newcommand{\ghomotop}{{\rho_2^{\square}}}
\newcommand{\gcalp}{{\mathsf{G}(\mathcal P)}}

\newcommand{\rf}{{R_{\mathcal F}}}
\newcommand{\glob}{{\rm glob}}
\newcommand{\loc}{{\rm loc}}
\newcommand{\TOP}{{\rm TOP}}

\newcommand{\wti}{\widetilde}
\newcommand{\what}{\widehat}

\renewcommand{\a}{\alpha}
\newcommand{\be}{\beta}
\newcommand{\ga}{\gamma}
\newcommand{\Ga}{\Gamma}
\newcommand{\de}{\delta}
\newcommand{\del}{\partial}
\newcommand{\ka}{\kappa}
\newcommand{\si}{\sigma}
\newcommand{\ta}{\tau}

\newcommand{\med}{\medbreak}
\newcommand{\medn}{\medbreak \noindent}
\newcommand{\bign}{\bigbreak \noindent}

\newcommand{\lra}{{\longrightarrow}}
\newcommand{\ra}{{\rightarrow}}
\newcommand{\rat}{{\rightarrowtail}}
\newcommand{\ovset}[1]{\overset {#1}{\ra}}
\newcommand{\ovsetl}[1]{\overset {#1}{\lra}}
\newcommand{\hr}{{\hookrightarrow}}
 
\newcommand{\<}{{\langle}}

%\usepackage{geometry, amsmath,amssymb,latexsym,enumerate}
%%%\usepackage{xypic}

\def\baselinestretch{1.1}


\hyphenation{prod-ucts}

%\geometry{textwidth= 16 cm, textheight=21 cm}

\newcommand{\sqdiagram}[9]{$$ \diagram  #1  \rto^{#2} \dto_{#4}&
#3  \dto^{#5} \\ #6    \rto_{#7}  &  #8   \enddiagram
\eqno{\mbox{#9}}$$ }

\def\C{C^{\ast}}

\newcommand{\labto}[1]{\stackrel{#1}{\longrightarrow}}

%\newenvironment{proof}{\noindent {\bf Proof} }{ \hfill $\Box$
%{\mbox{}}

\newcommand{\quadr}[4]{\begin{pmatrix} & #1& \\[-1.1ex] #2 & & #3\\[-1.1ex]& #4&
 \end{pmatrix}}
\def\D{\mathsf{D}}
\begin{document}
\subsection{Introduction: Background and definition of the groupoid C*--convolution algebra}

 Jean Renault introduced in ref. \cite{JR80} the \emph{$C^*$--algebra of a locally compact groupoid $\grp$} as follows: the space of continuous functions with compact support on a groupoid $\grp$ is made into a *-algebra whose multiplication is the \emph{convolution}, and that is also endowed with the smallest $C^*$--norm which makes its representations continuous,  as shown in ref.\cite{MAB2k3}. Furthermore, for this convolution to be defined, one needs also to have a \PMlinkname{Haar system}{GroupoidRepresentationsInducedByMeasure} 
associated to the \PMlinkname{locally compact groupoids}{LocallyCompactGroupoids} $\grp$ 
that are then called \emph{measured groupoids} because they are endowed with an associated Haar system which involves the concept of measure, as introduced in ref. \cite{Hahn1} by P. Hahn.

 With these concepts one can now sum up the definition (or construction) of the \emph{groupoid $C^*$-convolution algebra}, or \PMlinkexternal{groupoid $C^*$-algebra}{http://www.utgjiu.ro/math/mbuneci/preprint/p0024.pdf}, as follows.

\begin{definition}  a {\em groupoid C*--convolution algebra}, $G_{CA}$, is defined for \emph{measured groupoids}
as a \emph{*--algebra with ``$*$'' being defined by convolution so that it has a smallest $C^*$--norm which makes its representations continuous}.
\end{definition}

\begin{remark}
One can also produce a functorial construction of $G_{CA}$ that has additional interesting properties. 
\end{remark}

Next we recall a result due to P. Hahn \cite{PH78} which shows how groupoid representations relate to
induced *-algebra representations and also how--under certain conditions-- the former can be derived from
the appropriate *-algebra representations. 

\begin{theorem} 
(source: ref. \cite{PH78}). Any representation of a groupoid $(\grp,C)$ with Haar measure $(\nu, \mu)$ in a separable Hilbert space $\H$ induces a *-algebra representation $f \mapsto X_f$ of the associated 
groupoid algebra $ \Pi (\grp, \nu)$ in $L^2 (U_{\grp} , \mu, \H )$ with the following properties:

(1)  For any $l,m \in \H $ , one has that $\left|<X_f(u \mapsto l), (u \mapsto m)>\right|\leq \left\|f_l\right\| \left\|l \right\| \left\|m \right\|$ and
\med
(2)  $M_r (\alpha) X_f = X_{f \alpha \circ r}$,  where
\med 
  $M_r: L^\infty (U_{\grp}, \mu   \longrightarrow L[L^2 (U_{\grp}, \mu, \H]$, with
  
  $M_r (\alpha)j = \alpha \cdot j$.

\textit{Conversely, any *- algebra representation with the above two properties induces a groupoid representation, X, as follows:}
\med
$<X_f , j, k> ~ = ~ \displaystyle{\int} f(x)[X(x)j(d(x)),k(r(x))d \nu (x)].$
(viz. p. 50 of ref. \cite{PH78}).
\end{theorem}

Furthermore, according to Seda (ref. \cite {Seda86,Seda2k8}), the continuity of a Haar system is equivalent to the continuity of the convolution product $f*g$ for any pair $f$, $g$ of continuous functions with compact support. One may thus conjecture that similar results could be obtained for functions with \textit{locally compact} support in dealing with convolution products of either locally compact groupoids or quantum groupoids. Seda's result also implies that the convolution algebra $C_c (\G)$ of a groupoid $\G$ is closed with respect to convolution if and only if the fixed Haar system associated with the measured groupoid $\G$ is \textit{continuous} (see ref. \cite{MAB2k3}).

Thus, in the case of groupoid algebras of transitive groupoids, it was shown in \cite{MAB2k3} that any representation of a measured groupoid $(\G, [\displaystyle{\int} \nu ^u d \tilde{\lambda}(u)] = [\lambda])$ on a separable Hilbert space $\H$ induces a \textit{non-degenerate} *-representation  $f \mapsto  X_f$ of the associated groupoid algebra 
$\Pi (\G, \nu,\tilde{\lambda})$ with properties formally similar to (1) and (2) above.  
Moreover, as in the case of groups, \emph{there is a correspondence between the unitary representations of a groupoid and its associated C*-convolution algebra representations} (p. 182 of \cite{MAB2k3}), the latter involving however fiber bundles of Hilbert spaces instead of single Hilbert spaces.

\begin{thebibliography}{9}
      
\bibitem{Hahn1}
P. Hahn: Haar measure for measure groupoids., \textit{Trans. Amer. Math. Soc}. \textbf{242}: 1--33(1978).

\bibitem{PH78}
P. Hahn: The regular representations of measure groupoids., \textit{Trans. Amer. Math. Soc}. \textbf{242}:35--72(1978).
Theorem 3.4 on p. 50.

\bibitem{MAB2k3}
M. R. Buneci. \emph{Groupoid Representations}, Ed. Mirton: Timishoara (2003). 

\bibitem{MRB2k6}
M.R. Buneci. 2006.,
\PMlinkexternal{Groupoid C*-Algebras.}{http://www.utgjiu.ro/math/mbuneci/preprint/p0024.pdf},
{\em Surveys in Mathematics and its Applications}, Volume 1: 71--98.

\bibitem{MAB2k5}
M. R. Buneci. Isomorphic groupoid C*-algebras associated with
different Haar systems., {\em New York J. Math.}, \textbf{11} (2005):225--245.

\bibitem{JR80}
J. Renault. A groupoid approach to C*-algebras, \emph{Lecture Notes in Math}., 793, Springer,
Berlin, (1980).

\bibitem{JR97}
J. Renault. 1997. The Fourier Algebra of a Measured Groupoid and Its Multipliers, 
{\em Journal of Functional Analysis}, \textbf{145}, Number 2, April 1997, pp. 455--490.
 
\bibitem{Seda76}
A. K. Seda: Haar measures for groupoids, \emph{Proc. Roy. Irish Acad.
Sect. A} \textbf{76} No. 5, 25--36 (1976).

\bibitem{Seda82}
A. K. Seda: Banach bundles of continuous functions and an integral
representation theorem, \emph{Trans. Amer. Math. Soc.} \textbf{270} No.1 : 327-332(1982).

\bibitem{Seda86}
A. K. Seda: On the Continuity of Haar measures on topological groupoids, \emph{Proc. Amer Math. Soc.} \textbf{96}: 115--120 (1986).

\bibitem{Seda2k8}
A. K. Seda. 2008. \emph{Personal communication}, and also Seda (1986, on p.116).
\end{thebibliography}
%%%%%
%%%%%
\end{document}
