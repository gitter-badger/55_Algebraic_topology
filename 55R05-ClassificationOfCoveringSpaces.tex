\documentclass[12pt]{article}
\usepackage{pmmeta}
\pmcanonicalname{ClassificationOfCoveringSpaces}
\pmcreated{2013-03-22 13:27:53}
\pmmodified{2013-03-22 13:27:53}
\pmowner{Dr_Absentius}{537}
\pmmodifier{Dr_Absentius}{537}
\pmtitle{classification of covering spaces}
\pmrecord{8}{34033}
\pmprivacy{1}
\pmauthor{Dr_Absentius}{537}
\pmtype{Definition}
\pmcomment{trigger rebuild}
\pmclassification{msc}{55R05}
\pmclassification{msc}{55R15}
\pmrelated{EtaleFundamentalGroup}
\pmrelated{DeckTransformation}

%\documentclass{amsart}
\usepackage{amsmath}
\usepackage[all,poly,knot,dvips]{xy}
%\usepackage{pstricks,pst-poly,pst-node,pstcol}


\usepackage{amssymb,latexsym}

\usepackage{amsthm,latexsym}
\usepackage{eucal,latexsym}

% THEOREM Environments --------------------------------------------------

\newtheorem{thm}{Theorem}
 \newtheorem*{Thm}{Theorem}
 \newtheorem{cor}[thm]{Corollary}
 \newtheorem{lem}[thm]{Lemma}
 \newtheorem{prop}[thm]{Proposition}
 \newtheorem{claim}[thm]{Claim}
 \theoremstyle{definition}
 \newtheorem{defn}[thm]{Definition}
 \theoremstyle{remark}
 \newtheorem{rem}[thm]{Remark}
 \numberwithin{equation}{subsection}


%---------------------  Greek letters, etc ------------------------- 

\newcommand{\CA}{\mathcal{A}}
\newcommand{\CC}{\mathcal{C}}
\newcommand{\CM}{\mathcal{M}}
\newcommand{\CP}{\mathcal{P}}
\newcommand{\CS}{\mathcal{S}}
\newcommand{\BC}{\mathbb{C}}
\newcommand{\BN}{\mathbb{N}}
\newcommand{\BR}{\mathbb{R}}
\newcommand{\BZ}{\mathbb{Z}}
\newcommand{\FF}{\mathfrak{F}}
\newcommand{\FL}{\mathfrak{L}}
\newcommand{\FM}{\mathfrak{M}}
\newcommand{\Ga}{\alpha}
\newcommand{\Gb}{\beta}
\newcommand{\Gg}{\gamma}
\newcommand{\GG}{\Gamma}
\newcommand{\Gd}{\delta}
\newcommand{\GD}{\Delta}
\newcommand{\Ge}{\varepsilon}
\newcommand{\Gz}{\zeta}
\newcommand{\Gh}{\eta}
\newcommand{\Gq}{\theta}
\newcommand{\GQ}{\Theta}
\newcommand{\Gi}{\iota}
\newcommand{\Gk}{\kappa}
\newcommand{\Gl}{\lambda}
\newcommand{\GL}{\Lamda}
\newcommand{\Gm}{\mu}
\newcommand{\Gn}{\nu}
\newcommand{\Gx}{\xi}
\newcommand{\GX}{\Xi}
\newcommand{\Gp}{\pi}
\newcommand{\GP}{\Pi}
\newcommand{\Gr}{\rho}
\newcommand{\Gs}{\sigma}
\newcommand{\GS}{\Sigma}
\newcommand{\Gt}{\tau}
\newcommand{\Gu}{\upsilon}
\newcommand{\GU}{\Upsilon}
\newcommand{\Gf}{\varphi}
\newcommand{\GF}{\Phi}
\newcommand{\Gc}{\chi}
\newcommand{\Gy}{\psi}
\newcommand{\GY}{\Psi}
\newcommand{\Gw}{\omega}
\newcommand{\GW}{\Omega}
\newcommand{\Gee}{\epsilon}
\newcommand{\Gpp}{\varpi}
\newcommand{\Grr}{\varrho}
\newcommand{\Gff}{\phi}
\newcommand{\Au}{ \operatorname{Aut}}
\newcommand{\id}{ \operatorname{id}}
\newcommand{\Gss}{\varsigma}

\def\co{\colon\thinspace}
\begin{document}
Let $X$ be a connected, locally path connected and semilocally simply
connected space. Assume furthermore that $X$ has a basepoint $*$.

A covering $p\co E\to X$ is called \emph{based} if $E$ is endowed with a
basepoint $e$ and $p(e)=*$. Two based coverings $p_i\co E_i\to X$, $i=1,2$ are called
equivalent if there is a basepoint preserving equivalence $T\co E_1\to E_2$ that
covers the identity, i.e. $T$ is a homeomorphism and the following diagram
commutes
 $$\xymatrix{  {(E_1,e_1)}\ar[dr]_{p}\ar[rr]^{T}&&{(E_2,e_2)}\ar[dl]^{p}\\ 
 &{(X,*)}.&  }$$


\begin{Thm}[\textbf{Classification of connected coverings}]$ $
  \begin{itemize}
  \item  Equivalence classes of based coverings $p\co (E,e)\to (X,*)$ with connected total
  space $E$ are in bijective correspondence with subgroups of the
  fundamental group $\pi_1(X,*)$. The bijection assigns to the based covering
  $p$ the subgroup $p_*\left(\pi_1(E,e)\right)$.
\item Equivalence classes of coverings (not based) are in bijective
  correspondence with conjugacy class of subgroups of $\pi_1(X,*)$.
  \end{itemize}
  \end{Thm}

Under the bijection of the above theorem normal coverings correspond to
normal subgroups of $\pi_1(X,e)$, and in particular the universal covering
$\tilde \pi\co \tilde X\to X$
corresponds to the trivial subgroup while the trivial covering $\id\co X\to X$
corresponds to the whole group.

Normal coverings are sometimes called Galois coverings, and the group of deck transformations is someitmes called the Galois group of the covering space.  The reason for this is that this theorem provides a direct analogy for the fundamental theorem of Galois theory. This theorem provides a correspondence between subgroups of the Galois group of a cover and covers that are its quotients, just as the fundamental theorem of Galois theory provides a correspondence between subextensions of a field extension and subgroups of its Galois group. Both fundamental theorems can be viewed as special cases of a more general theorem in the category of schemes; the correct tool is the study of \'etale morphisms and the \'etale fundamental group.

\begin{proof}[Rough sketch of proof] We describe the based version.  
Clearly the set of equivalences of two based coverings form a torsor of the
group of deck transformations $\Au(p)$. From our discussion of that group it
follows then that equivalent (based) coverings give the same subgroup. Thus
the map is well defined. To see that it is a bijection construct its inverse
as follows: There is a universal covering $\tilde \pi\co \tilde X\to X$ and
a subgroup $\pi$ of $\pi_1(X,*)$ acts on $\tilde X$ by the restriction of
the monodromy action. The covering which corresponds to $\pi$ is then
$\tilde X/\pi$.
\end{proof}
%%%%%
%%%%%
\end{document}
