\documentclass[12pt]{article}
\usepackage{pmmeta}
\pmcanonicalname{AssociatedBundleConstruction}
\pmcreated{2013-03-22 13:26:46}
\pmmodified{2013-03-22 13:26:46}
\pmowner{rspuzio}{6075}
\pmmodifier{rspuzio}{6075}
\pmtitle{associated bundle construction}
\pmrecord{9}{34009}
\pmprivacy{1}
\pmauthor{rspuzio}{6075}
\pmtype{Definition}
\pmcomment{trigger rebuild}
\pmclassification{msc}{55R10}
\pmdefines{associated bundle}

%\documentclass{amsart}
\usepackage{amsmath}
%\usepackage[all,poly,knot,dvips]{xy}
%\usepackage{pstricks,pst-poly,pst-node,pstcol}


\usepackage{amssymb,latexsym}

\usepackage{amsthm,latexsym}
\usepackage{eucal,latexsym}

% THEOREM Environments --------------------------------------------------

\newtheorem{thm}{Theorem}
 \newtheorem*{mainthm}{Main~Theorem}
 \newtheorem{cor}[thm]{Corollary}
 \newtheorem{lem}[thm]{Lemma}
 \newtheorem{prop}[thm]{Proposition}
 \newtheorem{claim}[thm]{Claim}
 \theoremstyle{definition}
 \newtheorem{defn}[thm]{Definition}
 \theoremstyle{remark}
 \newtheorem{rem}[thm]{Remark}
 \numberwithin{equation}{subsection}


%---------------------  Greek letters, etc ------------------------- 

\newcommand{\CA}{\mathcal{A}}
\newcommand{\CC}{\mathcal{C}}
\newcommand{\CM}{\mathcal{M}}
\newcommand{\CP}{\mathcal{P}}
\newcommand{\CS}{\mathcal{S}}
\newcommand{\BC}{\mathbb{C}}
\newcommand{\BN}{\mathbb{N}}
\newcommand{\BR}{\mathbb{R}}
\newcommand{\BZ}{\mathbb{Z}}
\newcommand{\FF}{\mathfrak{F}}
\newcommand{\FL}{\mathfrak{L}}
\newcommand{\FM}{\mathfrak{M}}
\newcommand{\Ga}{\alpha}
\newcommand{\Gb}{\beta}
\newcommand{\Gg}{\gamma}
\newcommand{\GG}{\Gamma}
\newcommand{\Gd}{\delta}
\newcommand{\GD}{\Delta}
\newcommand{\Ge}{\varepsilon}
\newcommand{\Gz}{\zeta}
\newcommand{\Gh}{\eta}
\newcommand{\Gq}{\theta}
\newcommand{\GQ}{\Theta}
\newcommand{\Gi}{\iota}
\newcommand{\Gk}{\kappa}
\newcommand{\Gl}{\lambda}
\newcommand{\GL}{\Lamda}
\newcommand{\Gm}{\mu}
\newcommand{\Gn}{\nu}
\newcommand{\Gx}{\xi}
\newcommand{\GX}{\Xi}
\newcommand{\Gp}{\pi}
\newcommand{\GP}{\Pi}
\newcommand{\Gr}{\rho}
\newcommand{\Gs}{\sigma}
\newcommand{\GS}{\Sigma}
\newcommand{\Gt}{\tau}
\newcommand{\Gu}{\upsilon}
\newcommand{\GU}{\Upsilon}
\newcommand{\Gf}{\varphi}
\newcommand{\GF}{\Phi}
\newcommand{\Gc}{\chi}
\newcommand{\Gy}{\psi}
\newcommand{\GY}{\Psi}
\newcommand{\Gw}{\omega}
\newcommand{\GW}{\Omega}
\newcommand{\Gee}{\epsilon}
\newcommand{\Gpp}{\varpi}
\newcommand{\Grr}{\varrho}
\newcommand{\Gff}{\phi}
\newcommand{\Gss}{\varsigma}
\newcommand{\Au}{\text{Aut}}
\def\co{\colon\thinspace}
\begin{document}
Let $G$ be a topological group, $\pi\co P\to X$ a (right) principal $G$-bundle,
$F$  a topological  space and $\Gr\co G\to \Au(F)$ a 
representation of  $G$ as homeomorphisms of $F$. Then the fiber bundle
\emph{associated} to $P$ by $\Gr$, is a fiber bundle $\pi_\Gr\co P\times_\Gr F 
\to X$ with fiber $F$ and group $G$ that is defined as follows:
\begin{itemize}
\item The total space is defined as 
$$P\times_\Gr F:=P\times F/G$$
where the (left) action of $G$ on $P\times F$ is defined by
$$g\cdot (p,f):=\left(pg^{-1},\Gr(g)(f)\right), \quad \forall g\in G, p\in P, F\in
F\,.$$
\item The projection $\pi_\Gr$ is defined by 
$$\pi_\Gr[p,f]:=\pi(p)\,,$$
where $[p,f]$ denotes the $G$--orbit of $(p,f)\in P\times F$. 
\end{itemize}

\begin{thm}
  The above is well defined and  defines a $G$--bundle over $X$ with fiber
  $F$. Furthermore $P\times_\Gr F$ has the same transition functions as $P$. 
\end{thm}

\begin{proof}[Sketch of proof]
  To see that $\pi_\Gr$ is well defined just notice that for $p\in P$ and
  $g\in G$,  $\pi(pg)=\pi(p)$. To see that the fiber is $F$ notice that since
  the principal action is simply transitive, given $p\in P$ any orbit of the
  $G$--action on $P\times F$ contains a \emph{unique} representative of the
  form $(p,f)$ for some $f\in F$. It is clear that an open cover that
  trivializes $P$ trivializes $P\times_\Gr F$ as well.
  To see that $P\times_\Gr F$ has the same
  transition functions as $P$ notice that transition functions of $P$  act on the
  left and thus commute with the principal $G$--action on $P$.  
\end{proof}

Notice that if $G$ is a Lie group, $P$ a smooth principal bundle and $F$ is a
smooth manifold and $\Gr$ maps inside the diffeomorphism group of $F$, the
above construction produces a smooth bundle. Also quite often $F$ has extra
structure and $\Gr$ maps into the homeomorphisms of $F$ that preserve that
structure. In that case the above construction produces a ``bundle of such
structures.''  For example    when $F$ is a vector space and
$\Gr(G) \subset \operatorname{GL}(F) $, i.e. $\Gr$ is a linear
representation  of $G$ we get
a vector bundle; if   $\Gr(G) \subset \operatorname{SL}(F)$ we get an
oriented vector bundle, etc.    

%% To be added: 
% It turns out that the above construction has an inverse. Namely let $\pi\co
% E \to X$ be a fiber bundle with group $G$ and fiber $F$. Define the
% associated principal bundle $\pi_p\co P \to X$ as follows. The fiber of
% $\pi_p$ over $x\in X$ consists
%%%%%
%%%%%
\end{document}
