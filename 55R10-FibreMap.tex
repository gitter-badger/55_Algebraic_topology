\documentclass[12pt]{article}
\usepackage{pmmeta}
\pmcanonicalname{FibreMap}
\pmcreated{2013-03-22 14:32:27}
\pmmodified{2013-03-22 14:32:27}
\pmowner{whm22}{2009}
\pmmodifier{whm22}{2009}
\pmtitle{fibre map}
\pmrecord{11}{36087}
\pmprivacy{1}
\pmauthor{whm22}{2009}
\pmtype{Definition}
\pmcomment{trigger rebuild}
\pmclassification{msc}{55R10}
\pmsynonym{fiber map}{FibreMap}
%\pmkeywords{fibre}
%\pmkeywords{base}
\pmrelated{FibreBundle}
\pmrelated{LocallyTrivialBundle}
\pmrelated{LongExactSequenceLocallyTrivialBundle}
\pmrelated{Fibration2}
\pmrelated{HomotopyLiftingProperty}
\pmdefines{fibre map}

\endmetadata

% this is the default PlanetMath preamble.  as your knowledge
% of TeX increases, you will probably want to edit this, but
% it should be fine as is for beginners.

% almost certainly you want these
\usepackage{amssymb}
\usepackage{amsmath}
\usepackage{amsfonts}

% used for TeXing text within eps files
%\usepackage{psfrag}
% need this for including graphics (\includegraphics)
%\usepackage{graphicx}
% for neatly defining theorems and propositions
%\usepackage{amsthm}
% making logically defined graphics
%%%\usepackage{xypic}

% there are many more packages, add them here as you need them

% define commands here
\begin{document}
A fibre map is a map of topological spaces $f: E \rightarrow B$, for which  there exists a space $F$, such that any point of $B$ is contained in some neighbourhood $U$ satisfying the following: on $f^{-1}U$, $f$ restricts to the natural projection $F \times U \rightarrow U$, via some homeomorphic identification of $f^{-1}U$ with $F \times U$.  

One class of examples are covering maps.  Another example is the map $SO_3 \rightarrow S^2$ sending a rotation to the point it sends the ``North Pole'' to.
%%%%%
%%%%%
\end{document}
