\documentclass[12pt]{article}
\usepackage{pmmeta}
\pmcanonicalname{FundamentalGroupoidFunctor}
\pmcreated{2013-03-22 18:12:03}
\pmmodified{2013-03-22 18:12:03}
\pmowner{bci1}{20947}
\pmmodifier{bci1}{20947}
\pmtitle{fundamental groupoid functor}
\pmrecord{48}{40779}
\pmprivacy{1}
\pmauthor{bci1}{20947}
\pmtype{Topic}
\pmcomment{trigger rebuild}
\pmclassification{msc}{55R10}
\pmclassification{msc}{55R65}
\pmclassification{msc}{22A22}
\pmclassification{msc}{55P99}
\pmclassification{msc}{20L05}
\pmclassification{msc}{18A30}
\pmclassification{msc}{55R37}
\pmsynonym{fundamental groupoid}{FundamentalGroupoidFunctor}
%\pmkeywords{fundamental groupoid from the category of topological spaces to the category of groupoids}
\pmrelated{FundamentalGroupoid}
\pmrelated{2Category}
\pmrelated{TopologicalSpace}
\pmrelated{HigherDimensionalAlgebraHDA}
\pmrelated{FundamentalGroupoid2}
\pmrelated{HomotopyDoubleGroupoidOfAHausdorffSpace}
\pmrelated{QuantumFundamentalGroupoids}
\pmrelated{HomotopyCategory}
\pmrelated{GrothendieckCategory}
\pmrelated{2CategoryOfDoubleGroupoids}
\pmrelated{DoubleCategory3}
\pmdefines{fundamental groupoid functor}
\pmdefines{double groupoid}
\pmdefines{double category}

% this is the default PlanetMath preamble.  as your 

% almost certainly you want these
\usepackage{amssymb}
\usepackage{amsmath}
\usepackage{amsfonts}

% define commands here
\usepackage{amsmath, amssymb, amsfonts, amsthm, amscd, latexsym,color,enumerate}
%%\usepackage{xypic}
\xyoption{curve}
\usepackage[mathscr]{eucal}
\newcommand{\directs}[2]{\def\objectstyle{\scriptstyle} \objectmargin={0pt}
\xy
(0,4)*+{}="a",(0,-2)*+{\rule{0em}{1.5ex}#2}="b",(7,4)*+{\;#1}="c"
\ar@{->} "a";"b" \ar @{->}"a";"c" \endxy }

\theoremstyle{plain}
\newtheorem{lemma}{Lemma}[section]
\newtheorem{proposition}{Proposition}[section]
\newtheorem{theorem}{Theorem}[section]
\newtheorem{corollary}{Corollary}[section]
\newtheorem{conjecture}{Conjecture}[section]
\theoremstyle{definition}
\newtheorem{definition}{Definition}[section]
\newtheorem{example}{Example}[section]
%\theoremstyle{remark}
\newtheorem{remark}{Remark}[section]
\newtheorem*{notation}{Notation}
\newtheorem*{claim}{Claim}
\theoremstyle{plain}
\renewcommand{\thefootnote}{\ensuremath{\fnsymbol{footnote}}}
\numberwithin{equation}{section}
\newcommand{\Ad}{{\rm Ad}}
\newcommand{\Aut}{{\rm Aut}}
\newcommand{\Cl}{{\rm Cl}}
\newcommand{\Co}{{\rm Co}}
\newcommand{\DES}{{\rm DES}}
\newcommand{\Diff}{{\rm Diff}}
\newcommand{\Dom}{{\rm Dom}}
\newcommand{\Hol}{{\rm Hol}}
\newcommand{\Mon}{{\rm Mon}}
\newcommand{\Hom}{{\rm Hom}}
\newcommand{\Ker}{{\rm Ker}}
\newcommand{\Ind}{{\rm Ind}}
\newcommand{\IM}{{\rm Im}}
\newcommand{\Is}{{\rm Is}}
\newcommand{\ID}{{\rm id}}
\newcommand{\GL}{{\rm GL}}
\newcommand{\Iso}{{\rm Iso}}
\newcommand{\Sem}{{\rm Sem}}
\newcommand{\St}{{\rm St}}
\newcommand{\Sym}{{\rm Sym}}
\newcommand{\SU}{{\rm SU}}
\newcommand{\Tor}{{\rm Tor}}
\newcommand{\U}{{\rm U}}
\newcommand{\A}{\mathcal A}
\newcommand{\D}{\mathcal D}
\newcommand{\E}{\mathcal E}
\newcommand{\F}{\mathcal F}
\newcommand{\G}{\mathcal G}
\newcommand{\R}{\mathcal R}
\newcommand{\cS}{\mathcal S}
\newcommand{\cU}{\mathcal U}
\newcommand{\W}{\mathcal W}
\newcommand{\Ce}{\mathsf{C}}
\newcommand{\Q}{\mathsf{Q}}
\newcommand{\grp}{\mathsf{G}}
\newcommand{\dgrp}{\mathsf{D}}
\newcommand{\bA}{\mathbb{A}}
\newcommand{\bB}{\mathbb{B}}
\newcommand{\bC}{\mathbb{C}}
\newcommand{\bD}{\mathbb{D}}
\newcommand{\bE}{\mathbb{E}}
\newcommand{\bF}{\mathbb{F}}
\newcommand{\bG}{\mathbb{G}}
\newcommand{\bK}{\mathbb{K}}
\newcommand{\bM}{\mathbb{M}}
\newcommand{\bN}{\mathbb{N}}
\newcommand{\bO}{\mathbb{O}}
\newcommand{\bP}{\mathbb{P}}
\newcommand{\bR}{\mathbb{R}}
\newcommand{\bV}{\mathbb{V}}
\newcommand{\bZ}{\mathbb{Z}}
\newcommand{\bfE}{\mathbf{E}}
\newcommand{\bfX}{\mathbf{X}}
\newcommand{\bfY}{\mathbf{Y}}
\newcommand{\bfZ}{\mathbf{Z}}
\renewcommand{\O}{\Omega}
\renewcommand{\o}{\omega}
\newcommand{\vp}{\varphi}
\newcommand{\vep}{\varepsilon}
\newcommand{\diag}{{\rm diag}}
\newcommand{\desp}{{\mathbb D^{\rm{es}}}}
\newcommand{\Geod}{{\rm Geod}}
\newcommand{\geod}{{\rm geod}}
\newcommand{\hgr}{{\mathbb H}}
\newcommand{\mgr}{{\mathbb M}}
\newcommand{\ob}{\operatorname{Ob}}
\newcommand{\obg}{{\rm Ob(\mathbb G)}}
\newcommand{\obgp}{{\rm Ob(\mathbb G')}}
\newcommand{\obh}{{\rm Ob(\mathbb H)}}
\newcommand{\Osmooth}{{\Omega^{\infty}(X,*)}}
\newcommand{\ghomotop}{{\rho_2^{\square}}}
\newcommand{\gcalp}{{\mathbb G(\mathcal P)}}
\newcommand{\rf}{{R_{\mathcal F}}}
\newcommand{\glob}{{\rm glob}}
\newcommand{\loc}{{\rm loc}}
\newcommand{\TOP}{{\rm TOP}}

\newcommand{\wti}{\widetilde}
\newcommand{\what}{\widehat}
\renewcommand{\a}{\alpha}
\newcommand{\be}{\beta}
\newcommand{\ga}{\gamma}
\newcommand{\Ga}{\Gamma}
\newcommand{\de}{\delta}
\newcommand{\del}{\partial}
\newcommand{\ka}{\kappa}
\newcommand{\si}{\sigma}
\newcommand{\ta}{\tau}
\newcommand{\lra}{{\longrightarrow}}
\newcommand{\ra}{{\rightarrow}}
\newcommand{\rat}{{\rightarrowtail}}
\newcommand{\oset}[1]{\overset {#1}{\ra}}
\newcommand{\osetl}[1]{\overset {#1}{\lra}}
\newcommand{\hr}{{\hookrightarrow}}
\newcommand{\hdgb}{\boldsymbol{\rho}^\square}
\newcommand{\hdg}{\rho^\square_2}
\renewcommand{\leq}{{\leqslant}}
\renewcommand{\geq}{{\geqslant}}
\def\red{\textcolor{red}}
\def\magenta{\textcolor{magenta}}
\def\blue{\textcolor{blue}}
\def\<{\langle}
\def\>{\rangle}

\begin{document}
The following quote indicates how \PMlinkname{fundamental groupoids}{FundamentalGroupoid} 
can be alternatively defined {\em via} the Yoneda-Grothendieck construction specified by
the \emph{fundamental groupoid functor} as in reference \cite{BR-JG2k4}. 

\subsection{Fundamental groupoid functor:}
\begin{quote}
``Therefore the \emph{fundamental groupoid}, $\Pi$ can (and should) be regarded as a {\em functor} from the category of topological spaces to the category of groupoids. This functor is not really homotopy invariant but it is 
``homotopy invariant up to homotopy" in the sense that the following holds: 
\end{quote}


\begin{theorem} 
  ``A homotopy between two continuous maps induces a natural transformation between the corresponding functors.'' 
({\em provided without proof}).
\end{theorem}

\subsection{Remarks}  

 On the other hand, the category of groupoids $G_2$, as defined previously, is in fact a $2-category$, whereas the category \textbf{$Top$}- as defined in the above quote- is not viewed as a $2-category$. An alternative approach involves the representation of the category \textbf{$Top$} {\em via} the Yoneda-Grothendieck construction as recently reported by Brown and Janelidze. This leads then to an extension of the Galois theory involving groupoids viewed as single object categories with invertible morphisms, and also to a more useful definition of the {\em fundamental groupoid functor}, as reported by Brown and Janelidze (2004); they have used the generalised Galois Theory to construct a homotopy double groupoid of a surjective fibration of Kan simplicial sets, and proceeded to utilize the latter to construct a new homotopy double groupoid of a map of spaces, which includes constructions by several other authors of a $2-groupoid$, the $cat1-group$ or crossed modules. Another advantage of such a categorical construction utilizing a double groupoid is that it provides an algebraic model of a foliated bundle (\cite{BR-JG2k4}).
A natural extension of the double groupoid is a \emph{double category} that is not restricted to the condition of all invertible morphisms of the double groupoid; (for further details see ref. \cite{BR-JG2k4}).
Note also that an alternative definition of the fundamental functor(s) was introduced by Alexander Grothendieck 
in ref. \cite{Alex1}. 

\begin{thebibliography}{9}


\bibitem{BR-JG2k4}
R. Brown and G. Janelidze.(2004). Galois theory and a new homotopy double groupoid of a map of spaces.(2004). 
{\em Applied Categorical Structures},\textbf{12}: 63-80. Pdf file in arxiv: math.AT/0208211 .

\bibitem{Alex1}
Alexander Grothendieck. 1971, Rev$\^{e}$tements $\'E$tales et Groupe Fondamental (SGA1),
chapter VI: Cat$\'e$gories fibr$\'e$es et descente, \emph{Lecture Notes in Math.},
\textbf{224}, Springer--Verlag: Berlin.


\end{thebibliography}

%%%%%
%%%%%
\end{document}
