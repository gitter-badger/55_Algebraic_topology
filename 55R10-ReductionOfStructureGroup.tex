\documentclass[12pt]{article}
\usepackage{pmmeta}
\pmcanonicalname{ReductionOfStructureGroup}
\pmcreated{2013-03-22 13:26:06}
\pmmodified{2013-03-22 13:26:06}
\pmowner{antonio}{1116}
\pmmodifier{antonio}{1116}
\pmtitle{reduction of structure group}
\pmrecord{12}{33995}
\pmprivacy{1}
\pmauthor{antonio}{1116}
\pmtype{Definition}
\pmcomment{trigger rebuild}
\pmclassification{msc}{55R10}
\pmrelated{VectorBundle}
\pmrelated{FiberBundle}
\pmdefines{Euclidean structure}
\pmdefines{Riemannian structure}
\pmdefines{complex structure}
\pmdefines{almost-complex structure}

% used for TeXing text within eps files
%\usepackage{psfrag}
% need this for including graphics (\includegraphics)
%\usepackage{graphicx}
% for neatly defining theorems and propositions
%\usepackage{amsthm}
% making logically defined graphics
%%%\usepackage{xypic}

\usepackage{theorem}
\usepackage{amsmath}
\usepackage{amsfonts}
\usepackage{amssymb}
\newcommand{\limv}[2]{\lim\limits_{#1\rightarrow #2}}
\newcommand{\eb}{\mathbf{e}} % Standard basis
\newcommand{\comp}{\circ} % Function composition
\newcommand{\reals}{{\mathbb R}} % The reals
\newcommand{\integs}{{\mathbb Z}} % The integers
\newcommand{\cpxs}{{\mathbb C}} % The "complexes" :)
\newcommand{\setc}[2]{\left\{#1:\: #2\right\}}
\newcommand{\set}[1]{{\left\{#1\right\}}}
\newcommand{\cycle}[1]{\left(#1\right)}
\newcommand{\tuple}[1]{\left(#1\right)}
\newcommand{\Partial}[2]{\frac{\partial #1}{\partial #2}}
\newcommand{\PartialSl}[2]{\partial #1/\partial #2}
\newcommand{\funcsig}[2]{#1\rightarrow #2}
\newcommand{\funcdef}[3]{#1:\funcsig{#2}{#3}}
\newcommand{\supp}{\mathop{\mathrm{Supp}}} % Support of a function
\newcommand{\sgn}{\mathop{\mathrm{sgn}}} % Sign function
\newcommand{\tr}[1]{#1^\mathrm{tr}} % Transpose of a matrix
\newcommand{\inprod}[2]{\left<#1,#2\right>} % Inner product
\newenvironment{smallbmatrix}{\left[\begin{smallmatrix}}{\end{smallmatrix}\right]}
\newcommand{\maps}[2]{\mathop{\mathrm{Maps}}\left(#1,#2\right)}
\newcommand{\intoc}[2]{\left(#1,#2\right]}
\newcommand{\intco}[2]{\left[#1,#2\right)}
\newcommand{\intoo}[2]{\left(#1,#2\right)}
\newcommand{\intcc}[2]{\left[#1,#2\right]}
\newcommand{\transv}{\pitchfork}
\newcommand{\pair}[2]{\left\langle#1,#2\right\rangle}
\newcommand{\norm}[1]{\left\|#1\right\|}
\newcommand{\sqnorm}[1]{\left\|#1\right\|^2}
\newcommand{\bdry}{\partial}
\newcommand{\inv}[1]{#1^{-1}}
\newcommand{\tensor}{\otimes}
\newcommand{\bigtensor}{\bigotimes}
\newcommand{\im}{\operatorname{im}}
\newcommand{\coker}{\operatorname{im}}
\newcommand{\map}{\operatorname{Map}}
\newcommand{\crit}{\operatorname{Crit}}
\theorembodyfont{\upshape}
\newtheorem{thm}{Theorem}
\newtheorem{dthm}[thm]{Desired Theorem}
\newtheorem{cor}[thm]{Corollary}
\newtheorem{dcor}[thm]{Desired Corollary}
\newtheorem{lem}[thm]{Lemma}
\newtheorem{prop}[thm]{Proposition}
\newtheorem{defn}{Definition}
\newtheorem{rmk}{Remark}
\newtheorem{exm}{Example}
\newcommand{\cross}{\times}
\newcommand{\del}{\nabla}
\newcommand{\homeo}{\cong}
\newcommand{\isom}{\cong}
\newcommand{\htpyeq}{\backsimeq}
\newcommand{\codim}{\operatorname{codim}}
\newcommand{\projp}{{\mathbb R}P}

% open cells (not very nice...)
\newcommand{\oce}{\smash{\overset{\circ}e}} 
\newcommand{\ocD}{\smash{\overset{\circ}D}} 

\newcommand{\susp}{\Sigma}
\newcommand{\restr}[2]{{#1}|_{#2}}

\renewcommand{\hom}{\mathop{\mathrm{Hom}}} % Homomorphisms functor
\newcommand{\rp}{\reals P} % real projective space
\newcommand{\cp}{\cpxs P} % complex projective space
\newcommand{\zmod}[1]{\integs / #1\integs} % Z/nZ
\begin{document}
\PMlinkescapeword{structure}

Given a fiber bundle $\funcdef{p}{E}{B}$ with typical fiber $F$ and structure group $G$ (henceforth called an $(F,G)$-bundle over $B$), we say that the bundle admits a {\em reduction of its structure group to $H$,} where $H<G$ is a subgroup, if it is isomorphic to an $(F,H)$-bundle over $B.$ 


Equivalently, $E$ admits a reduction of structure group to $H$ if there is
a choice of local trivializations covering $E$ such that the transition
functions all belong to $H.$

\begin{rmk}
Here, the action of $H$ on $F$ is the restriction of the $G$-action; in 
particular, this means that an $(F,H)$-bundle is automatically an 
$(F,G)$-bundle. The bundle isomorphism in the definition then becomes meaningful
in the category of $(F,G)$-bundles over $B$.
\end{rmk}

\begin{exm}
Let $H$ be the trivial subgroup. Then, the existence of a reduction of structure group to $H$ is equivalent to the bundle being trivial.
\end{exm}

For the following examples, let $E$ be an $n$-dimensional vector bundle, so that
$F\isom\reals^n$ with $G=GL(n,\reals),$ the general linear group acting as 
usual.

\begin{exm}
Set $H=GL^+(n,\reals),$ the subgroup of $GL(n,\reals)$ consisting of matrices with positive determinant. A reduction to $H$ is equivalent to an orientation of the vector bundle. In the case where $B$ is a smooth manifold and $E=TB$ is its tangent bundle, this coincides with other definitions of an orientation of $B$.
\end{exm}

\begin{exm}
Set $H=O(n)$, the orthogonal group. A reduction to $H$ is called a {\em Riemannian\/} or {\em Euclidean structure\/} on the vector bundle. It coincides with a continuous fiberwise choice
of a positive definite inner product, and for the case of the tangent bundle, 
with the usual notion of a Riemannian metric on a manifold.

When $B$ is paracompact, an argument with partitions of unity shows that
a Riemannian structure always exists on any given vector bundle. For this reason, it is often convenient to start out assuming the structure group
to be $O(n).$
\end{exm}

\begin{exm}
Let $n=2m$ be even, and let $H=GL(m,\cpxs),$ the group of invertible complex matrices, embedded in $GL(n,\reals)$ by means of the usual identification of $\cpxs$ with $\reals^2.$
A reduction to $H$ is called a {\em complex structure} on the vector bundle, and 
it is equivalent to a continuous fiberwise choice of an endomorphism $J$ satisfying $J^2=-I.$ 

A complex structure on a tangent bundle is called an {\em almost-complex structure\/} on the manifold. This is to distinguish it from the
more restrictive notion of a complex structure on a manifold, which requires the existence of an atlas with charts in $\cpxs^m$ such that the transition functions are holomorphic.
\end{exm}

\begin{exm}
Let $H=GL(1,\reals)\cross GL(n-1,\reals),$ embedded in $GL(n,\reals)$ by
$\tuple{A,B}\mapsto A\oplus B.$ A reduction to $H$ is equivalent to the 
existence of a splitting $E\isom E_1\oplus E_2,$ where $E_1$ is a line bundle.
More generally, a reduction to $GL(k,\reals)\cross GL(n-k,\reals)$ is equivalent to a splitting $E\isom E_1\oplus E_2,$ where $E_1$ is a $k$-plane bundle.
\end{exm}

\begin{rmk}
These examples all have two features in common, namely:
\begin{itemize}
\item
the subgroup $H$ can be interpreted as being precisely the subgroup of $G$ which preserves a particular structure, and,

\item
a reduction to $H$ is equivalent to a continuous fiber-by-fiber choice of a
structure of the same kind.
\end{itemize}

For example, $O(n)$ is the subgroup of $GL(n,\reals)$ which preserves the
standard inner product of $\reals^n,$ and reduction of structure to $O(n)$ is
equivalent to a fiberwise choice of inner products. 

This is not a coincidence. The intuition behind this is as follows. There
is no obstacle to choosing a fiberwise inner product in a neighborhood of any given point $x\in B$: we simply choose a neighborhood $U$ on which the bundle is trivial, and with respect to a trivialization $\inv{p}(U)\homeo\reals^n\cross U$, we can let the inner product on each $\inv{p}(y)$ be the standard inner product. However, if we make these choices locally around every point in $B$, 
there is no guarantee that they ``glue together'' properly to yield a global 
continuous choice, {\em unless} the transition functions preserve the standard 
inner product. But this is precisely what reduction of structure to $O(n)$ 
means.

The same explanation holds for subgroups preserving other kinds of structure. 

\end{rmk}

% Coming soon: relation with classifying spaces and classifying maps.
%%%%%
%%%%%
\end{document}
