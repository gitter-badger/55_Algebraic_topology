\documentclass[12pt]{article}
\usepackage{pmmeta}
\pmcanonicalname{SomeExamplesOfUniversalBundles}
\pmcreated{2013-03-22 13:12:05}
\pmmodified{2013-03-22 13:12:05}
\pmowner{bwebste}{988}
\pmmodifier{bwebste}{988}
\pmtitle{some examples of universal bundles}
\pmrecord{11}{33663}
\pmprivacy{1}
\pmauthor{bwebste}{988}
\pmtype{Example}
\pmcomment{trigger rebuild}
\pmclassification{msc}{55R15}
\pmclassification{msc}{55R10}
\pmsynonym{universal family of spaces}{SomeExamplesOfUniversalBundles}
%\pmkeywords{Universal Bundle}
%\pmkeywords{bundle of Hilbert spaces in definition of  quantum automata}
\pmrelated{CategoryOfQuantumAutomata}
\pmdefines{Hilbert bundle}

% this is the default PlanetMath preamble.  as your knowledge
% of TeX increases, you will probably want to edit this, but
% it should be fine as is for beginners.

% almost certainly you want these
\usepackage{amssymb}
\usepackage{amsmath}
\usepackage{amsfonts}

% used for TeXing text within eps files
%\usepackage{psfrag}
% need this for including graphics (\includegraphics)
%\usepackage{graphicx}
% for neatly defining theorems and propositions
%\usepackage{amsthm}
% making logically defined graphics
%%%\usepackage{xypic}

% there are many more packages, add them here as you need them

% define commands here
\begin{document}
The universal bundle  for a topological group $G$ is usually written as $\pi:EG\to BG$.  Any principal $G$-bundle for which the total space is contractible 
is universal; this will help us to find universal bundles without worrying about Milnor's construction of $EG$ involving infinite joins.

\begin{itemize}

\item $G=\mathbb{Z}_2$: $E\mathbb{Z}_2=S^\infty$ and $B\mathbb{Z}_2=\mathbb{R}P^\infty$.

\item $G=\mathbb{Z}_n$: $E\mathbb{Z}_n=S^\infty$ and $B\mathbb{Z}_n=S^\infty/\mathbb{Z}_n$. Here $\mathbb{Z}_n$ acts on $S^\infty$ (considered as a subset of a separable complex Hilbert space) via multiplication with an $n$-th root of unity.

\item $G=\mathbb{Z}^n$: $E\mathbb{Z}^n=\mathbb{R}^n$ and $B\mathbb{Z}^n=T^n$.

\item More generally, if $G$ is any discrete group then one can take  $BG$ to be any Eilenberg-Mac Lane space $ K(G,1)$ and $EG$ to be its universal cover. Indeed $EG$ is simply connected, and it follows from the lifting theorem that $\pi_n(EG)=0$ for $n\geq 0$. This example includes the previous three and many more. 

\item $G=S^1$: $ES^1=S^\infty$ and $BS^1=\mathbb{C}P^\infty$.

\item $G=SU(2)$: $ESU(2)=S^\infty$ and $BSU(2)=\mathbb{H}P^\infty$.

\item $G=O(n)$, the $n$-th orthogonal group: $EO(n)=V(\infty,n)$, the manifold of frames
of $n$ orthonormal vectors in $\mathbb{R}^\infty$, and $BO(n)=G(\infty, n)$,  the
Grassmanian of $n$-planes in $\mathbb{R}^\infty$.  The projection map is
taking the subspace spanned by a frame of vectors.

\end{itemize}
%%%%%
%%%%%
\end{document}
