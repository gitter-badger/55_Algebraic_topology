\documentclass[12pt]{article}
\usepackage{pmmeta}
\pmcanonicalname{SheafOfSections}
\pmcreated{2013-03-22 15:46:36}
\pmmodified{2013-03-22 15:46:36}
\pmowner{guffin}{12505}
\pmmodifier{guffin}{12505}
\pmtitle{sheaf of sections}
\pmrecord{7}{37734}
\pmprivacy{1}
\pmauthor{guffin}{12505}
\pmtype{Definition}
\pmcomment{trigger rebuild}
\pmclassification{msc}{55R25}
%\pmkeywords{sheaf}
%\pmkeywords{vector bundle}
%\pmkeywords{sections}
\pmrelated{VectorBundle}
\pmdefines{Sheaf of Sections}

\endmetadata

% this is the default PlanetMath preamble.  as your knowledge
% of TeX increases, you will probably want to edit this, but
% it should be fine as is for beginners.

% almost certainly you want these
\usepackage{amssymb}
\usepackage{amsmath}
\usepackage{amsfonts}

% used for TeXing text within eps files
%\usepackage{psfrag}
% need this for including graphics (\includegraphics)
%\usepackage{graphicx}
% for neatly defining theorems and propositions
%\usepackage{amsthm}
% making logically defined graphics
%%%\usepackage{xypic}

% there are many more packages, add them here as you need them

% define commands here

\newcommand{\spe}[1]{\ensuremath{ \text{Sp\'e}(#1)}}
\newcommand{\RR}[0]{\ensuremath{\mathbb R}}
\newcommand{\takes}[2]{\!:\!#1 \rightarrow #2}
\begin{document}
\subsection{Presheaf Definition}

Consider a rank $r$ vector bundle $E\rightarrow M$, whose typical fibre is defined with respect to a field $k$.  Let $\{U_\alpha\}$ constitute a cover
for $M$.  Then, sections of the bundle over some $U\subset M$ are defined
as continuous functions $U\rightarrow E$, which commute with the natural projection
map $\pi \takes E M$; $\pi \circ s = id_M$. Denote the space of sections of
the bundle over U to be $\Gamma(U,E)$.  The space of sections is a vector
space over the field $k$ by defining addition and scalar multiplication pointwise: for $s,t \in \Gamma(U,E)$, $p\in U$ and $a\in k$
\[ (s+t)(p) \equiv s(p) + t(p) \qquad \qquad (a \cdot s) (p) \equiv a \cdot s(p).\]

Then, this forms a presheaf $\mathcal E$, a functor from $((\text{top}_M))$
to the category of vector spaces, with restriction maps the natural
restriction of functions.

\subsection{Sheaf Axioms}
It is easy to see that it satisfies the sheaf
axioms: for $U$ open and $\{V_i\}$ a cover of $U$,

\begin{enumerate}
\item if $s\in \mathcal E(U)$ and $s\vert_{V_i} =0 $ for all $i$, then $s=0$.
\item if $s_i\in \mathcal E(V_i)$ for all $i$, such that for each $i,j$ with
$V_i\cap V_j \ne \emptyset$, $s_i\vert_{V_i\cap V_j} = s_j\vert_{V_i\cap
V_j}$, then there is an $s \in \mathcal E(U)$ with $s\vert_{V_i} = s_i$ for
all $i$.
\end{enumerate}

The first follows from the fact that for any $U$, there is always at least
one element of $\mathcal E(U)$, the zero section, and that the transition
functions of the bundle are linear maps. The second follows by the construction of the bundle.\\

\section{Sheafification}

We may also see the vector bundle by applying associated sheaf construction
to the presheaf $U \mapsto \Gamma(U,E)$.  First though, we show that the
stalk of the sheaf $\mathcal E$ at a point is isomorphic to the fibre of the
bundle $E$ at the point.  Let $[s,U]$ be a germ at $p\in M$ $(p\in U
\subset M)$, and define a map $\psi \takes{\mathcal E_p}{ E_p}$ by
\[\psi:[s,U] \mapsto s_p.\]

First, we show that the map is a vector space homomorphism.  Consider two
germs $[s,U]$ and $[t,V]$ in $\mathcal E_p$.  These map to $s_p$ and $t_p$
respectively.  We add the germs by finding an open set $W\in U\cap V$ and
adding the restrictions of the sections;
\[[s,U] + [t,V] \equiv [s|_W + t|_W, W].\]

Of course, $p\in W$, so we have $\psi(s|_W + t|_W) = s_p + t_p$, since the
restriction maps are simply restriction of functions.
Now, it is easy to show that $\psi$ is injective.  
Assume $\psi([t,V]) = \psi([s,U]) = s_p$.  Then 
\begin{align*}
\psi([t,V]) - \psi([s,U]) &= s_p - s_p \\
\psi([t,V] - [s,U]) &= 0\\
[t,V] &= [s,U]
\end{align*}

Now, we show that $\psi$ is surjective.  For $s_p\in E_p$, let $U\subset M$
open be isomorphic to some subset $U_\RR$ of $\RR^m$.  Then, $\Gamma(U,E)$ is the
set of continuous maps $U\rightarrow V_E$, where $V_E$ is the typical fibre of $E$;
\[\Gamma(U,E) = \bigoplus_{i=1}^r \mathcal C_{U_\RR}^\infty.\]
Then let $[s,U]$ be the constant function $s:U_\RR\mapsto s_x$, and we
have constructed an isomorphism $\psi$ between $\mathcal E_p$ and $E_p$.\\

To construct the \'Etal\'e space, take the disjoint union of stalks,
\(\text{Sp\'e}(\mathcal E) = \coprod_{p\in M} \mathcal E_p\), and endow it with
the following topology:  the open sets shall be  of the form
\[ U_s = \bigl\{s_p | s\in \Gamma(U,\mathcal E), p\in U\subset M\bigr\},\]
collection of germs of sections at points in $U\subset M$.\\

Then, the associated sheaf to $\mathcal E$ is the presheaf which assigns
continuous maps $\Gamma(U,\spe {\mathcal E})$ to each open $U$.  These are
maps where the preimage of $U_s$ is open. Clearly, this implies that
$\Gamma(U,E) \subset \Gamma(U,\spe{\mathcal E})$.  To go the other way, note
that open sets of $\spe{\mathcal E}$ are the images of continuous maps $U\rightarrow
E$.  An open subset of $\spe{\mathcal E}$ may be written as a union of $U_t$;
$U_{ts} \equiv \{t_p, s_p | p\in U\}$.  Then, by single-valuedness of maps,
a continuous map $U\rightarrow \spe{\mathcal E}$ must map to $U_t$ for some $t\in
\Gamma(U,E)$, so we have $\Gamma(U,E) \supset \Gamma(U,\spe{\mathcal E})$.
%%%%%
%%%%%
\end{document}
