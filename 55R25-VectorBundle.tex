\documentclass[12pt]{article}
\usepackage{pmmeta}
\pmcanonicalname{VectorBundle}
\pmcreated{2013-03-22 13:07:15}
\pmmodified{2013-03-22 13:07:15}
\pmowner{rspuzio}{6075}
\pmmodifier{rspuzio}{6075}
\pmtitle{vector bundle}
\pmrecord{7}{33554}
\pmprivacy{1}
\pmauthor{rspuzio}{6075}
\pmtype{Definition}
\pmcomment{trigger rebuild}
\pmclassification{msc}{55R25}
\pmrelated{ReductionOfStructureGroup}
\pmrelated{SheafOfSections2}
\pmrelated{FrameGroupoid}
\pmdefines{section}
\pmdefines{trivial vector bundle}
\pmdefines{sheaf of sections}

\endmetadata

% this is the default PlanetMath preamble.  as your knowledge
% of TeX increases, you will probably want to edit this, but
% it should be fine as is for beginners.

% almost certainly you want these
\usepackage{amssymb}
\usepackage{amsmath}
\usepackage{amsfonts}

% used for TeXing text within eps files
%\usepackage{psfrag}
% need this for including graphics (\includegraphics)
%\usepackage{graphicx}
% for neatly defining theorems and propositions
%\usepackage{amsthm}
% making logically defined graphics
%%%\usepackage{xypic} 

% there are many more packages, add them here as you need them

% define commands here
\begin{document}
\section*{Basic definition}

A \emph{vector bundle} is a fiber bundle having a vector space as a fiber and the general linear group of that vector space (or some subgroup) as structure group. Common examples of a vector bundle include the tangent bundle of a differentiable manifold and the M\"obius strip (of infinite width). 

\section*{Vector bundles in various categories}

As with fiber bundles, the idea of a vector bundle exists in many categories.  We talk about topological vector bundles (in the category of topological spaces), we talk about differentiable vector bundles, we talk about complex analytic (or holomorphic) vector bundles, and we talk about algebraic vector bundles.  In each case, the fiber must have a structure from the appropriate category, and the general linear group must also be equipped with a structure from the appropriate category (generally this means it must be a group object and it must act through morphisms in the category). 

Specifically, if we want a topological vector bundle, we must supply a topological space for the base space, a topological space for the whole space, and the projection map must be continuous; this specifies a topology on each fiber.  The general linear group must also act continuously.

If we are in the category of schemes, each local trivialization must be an affine space over the affine ring of the neighborhood on the scheme, and the general linear group scheme must act on it through morphisms of schemes. 

\section*{Sections of a vector bundle}

As with any fiber bundle, a vector bundle may have sections.  If a vector bundle on $X$ is defined on an open cover $\{U_\alpha\}$ with transition functions $\phi_{\alpha\beta}$ taken from $\mathop{GL}_n$, a section is a collection of functions $f_\alpha:U_\alpha\to U_\alpha\times V$ which give the identity when projected down to $U_\alpha$ and such that 
\[
\phi_{\alpha\beta}\circ f_\beta|_{U_\alpha \cap U_\beta} = f_\alpha|_{U_\alpha \cap U_\beta}.
\]

Sections may be added and scaled by field elements by simply applying these operations to each fiber, so they form a vector space.  A very common application of the Riemann-Roch theorem is to count the number of linearly independent sections on a curve, surface, or higher-dimensional variety.  

One is often interested in families of sections that are linearly independent in each fiber.  If the vector bundle has dimension $n$ and there are $n$ sections that are linearly independent on every fiber, then the vector bundle is isomorphic to the Cartesian product of $X\times V$, which is called the \emph{trivial vector bundle}.  Such a family of sections is therefore called a \emph{trivialization}.

One is sometimes interested in sections of a related vector bundle obtained by restricting the base space to some open subset.  In this way, one can obtain a sheaf from a vector bundle, called the \emph{sheaf of sections}.

\section*{Operations on vector bundles}

Since the fiber of a vector bundle is a vector space, one can do many operations on vector bundles over a fixed space $X$; in fact, almost all the usual operations on vector spaces can be applied.  However, they are often not quite as simple as in the case of finite-dimensional vector spaces.

One can take direct sums and tensor products of vector bundles; the dimensions (if finite) behave as expected.  Morphisms between vector bundles over $X$ are just linear maps on the fibers, with appropriate continuity conditions: since the space of linear maps between two vector spaces is again a vector space, a morphism between vector bundles must be a vector bundle itself.

If one has a short exact sequence of vector bundles over $X$,
\[
0 \to T \to U \to V \to 0,
\]
then the dimension of $U$ is the sum of the dimensions of $T$ and $V$, as one might expect; but one often cannot write $U$ as the direct sum of $T$ and $V$.  In this way, vector bundles resemble modules over a ring or abelian groups; in fact it is the behaviour of finite-dimensional vector spaces that is ``too good to be true''. 

\section*{Relation to other objects}

In the algebraic category, that is, vector bundles over schemes, there is a very nice correspondence between vector bundles and locally free sheaves; when the dimension is one and the scheme is nice enough, there is a further correspondence with Cartier divisors.
%%%%%
%%%%%
\end{document}
