\documentclass[12pt]{article}
\usepackage{pmmeta}
\pmcanonicalname{Omegaspectrum}
\pmcreated{2013-03-22 18:24:01}
\pmmodified{2013-03-22 18:24:01}
\pmowner{bci1}{20947}
\pmmodifier{bci1}{20947}
\pmtitle{Omega-spectrum}
\pmrecord{80}{41046}
\pmprivacy{1}
\pmauthor{bci1}{20947}
\pmtype{Topic}
\pmcomment{trigger rebuild}
\pmclassification{msc}{55T20}
\pmclassification{msc}{55T25}
\pmclassification{msc}{55T05}
\pmsynonym{Omega spectrum}{Omegaspectrum}
\pmsynonym{$\Omega $ spectrum}{Omegaspectrum}
%\pmkeywords{omega-spectrum}
%\pmkeywords{sequence of based spaces}
%\pmkeywords{weak homotopy equivalences}
\pmrelated{PointedTopologicalSpace}
\pmrelated{CategoricalSequence}
\pmrelated{ClassesOfAlgebras}
\pmrelated{HomotopyCategory}
\pmrelated{WeakHomotopyEquivalence}
\pmrelated{WeakHomotopyDoubleGroupoid}
\pmrelated{CohomologyGroupTheorem}
\pmrelated{GroupCohomology}
\pmrelated{ProofOfCohomologyGroupTheorem}
\pmdefines{equence of CW complexes}
\pmdefines{spectrum}
\pmdefines{$\Omega$--spectrum}
\pmdefines{Omega spectrum}
\pmdefines{unit circle}
\pmdefines{cohomology group}
\pmdefines{category of spectra}

\endmetadata

% this is the default PlanetMath preamble.  as your knowledge
% of TeX increases, you will probably want to edit this, but
% it should be fine as is for beginners.

% almost certainly you want these
\usepackage{amssymb}
\usepackage{amsmath}
\usepackage{amsfonts}

% used for TeXing text within eps files
%\usepackage{psfrag}
% need this for including graphics (\includegraphics)
%\usepackage{graphicx}
% for neatly defining theorems and propositions
%\usepackage{amsthm}
% making logically defined graphics
%%%\usepackage{xypic}

% there are many more packages, add them here as you need them

% define commands here
\usepackage{amsmath, amssymb, amsfonts, amsthm, amscd, latexsym}
%%\usepackage{xypic}
\usepackage[mathscr]{eucal}

\setlength{\textwidth}{6.5in}
%\setlength{\textwidth}{16cm}
\setlength{\textheight}{9.0in}
%\setlength{\textheight}{24cm}

\hoffset=-.75in     %%ps format
%\hoffset=-1.0in     %%hp format
\voffset=-.4in

\theoremstyle{plain}
\newtheorem{lemma}{Lemma}[section]
\newtheorem{proposition}{Proposition}[section]
\newtheorem{theorem}{Theorem}[section]
\newtheorem{corollary}{Corollary}[section]

\theoremstyle{definition}
\newtheorem{definition}{Definition}[section]
\newtheorem{example}{Example}[section]
%\theoremstyle{remark}
\newtheorem{remark}{Remark}[section]
\newtheorem*{notation}{Notation}
\newtheorem*{claim}{Claim}

\renewcommand{\thefootnote}{\ensuremath{\fnsymbol{footnote%%@
}}}
\numberwithin{equation}{section}

\newcommand{\Ad}{{\rm Ad}}
\newcommand{\Aut}{{\rm Aut}}
\newcommand{\Cl}{{\rm Cl}}
\newcommand{\Co}{{\rm Co}}
\newcommand{\DES}{{\rm DES}}
\newcommand{\Diff}{{\rm Diff}}
\newcommand{\Dom}{{\rm Dom}}
\newcommand{\Hol}{{\rm Hol}}
\newcommand{\Mon}{{\rm Mon}}
\newcommand{\Hom}{{\rm Hom}}
\newcommand{\Ker}{{\rm Ker}}
\newcommand{\Ind}{{\rm Ind}}
\newcommand{\IM}{{\rm Im}}
\newcommand{\Is}{{\rm Is}}
\newcommand{\ID}{{\rm id}}
\newcommand{\GL}{{\rm GL}}
\newcommand{\Iso}{{\rm Iso}}
\newcommand{\Sem}{{\rm Sem}}
\newcommand{\St}{{\rm St}}
\newcommand{\Sym}{{\rm Sym}}
\newcommand{\SU}{{\rm SU}}
\newcommand{\Tor}{{\rm Tor}}
\newcommand{\U}{{\rm U}}

\newcommand{\A}{\mathcal A}
\newcommand{\Ce}{\mathcal C}
\newcommand{\D}{\mathcal D}
\newcommand{\E}{\mathcal E}
\newcommand{\F}{\mathcal F}
\newcommand{\G}{\mathcal G}
\newcommand{\Q}{\mathcal Q}
\newcommand{\R}{\mathcal R}
\newcommand{\cS}{\mathcal S}
\newcommand{\cU}{\mathcal U}
\newcommand{\W}{\mathcal W}

\newcommand{\bA}{\mathbb{A}}
\newcommand{\bB}{\mathbb{B}}
\newcommand{\bC}{\mathbb{C}}
\newcommand{\bD}{\mathbb{D}}
\newcommand{\bE}{\mathbb{E}}
\newcommand{\bF}{\mathbb{F}}
\newcommand{\bG}{\mathbb{G}}
\newcommand{\bK}{\mathbb{K}}
\newcommand{\bM}{\mathbb{M}}
\newcommand{\bN}{\mathbb{N}}
\newcommand{\bO}{\mathbb{O}}
\newcommand{\bP}{\mathbb{P}}
\newcommand{\bR}{\mathbb{R}}
\newcommand{\bV}{\mathbb{V}}
\newcommand{\bZ}{\mathbb{Z}}

\newcommand{\bfE}{\mathbf{E}}
\newcommand{\bfX}{\mathbf{X}}
\newcommand{\bfY}{\mathbf{Y}}
\newcommand{\bfZ}{\mathbf{Z}}

\renewcommand{\O}{\Omega}
\renewcommand{\o}{\omega}
\newcommand{\vp}{\varphi}
\newcommand{\vep}{\varepsilon}

\newcommand{\diag}{{\rm diag}}
\newcommand{\grp}{{\mathbb G}}
\newcommand{\dgrp}{{\mathbb D}}
\newcommand{\desp}{{\mathbb D^{\rm{es}}}}
\newcommand{\Geod}{{\rm Geod}}
\newcommand{\geod}{{\rm geod}}
\newcommand{\hgr}{{\mathbb H}}
\newcommand{\mgr}{{\mathbb M}}
\newcommand{\ob}{{\rm Ob}}
\newcommand{\obg}{{\rm Ob(\mathbb G)}}
\newcommand{\obgp}{{\rm Ob(\mathbb G')}}
\newcommand{\obh}{{\rm Ob(\mathbb H)}}
\newcommand{\Osmooth}{{\Omega^{\infty}(X,*)}}
\newcommand{\ghomotop}{{\rho_2^{\square}}}
\newcommand{\gcalp}{{\mathbb G(\mathcal P)}}

\newcommand{\rf}{{R_{\mathcal F}}}
\newcommand{\glob}{{\rm glob}}
\newcommand{\loc}{{\rm loc}}
\newcommand{\TOP}{{\rm TOP}}

\newcommand{\wti}{\widetilde}
\newcommand{\what}{\widehat}

\renewcommand{\a}{\alpha}
\newcommand{\be}{\beta}
\newcommand{\ga}{\gamma}
\newcommand{\Ga}{\Gamma}
\newcommand{\de}{\delta}
\newcommand{\del}{\partial}
\newcommand{\ka}{\kappa}
\newcommand{\si}{\sigma}
\newcommand{\ta}{\tau}
\newcommand{\med}{\medbreak}
\newcommand{\medn}{\medbreak \noindent}
\newcommand{\bign}{\bigbreak \noindent}
\newcommand{\lra}{{\longrightarrow}}
\newcommand{\ra}{{\rightarrow}}
\newcommand{\rat}{{\rightarrowtail}}
\newcommand{\oset}[1]{\overset {#1}{\ra}}
\newcommand{\osetl}[1]{\overset {#1}{\lra}}
\newcommand{\hr}{{\hookrightarrow}}

\begin{document}
This is a topic entry on $\Omega$--spectra and their important role in reduced cohomology theories
on CW complexes.

\subsection{Introduction}
In algebraic topology a \emph{spectrum} ${\bf S}$ is defined as a 
sequence of topological spaces $[X_0;X_1;... X_i;X_{i+1};... ]$ together with 
\PMlinkname{structure mappings}{ClassesOfAlgebras} $S1 \bigwedge X_i \to X_{i+1}$, where $S1$ is the \emph{unit circle} (that is, a circle with a unit radius). 

\subsection{$\Omega$--spectrum}
 One can express the definition of an $\Omega$--spectrum in terms of a sequence of CW
complexes, $K_1,K_2,...$ as follows.

\begin{definition}
Let us consider $\Omega K$, the space of loops in a $CW$ complex $K$ called
the {\em loopspace of $K$}, which is topologized as a subspace of the space $K^I$
of all maps $I \to K$ , where $K^I$ is given the compact-open topology.
Then, an \emph{$\Omega$--spectrum} $\left\{ K_n\right\}$ is defined as a
sequence $K_1,K_2,...$ of CW complexes together with \PMlinkname{weak homotopy equivalences}{WeakHomotopyEquivalence} ($\epsilon_n$):

$$\epsilon_n: \Omega K_n \to K_{n + 1},$$ with $n$ being an integer. 
\end{definition}
 
An alternative definition of the $\Omega$--spectrum can also be formulated as follows.

\begin{definition}
An \emph{$\Omega$--spectrum}, or \emph{Omega spectrum}, is a spectrum ${\bf E}$ such that for every index $i$, 
the topological space $X_i$ is fibered, and also the adjoints of the \PMlinkname{structure mappings}{ClassesOfAlgebras} are all weak equivalences $X_i \cong \Omega X_{i+1}$.
\end{definition}

\subsection{The Role of $\Omega$-spectra in Reduced Cohomology Theories}


 A category of spectra (regarded as above as sequences) will provide a model category that enables one to construct a stable homotopy theory, so that the homotopy category of spectra is canonically defined in the classical manner. Therefore, for any given construction of an $\Omega$--spectrum one is able to canonically define an associated cohomology theory; thus, one defines the \PMlinkname{cohomology groups}{ProofOfCohomologyGroupTheorem} of a CW-complex $K$ associated with the $\Omega$--spectrum ${\bf E}$ by setting the rule: 
$H^n(K;{\bf E}) = [K, E_n].$ 

 The latter set when $K$ is a CW complex can be endowed with a group structure by requiring that 
$(\epsilon_n)* : [K, E_n] \to [K, \Omega E_{n+1}]$ is an isomorphism which defines the multiplication
in $[K, E_n]$ induced by $\epsilon_n$.

 One can prove that if $\left\{ K_n\right\}$ is a an $\Omega$-spectrum then the functors 
defined by the assignments $X \longmapsto h^n(X) = (X,K_n),$
with $n \in \mathbb{Z}$ define a reduced cohomology theory on the category of basepointed CW complexes and basepoint preserving maps; furthermore, every reduced cohomology theory on CW complexes arises in this manner from an $\Omega$-spectrum (the Brown representability theorem; p. 397 of \cite{AllenHatcher2k1}). 


\begin{thebibliography}{9}

\bibitem{SM}
H. Masana. 2008. {\em The Tate-Thomason Conjecture}. 
\PMlinkexternal{Section 1.0.4.}{http://www.math.uiuc.edu/K-theory/0919/TT.pdf} , on p.4.

\bibitem{MFA67}
M. F. Atiyah, ``{\em K-theory: lectures.}'', Benjamin (1967).

\bibitem{HB68}
H. Bass,``{\em Algebraic K-theory.}'' , Benjamin (1968) 

\bibitem{RGS68}
R. G. Swan, ``{\em Algebraic K-theory.}'' , Springer (1968) 

\bibitem{CBT69}
C. B. Thomas (ed.) and R.M.F. Moss (ed.) , ``{\em Algebraic K-theory and its geometric applications.}'', Springer  (1969) 

\bibitem{AllenHatcher2k1}
Hatcher, A. 2001. \PMlinkexternal{Algebraic Topology.}{http://www.math.cornell.edu/~hatcher/AT/AT.pdf}, Cambridge University Press; Cambridge, UK.

\end{thebibliography}

%%%%%
%%%%%
\end{document}
