\documentclass[12pt]{article}
\usepackage{pmmeta}
\pmcanonicalname{NaturalTransformationsOfOrganismicStructures}
\pmcreated{2013-03-22 18:13:07}
\pmmodified{2013-03-22 18:13:07}
\pmowner{bci1}{20947}
\pmmodifier{bci1}{20947}
\pmtitle{natural transformations of organismic structures}
\pmrecord{31}{40802}
\pmprivacy{1}
\pmauthor{bci1}{20947}
\pmtype{Topic}
\pmcomment{trigger rebuild}
\pmclassification{msc}{55T99}
\pmclassification{msc}{18C99}
\pmclassification{msc}{18A25}
\pmclassification{msc}{18A30}
\pmclassification{msc}{18A40}
\pmsynonym{dynamics in organismic supercategories}{NaturalTransformationsOfOrganismicStructures}
\pmsynonym{organismic supercategory biodynamics}{NaturalTransformationsOfOrganismicStructures}
%\pmkeywords{Biodynamics in Organismic Supercategories}
%\pmkeywords{Organismic Supercategory Dynamics}
\pmrelated{Supercategory}
\pmrelated{ETAC}
\pmrelated{ETAS}
\pmrelated{Supercategories3}
\pmrelated{SupercategoriesOfComplexSystems}
\pmrelated{ComplexSystemsBiology}
\pmdefines{natural transformations of heterofunctors in supercategories}

\endmetadata

% this is the default PlanetMath preamble.  as your knowledge
% of TeX increases, you will probably want to edit this, but
% it should be fine as is for beginners.

% almost certainly you want these
\usepackage{amssymb}
\usepackage{amsmath}
\usepackage{amsfonts}

% used for TeXing text within eps files
%\usepackage{psfrag}
% need this for including graphics (\includegraphics)
%\usepackage{graphicx}
% for neatly defining theorems and propositions
%\usepackage{amsthm}
% making logically defined graphics
%%%\usepackage{xypic}

% there are many more packages, add them here as you need them

% define commands here
\usepackage{amsmath, amssymb, amsfonts, amsthm, amscd,  enumerate}
\usepackage{xypic, xspace}
\usepackage[mathscr]{eucal}
\usepackage[dvips]{graphicx}
\usepackage[curve]{xy}
\setlength{\textwidth}{6.5in}
\setlength{\textheight}{9.0in}
\voffset=-.4in
\theoremstyle{plain}
\newtheorem{lemma}{Lemma}[section]
\newtheorem{proposition}{Proposition}[section]
\newtheorem{theorem}{Theorem}[section]
\newtheorem{corollary}{Corollary}[section]
\theoremstyle{definition}
\newtheorem{definition}{Definition}[section]
\newtheorem{example}{Example}[section]
\newtheorem{remark}{Remark}[section]
\newtheorem*{notation}{Notation}
\newtheorem*{claim}{Claim}
\renewcommand{\thefootnote}{\ensuremath{\fnsymbol{footnote}}}
\numberwithin{equation}{section}
\newcommand{\Ad}{{\rm Ad}}
\newcommand{\Aut}{{\rm Aut}}
\newcommand{\Cl}{{\rm Cl}}
\newcommand{\Co}{{\rm Co}}
\newcommand{\DES}{{\rm DES}}
\newcommand{\Diff}{{\rm Diff}}
\newcommand{\Dom}{{\rm Dom}}
\newcommand{\Hol}{{\rm Hol}}
\newcommand{\Mon}{{\rm Mon}}
\newcommand{\Hom}{{\rm Hom}}
\newcommand{\Ker}{{\rm Ker}}
\newcommand{\Ind}{{\rm Ind}}
\newcommand{\IM}{{\rm Im}}
\newcommand{\Is}{{\rm Is}}
\newcommand{\ID}{{\rm id}}
\newcommand{\grpL}{{\rm GL}}
\newcommand{\Iso}{{\rm Iso}}
\newcommand{\rO}{{\rm O}}
\newcommand{\Sem}{{\rm Sem}}
\newcommand{\SL}{{\rm Sl}}
\newcommand{\St}{{\rm St}}
\newcommand{\Sym}{{\rm Sym}}
\newcommand{\Symb}{{\rm Symb}}
\newcommand{\SU}{{\rm SU}}
\newcommand{\Tor}{{\rm Tor}}
\newcommand{\U}{{\rm U}}
\newcommand{\A}{\mathcal A}
\newcommand{\Ce}{\mathcal C}
\newcommand{\E}{\mathcal E}
\newcommand{\F}{\mathcal F}
%\newcommand{\grp}{\mathcal G}
\renewcommand{\H}{\mathcal H}
\renewcommand{\cL}{\mathcal L}
\newcommand{\Q}{\mathcal Q}
\newcommand{\R}{\mathcal R}
\newcommand{\cS}{\mathcal S}
\newcommand{\cU}{\mathcal U}
\newcommand{\W}{\mathcal W}
\newcommand{\bA}{\mathbb{A}}
\newcommand{\bB}{\mathbb{B}}
\newcommand{\bC}{\mathbb{C}}
\newcommand{\bD}{\mathbb{D}}
\newcommand{\bE}{\mathbb{E}}
\newcommand{\bF}{\mathbb{F}}
\newcommand{\bG}{\mathbb{G}}
\newcommand{\bK}{\mathbb{K}}
\newcommand{\bM}{\mathbb{M}}
\newcommand{\bN}{\mathbb{N}}
\newcommand{\bO}{\mathbb{O}}
\newcommand{\bP}{\mathbb{P}}
\newcommand{\bR}{\mathbb{R}}
\newcommand{\bV}{\mathbb{V}}
\newcommand{\bZ}{\mathbb{Z}}
\newcommand{\bfE}{\mathbf{E}}
\newcommand{\bfX}{\mathbf{X}}
\newcommand{\bfY}{\mathbf{Y}}
\newcommand{\bfZ}{\mathbf{Z}}
\renewcommand{\O}{\Omega}
\renewcommand{\o}{\omega}
\newcommand{\vp}{\varphi}
\newcommand{\vep}{\varepsilon}
\newcommand{\diag}{{\rm diag}}
\newcommand{\grp}{\mathcal G}
\newcommand{\dgrp}{{\mathsf{D}}}
\newcommand{\desp}{{\mathsf{D}^{\rm{es}}}}
\newcommand{\hgr}{{\mathsf{H}}}
\newcommand{\mgr}{{\mathsf{M}}}
\newcommand{\ob}{{\rm Ob}}
\newcommand{\obg}{{\rm Ob(\mathsf{G)}}}
\newcommand{\obgp}{{\rm Ob(\mathsf{G}')}}
\newcommand{\obh}{{\rm Ob(\mathsf{H})}}
\newcommand{\Osmooth}{{\Omega^{\infty}(X,*)}}
\newcommand{\grphomotop}{{\rho_2^{\square}}}
\newcommand{\grpcalp}{{\mathsf{G}(\mathcal P)}}
\newcommand{\rf}{{R_{\mathcal F}}}
\newcommand{\grplob}{{\rm glob}}
\newcommand{\loc}{{\rm loc}}
\newcommand{\TOP}{{\rm TOP}}
\newcommand{\wti}{\widetilde}
\newcommand{\what}{\widehat}
\renewcommand{\a}{\alpha}
\newcommand{\be}{\beta}
\newcommand{\de}{\delta}
\newcommand{\del}{\partial}
\newcommand{\ka}{\kappa}
\newcommand{\si}{\sigma}
\newcommand{\ta}{\tau}
\newcommand{\med}{\medbreak}
\newcommand{\medn}{\medbreak \noindent}
\newcommand{\bign}{\bigbreak \noindent}
\newcommand{\lra}{{\longrightarrow}}
\newcommand{\ra}{{\rightarrow}}
\newcommand{\rat}{{\rightarrowtail}}
\newcommand{\ovset}[1]{\overset {#1}{\ra}}
\newcommand{\ovsetl}[1]{\overset {#1}{\lra}}
\newcommand{\hr}{{\hookrightarrow}}
\begin{document}
\subsection{Natural Transformations of Organismic Structures} 

Biological systems, or living organisms are characterized by relational structures and their dynamic transformations which can be represented  as natural transformations of \PMlinkname{heterofunctors in organismic supercategories}{Supercategories3}(OS). Such OS-structures can be specified mathematically either by using the \PMlinkname{Yoneda-Grothendieck Lemma and construction}{YonedaEmbedding}, or they can be directly derived by a mathematical interpretation of the first ten axioms of ETAS, plus two additional axioms defining both `self-repair' of metabolic components and complete reproduction in terms of \PMlinkexternal{genetic coding, translational genomics}{http://planetphysics.org/?op=getobj&from=books&id=213} and epigenetic meta-processes. Further details concerning mathematical, logical and complex modeling are provided in the following list of publications and related web (html) links.


\begin{thebibliography}{9}

\bibitem{ICB6}
I.C. Baianu: 1977, A Logical Model of Genetic Activities in \L ukasiewicz Algebras: The Non-linear Theory. \emph{Bulletin of Mathematical Biophysics}, \textbf{39}: 249-258.

\bibitem{ICB7}
I.C. Baianu: 1980, Natural Transformations of Organismic Structures. \emph{Bulletin of Mathematical Biophysics}
\textbf{42}: 431-446.

\bibitem{ICB8}
I.C. Baianu: 1983, Natural Transformation Models in Molecular Biology., in \emph{Proceedings of the SIAM Natl. Meet}., Denver, CO.; \PMlinkexternal{An Eprint is here available}{http://cogprints.org/3675/1/Naturaltransfmolbionu6.pdf} .

\bibitem{ICB9}
I.C. Baianu: 1984, A Molecular-Set-Variable Model of Structural and Regulatory Activities in Metabolic and Genetic Networks., \emph{FASEB Proceedings} \textbf{43}, 917.

\bibitem{ICB2}
I.C. Baianu: 1987a, Computer Models and Automata Theory in Biology and Medicine.,  in M. Witten (ed.), 
\emph{Mathematical Models in Medicine}, vol. 7., Pergamon Press, New York, 1513-1577; \PMlinkexternal{CERN Preprint No. EXT-2004-072:}{http://documents.cern.ch/cgi-bin/setlink?base=preprint&categ=ext&id=ext-2004-067}.

\bibitem{ICB9b}
I.C. Baianu: 1987b, Molecular Models of Genetic and Organismic Structures, in \emph{Proceed. Relational Biology Symp.} Argentina; \PMlinkexternal{CERN Preprint No.EXT-2004-067:MolecularModelsICB3.doc}{http://documents.cern.ch/cgi-bin/setlink?base=preprint&categ=ext&id=ext-2004-067}.

\bibitem{Bgg2}
I.C. Baianu, Glazebrook, J. F. and G. Georgescu: 2004, Categories of Quantum Automata and 
N-Valued \L ukasiewicz Algebras in Relation to Dynamic Bionetworks, \textbf{(M,R)}--Systems and
Their Higher Dimensional Algebra, 
\PMlinkexternal{Abstract of Report is here available as a PDF}{http://www.ag.uiuc.edu/fs401/QAuto.pdf} and 
\PMlinkexternal{html document}{http://doc.cern.ch/archive/electronic/other/ext/ext-2004-058/QuantumAutnu3_ICB.pdf}


\bibitem{BHS2}
R. Brown R, P.J. Higgins, and R. Sivera.: \emph{``Non--Abelian Algebraic Topology''},({\em in preparation}).
\PMlinkexternal{available here as PDF}{http://arxiv.org/PS_cache/math/pdf/0407/0407275v2.pdf}.  

\bibitem{BGB2}
R. Brown, J. F. Glazebrook and I. C. Baianu: A categorical and higher dimensional algebra framework for complex systems and spacetime structures, \emph{Axiomathes} \textbf{17}:409-493.
(2007).


\bibitem{LO68}
L. L$\ddot{o}$fgren: 1968. On Axiomatic Explanation of Complete Self--Reproduction. \emph{Bull. Math. Biophysics}, 
\textbf{30}: 317-348.

\bibitem{ICBetal2k9}
Contributed Review. 2009. GNUL download.
\PMlinkexternal{``DNA Molecular Models and Dynamics.''}{http://planetphysics.org/?op=getobj&from=books&id=213} 


\end{thebibliography}
%%%%%
%%%%%
\end{document}
