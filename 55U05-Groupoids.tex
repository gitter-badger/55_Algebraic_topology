\documentclass[12pt]{article}
\usepackage{pmmeta}
\pmcanonicalname{Groupoids}
\pmcreated{2013-03-22 18:15:32}
\pmmodified{2013-03-22 18:15:32}
\pmowner{bci1}{20947}
\pmmodifier{bci1}{20947}
\pmtitle{groupoids}
\pmrecord{43}{40856}
\pmprivacy{1}
\pmauthor{bci1}{20947}
\pmtype{Topic}
\pmcomment{trigger rebuild}
\pmclassification{msc}{55U05}
\pmclassification{msc}{55U35}
\pmclassification{msc}{55U40}
\pmclassification{msc}{18G55}
\pmclassification{msc}{18B40}
\pmsynonym{groupoid categories}{Groupoids}
\pmsynonym{topological groupoids}{Groupoids}
\pmsynonym{supergroups}{Groupoids}
%\pmkeywords{groupoid categories}
%\pmkeywords{groupoid categories}
%\pmkeywords{topological groupoid}
%\pmkeywords{Lie groupoid}
%\pmkeywords{examples of groupoids}
%\pmkeywords{higher homotopy groupoids}
%\pmkeywords{HDA}
%\pmkeywords{higher dimensional algebraic topology (HDAT)}
%\pmkeywords{groupoid and group representations related to quantum symmetries}
\pmrelated{Groupoid}
\pmrelated{GroupoidCategory}
\pmrelated{GroupoidHomomorphisms}
\pmrelated{HomotopyDoubleGroupoidOfAHausdorffSpace}
\pmrelated{TopologicalGroupoid}
\pmrelated{QuantumGroups}
\pmrelated{GeneralizedVanKampenTheoremsHigherDimensional}
\pmrelated{EquivalentRepresentationsOfGroupoids}
\pmrelated{C_cG}
\pmrelated{GroupoidAndGroupRepresentationsRelate}

% this is the default PlanetMath preamble.  as your knowledge
% of TeX increases, you will probably want to edit this, but
% it should be fine as is for beginners.

% almost certainly you want these
\usepackage{amssymb}
\usepackage{amsmath}
\usepackage{amsfonts}

% used for TeXing text within eps files
%\usepackage{psfrag}
% need this for including graphics (\includegraphics)
%\usepackage{graphicx}
% for neatly defining theorems and propositions
%\usepackage{amsthm}
% making logically defined graphics
%%%\usepackage{xypic}

% there are many more packages, add them here as you need them

% define commands here
\usepackage{amsmath, amssymb, amsfonts, amsthm, amscd, latexsym}
%%\usepackage{xypic}
\usepackage[mathscr]{eucal}

\setlength{\textwidth}{6.5in}
%\setlength{\textwidth}{16cm}
\setlength{\textheight}{9.0in}
%\setlength{\textheight}{24cm}

\hoffset=-.75in     %%ps format
%\hoffset=-1.0in     %%hp format
\voffset=-.4in

\theoremstyle{plain}
\newtheorem{lemma}{Lemma}[section]
\newtheorem{proposition}{Proposition}[section]
\newtheorem{theorem}{Theorem}[section]
\newtheorem{corollary}{Corollary}[section]

\theoremstyle{definition}
\newtheorem{definition}{Definition}[section]
\newtheorem{example}{Example}[section]
%\theoremstyle{remark}
\newtheorem{remark}{Remark}[section]
\newtheorem*{notation}{Notation}
\newtheorem*{claim}{Claim}

\renewcommand{\thefootnote}{\ensuremath{\fnsymbol{footnote%%@
}}}
\numberwithin{equation}{section}

\newcommand{\Ad}{{\rm Ad}}
\newcommand{\Aut}{{\rm Aut}}
\newcommand{\Cl}{{\rm Cl}}
\newcommand{\Co}{{\rm Co}}
\newcommand{\DES}{{\rm DES}}
\newcommand{\Diff}{{\rm Diff}}
\newcommand{\Dom}{{\rm Dom}}
\newcommand{\Hol}{{\rm Hol}}
\newcommand{\Mon}{{\rm Mon}}
\newcommand{\Hom}{{\rm Hom}}
\newcommand{\Ker}{{\rm Ker}}
\newcommand{\Ind}{{\rm Ind}}
\newcommand{\IM}{{\rm Im}}
\newcommand{\Is}{{\rm Is}}
\newcommand{\ID}{{\rm id}}
\newcommand{\GL}{{\rm GL}}
\newcommand{\Iso}{{\rm Iso}}
\newcommand{\Sem}{{\rm Sem}}
\newcommand{\St}{{\rm St}}
\newcommand{\Sym}{{\rm Sym}}
\newcommand{\SU}{{\rm SU}}
\newcommand{\Tor}{{\rm Tor}}
\newcommand{\U}{{\rm U}}

\newcommand{\A}{\mathcal A}
\newcommand{\Ce}{\mathcal C}
\newcommand{\D}{\mathcal D}
\newcommand{\E}{\mathcal E}
\newcommand{\F}{\mathcal F}
\newcommand{\G}{\mathcal G}
\newcommand{\Q}{\mathcal Q}
\newcommand{\R}{\mathcal R}
\newcommand{\cS}{\mathcal S}
\newcommand{\cU}{\mathcal U}
\newcommand{\W}{\mathcal W}

\newcommand{\bA}{\mathbb{A}}
\newcommand{\bB}{\mathbb{B}}
\newcommand{\bC}{\mathbb{C}}
\newcommand{\bD}{\mathbb{D}}
\newcommand{\bE}{\mathbb{E}}
\newcommand{\bF}{\mathbb{F}}
\newcommand{\bG}{\mathbb{G}}
\newcommand{\bK}{\mathbb{K}}
\newcommand{\bM}{\mathbb{M}}
\newcommand{\bN}{\mathbb{N}}
\newcommand{\bO}{\mathbb{O}}
\newcommand{\bP}{\mathbb{P}}
\newcommand{\bR}{\mathbb{R}}
\newcommand{\bV}{\mathbb{V}}
\newcommand{\bZ}{\mathbb{Z}}

\newcommand{\bfE}{\mathbf{E}}
\newcommand{\bfX}{\mathbf{X}}
\newcommand{\bfY}{\mathbf{Y}}
\newcommand{\bfZ}{\mathbf{Z}}

\renewcommand{\O}{\Omega}
\renewcommand{\o}{\omega}
\newcommand{\vp}{\varphi}
\newcommand{\vep}{\varepsilon}

\newcommand{\diag}{{\rm diag}}
\newcommand{\grp}{{\mathbb G}}
\newcommand{\dgrp}{{\mathbb D}}
\newcommand{\desp}{{\mathbb D^{\rm{es}}}}
\newcommand{\Geod}{{\rm Geod}}
\newcommand{\geod}{{\rm geod}}
\newcommand{\hgr}{{\mathbb H}}
\newcommand{\mgr}{{\mathbb M}}
\newcommand{\ob}{{\rm Ob}}
\newcommand{\obg}{{\rm Ob(\mathbb G)}}
\newcommand{\obgp}{{\rm Ob(\mathbb G')}}
\newcommand{\obh}{{\rm Ob(\mathbb H)}}
\newcommand{\Osmooth}{{\Omega^{\infty}(X,*)}}
\newcommand{\ghomotop}{{\rho_2^{\square}}}
\newcommand{\gcalp}{{\mathbb G(\mathcal P)}}

\newcommand{\rf}{{R_{\mathcal F}}}
\newcommand{\glob}{{\rm glob}}
\newcommand{\loc}{{\rm loc}}
\newcommand{\TOP}{{\rm TOP}}

\newcommand{\wti}{\widetilde}
\newcommand{\what}{\widehat}

\renewcommand{\a}{\alpha}
\newcommand{\be}{\beta}
\newcommand{\ga}{\gamma}
\newcommand{\Ga}{\Gamma}
\newcommand{\de}{\delta}
\newcommand{\del}{\partial}
\newcommand{\ka}{\kappa}
\newcommand{\si}{\sigma}
\newcommand{\ta}{\tau}
\newcommand{\med}{\medbreak}
\newcommand{\medn}{\medbreak \noindent}
\newcommand{\bign}{\bigbreak \noindent}
\newcommand{\lra}{{\longrightarrow}}
\newcommand{\ra}{{\rightarrow}}
\newcommand{\rat}{{\rightarrowtail}}
\newcommand{\oset}[1]{\overset {#1}{\ra}}
\newcommand{\osetl}[1]{\overset {#1}{\lra}}
\newcommand{\hr}{{\hookrightarrow}}
\begin{document}
\subsection{Introduction}

Several classes of groupoids and large groupoids shall be considered in this topic with pertinent examples that illustrate the construction of groupoids through several extensions of the much simpler (and global) group symmetry to both higher order symmetries and dimensions, as well as internal (or local, partial) plus external symmetry. Considered as powerful tools for investigating both Abelian and non-Abelian structures, groupoids are now essential for understanding topology, and are one of the important--\emph{if not the most important}-- concepts in algebraic topology (\cite{BR2006})

\subsection{Groupoids and topology}  

  Groupoids are {generalizations or extensions of the concept of group, supergroup, `virtual group', and paragroup}, in several ways; one may simply extend the notion of a group viewed as an one-object category to a \emph{many-object category with group-like elements and all invertible morphisms}. Another enrichment of the notion of a group--as in the case of topological groups-- is the concept of topological groupoid $\mathsf{G}$. One can also think of a groupoid as a class of linked groups, and further extend the latter groupoid definition to higher dimensions through `geometric'-algebraic constructions, for example, to double groupoids, cubic groupoids, ..., groupoid categories, groupoid supercategories, and so on. Crossed modules of groups and crossed complexes also correspond to such extended groupoids.

  For precise definitions of specific classes of groupoids, see also groupoid and topological groupoid definitions,
as well as those entries listed next as examples. 

\subsection{Additional examples} of major classes of groupoids defining the several extensions and enrichment possibilities of the notions of group and group symmetry introduced in the above definition are the subject of several other entries:

\begin{enumerate}
\item 2-groupoids (please see \emph{groupoid categories})
\item Double groupoids; homotopy double groupoid of a Hausdorff space
\item Higher homotopy groupoids and the higher dimensional, generalized van Kampen theorems
\item Groupoid category
\item Crossed complexes
\item Higher dimensional algebra (HDA)
\item Groupoid super-categories ($n$-categories, etc.)
\item Groupoid supercategories
\end{enumerate}
\med

\begin{thebibliography}{9}

\bibitem{BR2006}
R. Brown. 2006. {\em Topology and Groupoids}. Booksurge PLC. 

\bibitem{BR2k7et8}
R. Brown. 2008. {\em Nonabelian Algebraic Topology }. {\em preprint}, (two volumes).

\end{thebibliography}
%%%%%
%%%%%
\end{document}
