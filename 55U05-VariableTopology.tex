\documentclass[12pt]{article}
\usepackage{pmmeta}
\pmcanonicalname{VariableTopology}
\pmcreated{2013-03-22 18:15:39}
\pmmodified{2013-03-22 18:15:39}
\pmowner{bci1}{20947}
\pmmodifier{bci1}{20947}
\pmtitle{variable topology}
\pmrecord{11}{40858}
\pmprivacy{1}
\pmauthor{bci1}{20947}
\pmtype{Definition}
\pmcomment{trigger rebuild}
\pmclassification{msc}{55U05}
\pmclassification{msc}{55U35}
\pmclassification{msc}{55U40}
\pmclassification{msc}{18G55}
\pmclassification{msc}{18B40}
\pmsynonym{variable topology}{VariableTopology}
%\pmkeywords{sequence of different topologies defined by distinct axioms or rules}
\pmrelated{TopologicalSpace}
\pmrelated{VariableTopology3}
\pmdefines{indexed family of topological spaces}

% this is the default PlanetMath preamble.  as your knowledge
% of TeX increases, you will probably want to edit this, but
% it should be fine as is for beginners.

% almost certainly you want these
\usepackage{amssymb}
\usepackage{amsmath}
\usepackage{amsfonts}

% used for TeXing text within eps files
%\usepackage{psfrag}
% need this for including graphics (\includegraphics)
%\usepackage{graphicx}
% for neatly defining theorems and propositions
%\usepackage{amsthm}
% making logically defined graphics
%%%\usepackage{xypic}

% there are many more packages, add them here as you need them

% define commands here
\usepackage{amsmath, amssymb, amsfonts, amsthm, amscd, latexsym}
%%\usepackage{xypic}
\usepackage[mathscr]{eucal}

\setlength{\textwidth}{6.5in}
%\setlength{\textwidth}{16cm}
\setlength{\textheight}{9.0in}
%\setlength{\textheight}{24cm}

\hoffset=-.75in     %%ps format
%\hoffset=-1.0in     %%hp format
\voffset=-.4in

\theoremstyle{plain}
\newtheorem{lemma}{Lemma}[section]
\newtheorem{proposition}{Proposition}[section]
\newtheorem{theorem}{Theorem}[section]
\newtheorem{corollary}{Corollary}[section]

\theoremstyle{definition}
\newtheorem{definition}{Definition}[section]
\newtheorem{example}{Example}[section]
%\theoremstyle{remark}
\newtheorem{remark}{Remark}[section]
\newtheorem*{notation}{Notation}
\newtheorem*{claim}{Claim}

\renewcommand{\thefootnote}{\ensuremath{\fnsymbol{footnote%%@
}}}
\numberwithin{equation}{section}

\newcommand{\Ad}{{\rm Ad}}
\newcommand{\Aut}{{\rm Aut}}
\newcommand{\Cl}{{\rm Cl}}
\newcommand{\Co}{{\rm Co}}
\newcommand{\DES}{{\rm DES}}
\newcommand{\Diff}{{\rm Diff}}
\newcommand{\Dom}{{\rm Dom}}
\newcommand{\Hol}{{\rm Hol}}
\newcommand{\Mon}{{\rm Mon}}
\newcommand{\Hom}{{\rm Hom}}
\newcommand{\Ker}{{\rm Ker}}
\newcommand{\Ind}{{\rm Ind}}
\newcommand{\IM}{{\rm Im}}
\newcommand{\Is}{{\rm Is}}
\newcommand{\ID}{{\rm id}}
\newcommand{\GL}{{\rm GL}}
\newcommand{\Iso}{{\rm Iso}}
\newcommand{\Sem}{{\rm Sem}}
\newcommand{\St}{{\rm St}}
\newcommand{\Sym}{{\rm Sym}}
\newcommand{\SU}{{\rm SU}}
\newcommand{\Tor}{{\rm Tor}}
\newcommand{\U}{{\rm U}}

\newcommand{\A}{\mathcal A}
\newcommand{\Ce}{\mathcal C}
\newcommand{\D}{\mathcal D}
\newcommand{\E}{\mathcal E}
\newcommand{\F}{\mathcal F}
\newcommand{\G}{\mathcal G}
\newcommand{\Q}{\mathcal Q}
\newcommand{\R}{\mathcal R}
\newcommand{\cS}{\mathcal S}
\newcommand{\cU}{\mathcal U}
\newcommand{\W}{\mathcal W}

\newcommand{\bA}{\mathbb{A}}
\newcommand{\bB}{\mathbb{B}}
\newcommand{\bC}{\mathbb{C}}
\newcommand{\bD}{\mathbb{D}}
\newcommand{\bE}{\mathbb{E}}
\newcommand{\bF}{\mathbb{F}}
\newcommand{\bG}{\mathbb{G}}
\newcommand{\bK}{\mathbb{K}}
\newcommand{\bM}{\mathbb{M}}
\newcommand{\bN}{\mathbb{N}}
\newcommand{\bO}{\mathbb{O}}
\newcommand{\bP}{\mathbb{P}}
\newcommand{\bR}{\mathbb{R}}
\newcommand{\bV}{\mathbb{V}}
\newcommand{\bZ}{\mathbb{Z}}

\newcommand{\bfE}{\mathbf{E}}
\newcommand{\bfX}{\mathbf{X}}
\newcommand{\bfY}{\mathbf{Y}}
\newcommand{\bfZ}{\mathbf{Z}}

\renewcommand{\O}{\Omega}
\renewcommand{\o}{\omega}
\newcommand{\vp}{\varphi}
\newcommand{\vep}{\varepsilon}

\newcommand{\diag}{{\rm diag}}
\newcommand{\grp}{{\mathbb G}}
\newcommand{\dgrp}{{\mathbb D}}
\newcommand{\desp}{{\mathbb D^{\rm{es}}}}
\newcommand{\Geod}{{\rm Geod}}
\newcommand{\geod}{{\rm geod}}
\newcommand{\hgr}{{\mathbb H}}
\newcommand{\mgr}{{\mathbb M}}
\newcommand{\ob}{{\rm Ob}}
\newcommand{\obg}{{\rm Ob(\mathbb G)}}
\newcommand{\obgp}{{\rm Ob(\mathbb G')}}
\newcommand{\obh}{{\rm Ob(\mathbb H)}}
\newcommand{\Osmooth}{{\Omega^{\infty}(X,*)}}
\newcommand{\ghomotop}{{\rho_2^{\square}}}
\newcommand{\gcalp}{{\mathbb G(\mathcal P)}}

\newcommand{\rf}{{R_{\mathcal F}}}
\newcommand{\glob}{{\rm glob}}
\newcommand{\loc}{{\rm loc}}
\newcommand{\TOP}{{\rm TOP}}

\newcommand{\wti}{\widetilde}
\newcommand{\what}{\widehat}

\renewcommand{\a}{\alpha}
\newcommand{\be}{\beta}
\newcommand{\ga}{\gamma}
\newcommand{\Ga}{\Gamma}
\newcommand{\de}{\delta}
\newcommand{\del}{\partial}
\newcommand{\ka}{\kappa}
\newcommand{\si}{\sigma}
\newcommand{\ta}{\tau}
\newcommand{\med}{\medbreak}
\newcommand{\medn}{\medbreak \noindent}
\newcommand{\bign}{\bigbreak \noindent}
\newcommand{\lra}{{\longrightarrow}}
\newcommand{\ra}{{\rightarrow}}
\newcommand{\rat}{{\rightarrowtail}}
\newcommand{\oset}[1]{\overset {#1}{\ra}}
\newcommand{\osetl}[1]{\overset {#1}{\lra}}
\newcommand{\hr}{{\hookrightarrow}}
\begin{document}
\textbf{Preliminary data}
Let us recall the basic notion that a \emph{topological space}
consists of a set $X$ and a `topology' on $X$ where the latter
gives a precise but general sense to the intuitive ideas of
`nearness' and `continuity'. Thus the initial task is to
axiomatize the notion of `neighborhood' and then consider a
topology in terms of open or of closed sets, a compact-open
topology, and so on (see Brown, 2006). In any case, a topological
space consists of a pair $(X, \mathcal T)$ where $\mathcal T$ is a
topology on $X$. For instance, suppose an \emph{open set topology}
is given by the set $\mathcal U$ of prescribed open sets of $X$
satisfying the usual axioms (Brown, 2006 Chapter 2). Now, to speak
of a variable open-set topology one might conveniently take in
this case a family of sets $\mathcal U_{\lambda}$ of \emph{a
system of prescribed open sets}, where $\lambda$ belongs to some
indexing set $\Lambda$. The system of open sets may of course be
based on a system of contained neighbourhoods of points where one
system may have a different geometric property compared say to
another system (a system of disc-like neighbourhoods compared with
those of cylindrical-type). 


\begin{definition} In general, we may speak of a topological space with a 
\emph{varying topology} as a pair $(X, \mathcal T_{\lambda})$ where $\lambda \in \Lambda$ 
is an index set.
\end{definition}

\textbf{Example} The idea of a varying topology has been introduced to describe possible topological
distinctions in bio-molecular organisms through stages of
development, evolution, neo-plasticity, etc. This is indicated
schematically in the diagram below where we have an $n$-stage
dynamic evolution (through complexity) of categories $\mathsf D_i$
where the vertical arrows denote the assignment of topologies
$\mathcal T_i$ to the class of objects of the $\mathsf D_i$ along
with functors  $\F_{i} : \mathsf D_{i} \lra \mathsf D_{i+1}$, for
$1 \leq i \leq n-1$~:

$$
 \diagram  &  \mathcal
T_{1} \dto<-.05ex> &  \mathcal T_{2} \dto<-1.2ex> & \cdots
 &  \mathcal
T_{n-1}  \dto <-.05ex> &  \mathcal T_{n} \dto<-1ex>_(0.45){}
\\ & \mathsf D_{1}\rto^{\F_1}
&  \mathsf D_{2} \rto^{\F_2}  \rule{0.5em}{0ex}  & & \cdots
\rto^{\F_{n-1}} \rule{0.5em}{0ex} \mathsf D_{n-1} &
\rule{0em}{0ex} \mathsf D_{n}
\enddiagram
$$

In this way a \PMlinkname{variable topology}{VariableTopology} can be realized through such
$n$-levels of complexity of the development of an organism.


  Another example is that of  cell/network topologies in a categorical approach
involving concepts such as \emph{the free groupoid over a graph}
(Brown, 2006). Thus a \emph{varying graph system} clearly induces an
accompanying \PMlinkname{system of variable groupoids}{VariableTopology3}. As suggested by
Golubitsky and Stewart (2006), symmetry groupoids of various cell
networks would appear relevant to the physiology of animal locomotion as one example.



%%%%%
%%%%%
\end{document}
