\documentclass[12pt]{article}
\usepackage{pmmeta}
\pmcanonicalname{SimplicialApproximationTheorem}
\pmcreated{2013-03-22 16:54:29}
\pmmodified{2013-03-22 16:54:29}
\pmowner{Mathprof}{13753}
\pmmodifier{Mathprof}{13753}
\pmtitle{simplicial approximation theorem}
\pmrecord{5}{39168}
\pmprivacy{1}
\pmauthor{Mathprof}{13753}
\pmtype{Theorem}
\pmcomment{trigger rebuild}
\pmclassification{msc}{55U10}

\endmetadata

% this is the default PlanetMath preamble.  as your knowledge
% of TeX increases, you will probably want to edit this, but
% it should be fine as is for beginners.

% almost certainly you want these
\usepackage{amssymb}
\usepackage{amsmath}
\usepackage{amsfonts}

% used for TeXing text within eps files
%\usepackage{psfrag}
% need this for including graphics (\includegraphics)
%\usepackage{graphicx}
% for neatly defining theorems and propositions
%\usepackage{amsthm}
% making logically defined graphics
%%%\usepackage{xypic}

% there are many more packages, add them here as you need them

% define commands here

\begin{document}
Let $f: |K| \to |L|$ be continuous function, where $|K|$ and $|L|$
are polyhedra having triangulations $K$ and $L$, respectively.

Then there is a barycentric subdivision $K^{(s)}$ of $K$ and a continuous
function $g:|K| \to |L|$ such that $g$ is a simplicial map from $K^{(s)}$
to $|L|$ and $g$ is homotopic to $f$.

The theorem is due to J.W. Alexander. 

\begin{thebibliography}{99}
\bibitem{jwa} J.W. Alexander , \emph{Combinatorial analysis situs},
Trans. Amer. Math. Soc. \textbf{28}, 301-329, (1926)
\end{thebibliography} 

%%%%%
%%%%%
\end{document}
