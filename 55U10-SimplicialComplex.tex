\documentclass[12pt]{article}
\usepackage{pmmeta}
\pmcanonicalname{SimplicialComplex}
\pmcreated{2013-03-22 12:34:46}
\pmmodified{2013-03-22 12:34:46}
\pmowner{djao}{24}
\pmmodifier{djao}{24}
\pmtitle{simplicial complex}
\pmrecord{11}{32828}
\pmprivacy{1}
\pmauthor{djao}{24}
\pmtype{Definition}
\pmcomment{trigger rebuild}
\pmclassification{msc}{55U10}
\pmclassification{msc}{54E99}
\pmrelated{HomologyTopologicalSpace}
\pmrelated{CWComplex}
\pmdefines{simplicial homology}
\pmdefines{simplicial cohomology}
\pmdefines{triangulation}
\pmdefines{abstract simplicial complex}
\pmdefines{abstract $n$-simplex}

% this is the default PlanetMath preamble.  as your knowledge
% of TeX increases, you will probably want to edit this, but
% it should be fine as is for beginners.

% almost certainly you want these
\usepackage{amssymb}
\usepackage{amsmath}
\usepackage{amsfonts}

% used for TeXing text within eps files
%\usepackage{psfrag}
% need this for including graphics (\includegraphics)
%\usepackage{graphicx}
% for neatly defining theorems and propositions
%\usepackage{amsthm}
% making logically defined graphics
%%%\usepackage{xypic} 

% there are many more packages, add them here as you need them

% define commands here
\newcommand{\R}{\mathbb{R}}
\begin{document}
An abstract \emph{simplicial complex} $K$ is a collection of nonempty
finite sets with the property that for any element $\sigma \in K$, if
$\tau \subset \sigma$ is a nonempty subset, then $\tau \in K$. An
element of $K$ of cardinality $n+1$ is called an \emph{
$n$-simplex}. An element of an element of $K$ is called a \emph{vertex}. In what follows, we may occasionally identify a vertex $v$ with its corresponding singleton set $\{v\} \in K$; the reader
will be alerted when this is the case.

The \emph{standard $n$-complex}, denoted by $\Delta_n$, is the
simplicial complex consisting of all nonempty subsets of
$\{0,1,\ldots,n\}$.

\section{Geometry of a simplicial complex}
Let $K$ be a simplicial complex, and let $V$ be the set of vertices of
$K$. Although there is an established notion of infinite simplicial
complexes, the geometrical treatment of simplicial complexes is much simpler in the finite case and so for
now we will assume that $V$ is a finite set of cardinality $k$.

We introduce the vector space $\mathbb{R}^V$ of formal
$\R$--linear combinations of elements of $V$; i.e.,
$$
\R^V := \{a_1 V_1 + a_2 V_2 + \cdots + a_k V_k \mid a_i \in \R,\ V_i \in
V\},
$$
and the vector space operations are defined by formal addition and
scalar multiplication. Note that we may regard each vertex in $V$ as a
one-term formal sum, and thus as a point in $\R^V$.

The \emph{geometric realization} of $K$, denoted $|K|$, is the subset
of $\mathbb{R}^V$ consisting of the union, over all $\sigma \in K$, of
the convex hull of $\sigma \subset \mathbb{R}^V$. If we fix a bijection $\phi\colon V \to \{1, \ldots, k\}$, then the vector space $\mathbb{R}^V$ is isomorphic to the Euclidean vector space $\mathbb{R}^k$ via $\phi$, and the set $|K|$
inherits a metric from $\mathbb{R}^k$ making it into a metric space
and topological space. The isometry class of $K$ is independent of the choice of the bijection $\phi$.

\textbf{Examples:}

\begin{enumerate}
\item $\Delta_2 = \{\{0\}, \{1\}, \{2\}, \{0,1\}, \{0,2\}, \{1,2\},
\{0,1,2\}\}$ has $V = 3$, so its realization $|\Delta_2|$ is a subset
of $\mathbb{R}^3$, consisting of all points on the hyperplane $x+y+z =
1$ that are inside or on the boundary of the first octant. These
points form a triangle in $\mathbb{R}^3$ with one face, three edges,
and three vertices (for example, the convex hull of $\{0,1\} \in
\Delta_2$ is the edge of this triangle that lies in the $xy$--plane).
\item Similarly, the realization of the standard $n$--simplex
$\Delta_n$ is an $n$--dimensional tetrahedron contained inside
$\mathbb{R}^{n+1}$.
\item A triangle without interior (a ``wire frame'' triangle) can be
geometrically realized by starting from the simplicial complex
$\{\{0\}, \{1\}, \{2\}, \{0,1\}, \{0,2\}, \{1,2\}\}$.
\end{enumerate}

Notice that, under this procedure, an element of $K$ of cardinality 1
is geometrically a vertex; an element of cardinality 2 is an edge;
cardinality 3, a face; and, in general, an element of cardinality $n$
is realized as an $n$-face inside $\R^V$.

In general, a \emph{triangulation} of a topological space $X$ is a
simplicial complex $K$ together with a homeomorphism from $|K|$ to
$X$.

\section{Homology of a simplicial complex}

In this section we define the homology and cohomology groups
associated to a simplicial complex $K$. We do so not because the
homology of a simplicial complex is so intrinsically interesting in
and of itself, but because the resulting homology theory is identical
to the singular homology of the associated topological space $|K|$,
and therefore provides an accessible way to calculate the latter
homology groups (and, by extension, the homology of any space $X$
admitting a triangulation by $K$).

As before, let $K$ be a simplicial complex, and let $V$ be the set of
vertices in $K$. Let the chain group $C_n(K)$ be the subgroup of the
exterior algebra $\Lambda(\R^V)$ generated by all elements of the form
$V_0 \wedge V_1 \wedge \cdots \wedge V_n$ such that $V_i \in V$ and
$\{V_0, V_1, \ldots, V_n\} \in K$. Note that we are ignoring here the
$\R$--vector space structure of $\R^V$; the group $C_n(K)$ under this
definition is merely a free abelian group, generated by the
alternating products of the above form and with the relations that are
implied by the properties of the wedge product.

Define the boundary map $\partial_n: C_n(K) \longrightarrow
C_{n-1}(K)$ by the formula
$$
\partial_n(V_0 \wedge V_1 \wedge \cdots \wedge V_n) := \sum_{j=0}^n
(-1)^j (V_0 \wedge \cdots \wedge \hat{V_j} \wedge \cdots \wedge V_n),
$$
where the hat notation means the term under the hat is left out of the
product, and extending linearly to all of $C_n(K)$. Then one checks
easily that $\partial_{n-1} \circ \partial_n = 0$, so the collection
of chain groups $C_n(K)$ and boundary maps $\partial_n$ forms a chain
complex $\mathcal{C}(K)$. The simplicial homology and cohomology
groups of $K$ are defined to be that of $\mathcal{C}(K)$.

\textbf{Theorem:} The simplicial homology and cohomology groups of $K$,
as defined above, are canonically isomorphic to the singular homology
and cohomology groups of the geometric realization $|K|$ of $K$.

The proof of this theorem is considerably more difficult than what we
have done to this point, requiring the techniques of barycentric
subdivision and simplicial approximation, and we refer the interested
reader to~\cite{munkres}.

\begin{thebibliography}{9}
\bibitem{munkres}{Munkres, James. {\em Elements of Algebraic
Topology}, Addison--Wesley, New York, 1984.}
\end{thebibliography}
%%%%%
%%%%%
\end{document}
