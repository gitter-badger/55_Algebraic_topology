\documentclass[12pt]{article}
\usepackage{pmmeta}
\pmcanonicalname{Supersymmetry}
\pmcreated{2013-03-22 18:17:00}
\pmmodified{2013-03-22 18:17:00}
\pmowner{bci1}{20947}
\pmmodifier{bci1}{20947}
\pmtitle{supersymmetry}
\pmrecord{22}{40893}
\pmprivacy{1}
\pmauthor{bci1}{20947}
\pmtype{Definition}
\pmcomment{trigger rebuild}
\pmclassification{msc}{55U40}
\pmclassification{msc}{55-02}
\pmclassification{msc}{81Q60}
\pmclassification{msc}{81R50}
\pmclassification{msc}{81R15}
\pmsynonym{extended quantum symmetry structures}{Supersymmetry}
\pmsynonym{generalized double algebras}{Supersymmetry}
\pmsynonym{supersymmetries}{Supersymmetry}
%\pmkeywords{Supergravity}
%\pmkeywords{Quantum Gravity}
%\pmkeywords{Superspace}
%\pmkeywords{Relativistic QFT and quantum field theories}
\pmrelated{QuantumAlgebraicTopology}
\pmrelated{AlgebraicFoundationsOfQuantumAlgebraicTopology}
\pmrelated{SuperfieldsSuperspace}
\pmrelated{LieSuperalgebra3}
\pmdefines{extended quantum symmetry structures}
\pmdefines{both local and global}

% this is the default PlanetMath preamble.  as your knowledge
% of TeX increases, you will probably want to edit this, but
% it should be fine as is for beginners.

% almost certainly you want these
\usepackage{amssymb}
\usepackage{amsmath}
\usepackage{amsfonts}

% used for TeXing text within eps files
%\usepackage{psfrag}
% need this for including graphics (\includegraphics)
%\usepackage{graphicx}
% for neatly defining theorems and propositions
%\usepackage{amsthm}
% making logically defined graphics
%%%\usepackage{xypic}

% there are many more packages, add them here as you need them

% define commands here
\usepackage{amsmath, amssymb, amsfonts, amsthm, amscd, latexsym}
%%\usepackage{xypic}
\usepackage[mathscr]{eucal}

\setlength{\textwidth}{6.5in}
%\setlength{\textwidth}{16cm}
\setlength{\textheight}{9.0in}
%\setlength{\textheight}{24cm}

\hoffset=-.75in     %%ps format
%\hoffset=-1.0in     %%hp format
\voffset=-.4in

\theoremstyle{plain}
\newtheorem{lemma}{Lemma}[section]
\newtheorem{proposition}{Proposition}[section]
\newtheorem{theorem}{Theorem}[section]
\newtheorem{corollary}{Corollary}[section]

\theoremstyle{definition}
\newtheorem{definition}{Definition}[section]
\newtheorem{example}{Example}[section]
%\theoremstyle{remark}
\newtheorem{remark}{Remark}[section]
\newtheorem*{notation}{Notation}
\newtheorem*{claim}{Claim}

\renewcommand{\thefootnote}{\ensuremath{\fnsymbol{footnote%%@
}}}
\numberwithin{equation}{section}

\newcommand{\Ad}{{\rm Ad}}
\newcommand{\Aut}{{\rm Aut}}
\newcommand{\Cl}{{\rm Cl}}
\newcommand{\Co}{{\rm Co}}
\newcommand{\DES}{{\rm DES}}
\newcommand{\Diff}{{\rm Diff}}
\newcommand{\Dom}{{\rm Dom}}
\newcommand{\Hol}{{\rm Hol}}
\newcommand{\Mon}{{\rm Mon}}
\newcommand{\Hom}{{\rm Hom}}
\newcommand{\Ker}{{\rm Ker}}
\newcommand{\Ind}{{\rm Ind}}
\newcommand{\IM}{{\rm Im}}
\newcommand{\Is}{{\rm Is}}
\newcommand{\ID}{{\rm id}}
\newcommand{\GL}{{\rm GL}}
\newcommand{\Iso}{{\rm Iso}}
\newcommand{\Sem}{{\rm Sem}}
\newcommand{\St}{{\rm St}}
\newcommand{\Sym}{{\rm Sym}}
\newcommand{\SU}{{\rm SU}}
\newcommand{\Tor}{{\rm Tor}}
\newcommand{\U}{{\rm U}}

\newcommand{\A}{\mathcal A}
\newcommand{\Ce}{\mathcal C}
\newcommand{\D}{\mathcal D}
\newcommand{\E}{\mathcal E}
\newcommand{\F}{\mathcal F}
\newcommand{\G}{\mathcal G}
\newcommand{\Q}{\mathcal Q}
\newcommand{\R}{\mathcal R}
\newcommand{\cS}{\mathcal S}
\newcommand{\cU}{\mathcal U}
\newcommand{\W}{\mathcal W}

\newcommand{\bA}{\mathbb{A}}
\newcommand{\bB}{\mathbb{B}}
\newcommand{\bC}{\mathbb{C}}
\newcommand{\bD}{\mathbb{D}}
\newcommand{\bE}{\mathbb{E}}
\newcommand{\bF}{\mathbb{F}}
\newcommand{\bG}{\mathbb{G}}
\newcommand{\bK}{\mathbb{K}}
\newcommand{\bM}{\mathbb{M}}
\newcommand{\bN}{\mathbb{N}}
\newcommand{\bO}{\mathbb{O}}
\newcommand{\bP}{\mathbb{P}}
\newcommand{\bR}{\mathbb{R}}
\newcommand{\bV}{\mathbb{V}}
\newcommand{\bZ}{\mathbb{Z}}

\newcommand{\bfE}{\mathbf{E}}
\newcommand{\bfX}{\mathbf{X}}
\newcommand{\bfY}{\mathbf{Y}}
\newcommand{\bfZ}{\mathbf{Z}}

\renewcommand{\O}{\Omega}
\renewcommand{\o}{\omega}
\newcommand{\vp}{\varphi}
\newcommand{\vep}{\varepsilon}

\newcommand{\diag}{{\rm diag}}
\newcommand{\grp}{{\mathbb G}}
\newcommand{\dgrp}{{\mathbb D}}
\newcommand{\desp}{{\mathbb D^{\rm{es}}}}
\newcommand{\Geod}{{\rm Geod}}
\newcommand{\geod}{{\rm geod}}
\newcommand{\hgr}{{\mathbb H}}
\newcommand{\mgr}{{\mathbb M}}
\newcommand{\ob}{{\rm Ob}}
\newcommand{\obg}{{\rm Ob(\mathbb G)}}
\newcommand{\obgp}{{\rm Ob(\mathbb G')}}
\newcommand{\obh}{{\rm Ob(\mathbb H)}}
\newcommand{\Osmooth}{{\Omega^{\infty}(X,*)}}
\newcommand{\ghomotop}{{\rho_2^{\square}}}
\newcommand{\gcalp}{{\mathbb G(\mathcal P)}}

\newcommand{\rf}{{R_{\mathcal F}}}
\newcommand{\glob}{{\rm glob}}
\newcommand{\loc}{{\rm loc}}
\newcommand{\TOP}{{\rm TOP}}

\newcommand{\wti}{\widetilde}
\newcommand{\what}{\widehat}

\renewcommand{\a}{\alpha}
\newcommand{\be}{\beta}
\newcommand{\ga}{\gamma}
\newcommand{\Ga}{\Gamma}
\newcommand{\de}{\delta}
\newcommand{\del}{\partial}
\newcommand{\ka}{\kappa}
\newcommand{\si}{\sigma}
\newcommand{\ta}{\tau}
\newcommand{\med}{\medbreak}
\newcommand{\medn}{\medbreak \noindent}
\newcommand{\bign}{\bigbreak \noindent}
\newcommand{\lra}{{\longrightarrow}}
\newcommand{\ra}{{\rightarrow}}
\newcommand{\rat}{{\rightarrowtail}}
\newcommand{\oset}[1]{\overset {#1}{\ra}}
\newcommand{\osetl}[1]{\overset {#1}{\lra}}
\newcommand{\hr}{{\hookrightarrow}}
\begin{document}
\begin{definition}
\emph{Supersymmetry} or Poincar\'e, (extended) quantum symmetry is usually defined as an extension of ordinary spacetime symmetries obtained by adjoining $N$ spinorial generators $Q$ whose anticommutator yields a
translation generator: $\left\{Q ,Q \right\} = \left\{P\right\}$.
\end{definition}
 
As further explained in ref. \cite{JSG98}: 
\begin{quote}
``This \emph{(super)} symmetry...(of the \emph{superspace})... can be realized on ordinary fields (that are defined as certain functions of physical spacetime(s)) by transformations that mix bosons and fermions. 
\emph{Such realizations suffice to study supersymmetry (one can write invariant actions, etc.) but are as
cumbersome and inconvenient as doing vector calculus component by component. A compact alternative to this `component field' approach is given by the \emph{superspace--superfield} approach}", which is defined next.  
\end{quote}  

\begin{definition}
\emph{Quantum superspace, or superspacetimes}, can be defined as an extension(s) of ordinary spacetime(s) to include
additional anticommuting coordinates, for example, in the form of $N$ two-component Weyl spinors $\theta$.
\end{definition}

\begin{definition}
\emph{(Quantum) superfields} $\Psi(x , \theta)$ are \emph{functions} defined over such superspaces, or superspacetimes. 
Taylor series expansions of the superfield functions can be then performed with respect to the anticommuting coordinates $\theta$; this Taylor series has only a finite number of terms and the series expansion
coefficients obtained in this manner are the ordinary `component fields' specified above. 
\end{definition}

\textbf{Remarks:}
Supersymmetry is expected to be manifested, or observable, in such superspaces, that is, the  \emph{supersymmetry algebras} are represented by translations and rotations involving \emph{both} the spacetime and the anticommuting coordinates. Then, the transformations of the `component fields' can be computed from the Taylor expansion of
the \emph{translated and rotated superfields}. Especially important are those transformations that mix boson
and fermion symmetries; further details are found in ref. \cite{LS2k}.


\begin{thebibliography}{9}
\bibitem{JSG98}
J.S. Gates, Jr, et al. ``Superspace''.,  arxiv-hep-th/0108200 preprint (1983).

\bibitem{LS2k}
``Preprint of 1,001 Lessons in Supersymmetry.'' \PMlinkexternal{on line PDF}{http://arxiv.org/abs/hep-th/0108200}.

\end{thebibliography}
%%%%%
%%%%%
\end{document}
