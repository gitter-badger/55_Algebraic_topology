\documentclass[12pt]{article}
\usepackage{pmmeta}
\pmcanonicalname{BibliographyForAxiomaticsAndMathematicsFoundationsInCategories}
\pmcreated{2013-03-22 18:19:31}
\pmmodified{2013-03-22 18:19:31}
\pmowner{bci1}{20947}
\pmmodifier{bci1}{20947}
\pmtitle{bibliography for axiomatics and mathematics foundations in categories}
\pmrecord{42}{40956}
\pmprivacy{1}
\pmauthor{bci1}{20947}
\pmtype{Bibliography}
\pmcomment{trigger rebuild}
\pmclassification{msc}{55U99}
\pmclassification{msc}{03B50}
\pmclassification{msc}{18A05}
\pmclassification{msc}{18-00}
\pmclassification{msc}{03-00}
\pmclassification{msc}{00A15}
\pmsynonym{metamathematics}{BibliographyForAxiomaticsAndMathematicsFoundationsInCategories}
\pmsynonym{mathematical foundations}{BibliographyForAxiomaticsAndMathematicsFoundationsInCategories}
\pmsynonym{axiomatics and formal logics}{BibliographyForAxiomaticsAndMathematicsFoundationsInCategories}
%\pmkeywords{bibliography for axiomatics}
%\pmkeywords{mathematics foundations in categories}
\pmrelated{BibliographyForMathematicalPhysicsFoundationsAxiomaticsAndCategories}
\pmrelated{NonAbelianTheories}
\pmrelated{TopicEntryOnMiscellaneousMathematics}

\endmetadata

% this is the default PlanetMath preamble.  
% almost certainly you want these
\usepackage{amssymb}
\usepackage{amsmath}
\usepackage{amsfonts}
% need this for including graphics (\includegraphics)
%\usepackage{graphicx}
% for neatly defining theorems and propositions
%\usepackage{amsthm}
% making logically defined graphics
%%%\usepackage{xypic}

% define commands here
\usepackage{amsmath, amssymb, amsfonts, amsthm, amscd, latexsym}
%%\usepackage{xypic}
\usepackage[mathscr]{eucal}

\setlength{\textwidth}{6.5in}
%\setlength{\textwidth}{16cm}
\setlength{\textheight}{9.0in}
%\setlength{\textheight}{24cm}

\hoffset=-.75in     %%ps format
%\hoffset=-1.0in     %%hp format
\voffset=-.4in

\theoremstyle{plain}
\newtheorem{lemma}{Lemma}[section]
\newtheorem{proposition}{Proposition}[section]
\newtheorem{theorem}{Theorem}[section]
\newtheorem{corollary}{Corollary}[section]

\theoremstyle{definition}
\newtheorem{definition}{Definition}[section]
\newtheorem{example}{Example}[section]
%\theoremstyle{remark}
\newtheorem{remark}{Remark}[section]
\newtheorem*{notation}{Notation}
\newtheorem*{claim}{Claim}

\renewcommand{\thefootnote}{\ensuremath{\fnsymbol{footnote
}}}
\numberwithin{equation}{section}

\newcommand{\Ad}{{\rm Ad}}
\newcommand{\Aut}{{\rm Aut}}
\newcommand{\Cl}{{\rm Cl}}
\newcommand{\Co}{{\rm Co}}
\newcommand{\DES}{{\rm DES}}
\newcommand{\Diff}{{\rm Diff}}
\newcommand{\Dom}{{\rm Dom}}
\newcommand{\Hol}{{\rm Hol}}
\newcommand{\Mon}{{\rm Mon}}
\newcommand{\Hom}{{\rm Hom}}
\newcommand{\Ker}{{\rm Ker}}
\newcommand{\Ind}{{\rm Ind}}
\newcommand{\IM}{{\rm Im}}
\newcommand{\Is}{{\rm Is}}
\newcommand{\ID}{{\rm id}}
\newcommand{\GL}{{\rm GL}}
\newcommand{\Iso}{{\rm Iso}}
\newcommand{\Sem}{{\rm Sem}}
\newcommand{\St}{{\rm St}}
\newcommand{\Sym}{{\rm Sym}}
\newcommand{\SU}{{\rm SU}}
\newcommand{\Tor}{{\rm Tor}}
\newcommand{\U}{{\rm U}}

\newcommand{\A}{\mathcal A}
\newcommand{\Ce}{\mathcal C}
\newcommand{\D}{\mathcal D}
\newcommand{\E}{\mathcal E}
\newcommand{\F}{\mathcal F}
\newcommand{\G}{\mathcal G}
\newcommand{\Q}{\mathcal Q}
\newcommand{\R}{\mathcal R}
\newcommand{\cS}{\mathcal S}
\newcommand{\cU}{\mathcal U}
\newcommand{\W}{\mathcal W}

\newcommand{\bA}{\mathbb{A}}
\newcommand{\bB}{\mathbb{B}}
\newcommand{\bC}{\mathbb{C}}
\newcommand{\bD}{\mathbb{D}}
\newcommand{\bE}{\mathbb{E}}
\newcommand{\bF}{\mathbb{F}}
\newcommand{\bG}{\mathbb{G}}
\newcommand{\bK}{\mathbb{K}}
\newcommand{\bM}{\mathbb{M}}
\newcommand{\bN}{\mathbb{N}}
\newcommand{\bO}{\mathbb{O}}
\newcommand{\bP}{\mathbb{P}}
\newcommand{\bR}{\mathbb{R}}
\newcommand{\bV}{\mathbb{V}}
\newcommand{\bZ}{\mathbb{Z}}

\newcommand{\bfE}{\mathbf{E}}
\newcommand{\bfX}{\mathbf{X}}
\newcommand{\bfY}{\mathbf{Y}}
\newcommand{\bfZ}{\mathbf{Z}}

\renewcommand{\O}{\Omega}
\renewcommand{\o}{\omega}
\newcommand{\vp}{\varphi}
\newcommand{\vep}{\varepsilon}

\newcommand{\diag}{{\rm diag}}
\newcommand{\grp}{{\mathbb G}}
\newcommand{\dgrp}{{\mathbb D}}
\newcommand{\desp}{{\mathbb D^{\rm{es}}}}
\newcommand{\Geod}{{\rm Geod}}
\newcommand{\geod}{{\rm geod}}
\newcommand{\hgr}{{\mathbb H}}
\newcommand{\mgr}{{\mathbb M}}
\newcommand{\ob}{{\rm Ob}}
\newcommand{\obg}{{\rm Ob(\mathbb G)}}
\newcommand{\obgp}{{\rm Ob(\mathbb G')}}
\newcommand{\obh}{{\rm Ob(\mathbb H)}}
\newcommand{\Osmooth}{{\Omega^{\infty}(X,*)}}
\newcommand{\ghomotop}{{\rho_2^{\square}}}
\newcommand{\gcalp}{{\mathbb G(\mathcal P)}}

\newcommand{\rf}{{R_{\mathcal F}}}
\newcommand{\glob}{{\rm glob}}
\newcommand{\loc}{{\rm loc}}
\newcommand{\TOP}{{\rm TOP}}

\newcommand{\wti}{\widetilde}
\newcommand{\what}{\widehat}

\renewcommand{\a}{\alpha}
\newcommand{\be}{\beta}
\newcommand{\ga}{\gamma}
\newcommand{\Ga}{\Gamma}
\newcommand{\de}{\delta}
\newcommand{\del}{\partial}
\newcommand{\ka}{\kappa}
\newcommand{\si}{\sigma}
\newcommand{\ta}{\tau}
\newcommand{\med}{\medbreak}
\newcommand{\medn}{\medbreak \noindent}
\newcommand{\bign}{\bigbreak \noindent}
\newcommand{\lra}{{\longrightarrow}}
\newcommand{\ra}{{\rightarrow}}
\newcommand{\rat}{{\rightarrowtail}}
\newcommand{\oset}[1]{\overset {#1}{\ra}}
\newcommand{\osetl}[1]{\overset {#1}{\lra}}
\newcommand{\hr}{{\hookrightarrow}}
\begin{document}
\subsection{A Bibliography for Axiomatic Theories and Categorical Foundations of Mathematical Physics and Mathematics}
 
\subsubsection{a. Foundations of Mathematics, Logics and 
\PMlinkname{Formal Logics}{AnalyticsAndOntologyFormalLogics}: 
Axiomatics, Categories, Topoi and \PMlinkname{Higher Dimensional Algebra}{HigherDimensionalAlgebraHDA}}

\begin{thebibliography}{99}

\bibitem{AS96}
Awodey, S. 1996. ``Structure in Mathematics and Logic: A Categorical Perspective.'', {\em Philosophia Mathematica}, 3, 209--237. 

\bibitem{AS2k6}
Awodey, S., 2006, {\em Category Theory}, Oxford: Clarendon Press.

\bibitem{BAJ-DJ98a}
Baez, J. and  Dolan, J., 1998a, Higher-Dimensional Algebra III. $n$-Categories and the Algebra of Opetopes, 
{\em Advances in Mathematics}, 135: 145--206.  

\bibitem{BAJ-DJ98B}
Baez, J. and  Dolan, J., 1998b, ``Categorification'', {\em Higher Category Theory, Contemporary Mathematics}, 230, Providence: AMS, 1--36. 

\bibitem{BBGG1}
Baianu I. C., Brown R., Georgescu G. and J. F. Glazebrook: 2006, Complex Nonlinear Biodynamics in Categories, Higher Dimensional Algebra and \L{}ukasiewicz-Moisil Topos: Transformations of Neuronal, Genetic and Neoplastic Networks., 
\emph{Axiomathes}, \textbf{16} Nos. 1-2: 65-122.

\bibitem{ICBDS73}
Baianu, I.C. and D. Scripcariu: 1973, On Adjoint Dynamical Systems. 
\emph{Bulletin of Mathematical Biophysics}, \textbf{35}(4): 475-486.

\bibitem{Bgg2}
Baianu, I. C., Glazebrook, J. F. and G. Georgescu: 2004, Categories of Quantum Automata and N-Valued \L ukasiewicz Algebras in Relation to Dynamic Bionetworks, \textbf{(M,R)}-Systems and Their Higher Dimensional Algebra,
\emph{Preprint of Report}.

\bibitem{Ba-We2k}
Barr, M. and C.~Wells. {\em Toposes, Triples and Theories}. Montreal: McGill University, 2000.

\bibitem{BM-CW99}
Barr, M. and Wells, C. 1999.\emph{Category Theory for Computing Science}, Montreal: CRM. 
 
\bibitem{BaM98}
Batanin, M. 1998. Monoidal Globular Categories as a Natural Environment for the Theory of Weak n-Categories., Advances in Mathematics, 136, 39--103.   

\bibitem{BJL81}
Bell, J. L. 1981. Category Theory and the Foundations of Mathematics, \emph{British Journal for the Philosophy of Science}, 32, 349--358. 
 
\bibitem{BJL82}
Bell, J. L., 1982. Categories, Toposes and Sets, \emph{Synthese}, 51(3): 293--337. 
 
\bibitem{BA-SA83}
Blass, A. and Scedrov, A., 1983, Classifying Topoi and Finite Forcing , Journal of Pure and Applied Algebra, 28, 111--140. 

\bibitem{BA-SA89}
Blass, A. and Scedrov, A., 1989, Freyd's Model for the Independence of the Axiom of Choice, Providence: AMS.  

\bibitem{BASA92}
Blass, A. and Scedrov, A., 1992. Complete Topoi Representing Models of Set Theory, Annals of Pure and Applied Logic , 57, no. 1, 1--26.  

\bibitem{BA84}
Blass, A., 1984, The Interaction Between Category Theory and Set Theory., Mathematical Applications of Category Theory, 30, Providence: AMS, 5--29. 

\bibitem{BR-SP2k4}
Blute, R. and Scott, P., 2004, Category Theory for Linear Logicians., in Linear Logic in Computer Science

\bibitem{BP2k3}
Brown R. and T. Porter: 2003, Category theory and higher dimensional algebra: potential descriptive tools in neuroscience, In: {\em Proceedings of the International Conference on Theoretical Neurobiology}, Delhi, February 2003, edited by Nandini Singh, National Brain Research Centre, {\em Conference Proceedings} \textbf{1}: 80-92.

\bibitem{Br-Har-Ka-Po2k2}
Brown, R., Hardie, K., Kamps, H. and T. Porter: 2002, The homotopy double groupoid of a Hausdorff space., 
\emph{Theory and Applications of Categories} \textbf{10}, 71-93.

\bibitem{Br-Sp76}
Brown, R. and Spencer, C.B.: 1976, Double groupoids and crossed modules, \emph{Cah.  Top. G\'{e}om. Diff.} \textbf{17}, 343-362.

\bibitem{BR-SCB76}
Brown R, Razak Salleh A (1999) Free crossed resolutions of groups and presentations of modules of
identities among relations. {\em LMS J. Comput. Math.}, \textbf{2}: 25--61.

\bibitem{BDA55}
Buchsbaum, D. A.: 1955, Exact categories and duality., {\em Trans. Amer. Math. Soc.} \textbf{80}: 1-34.

\bibitem{BL2k3}
Bunge, M. and S. Lack: 2003, Van Kampen theorems for toposes, \emph{Adv. in Math.} \textbf{179}, 291-317.

\bibitem{BM84}
Bunge, M., 1984, Toposes in Logic and Logic in Toposes, {\em Topoi}, 3, no. 1, 13-22. 

\bibitem{BM-LS2k3}
Bunge M, Lack S (2003) Van Kampen theorems for toposes. {\em Adv Math}, \textbf {179}: 291-317.

\bibitem{EC}
Ehresmann, C.: 1965, \emph{Cat\'egories et Structures}, Dunod, Paris.

\bibitem{EC}
Ehresmann, C.: 1966, Trends Toward Unity in Mathematics., \emph{Cahiers de Topologie et Geometrie Differentielle}
\textbf{8}: 1-7.

\end{thebibliography}


\subsubsection{b. Universal Algebra, Classes of Algebraic Structures and Homology; Abelian and 
\PMlinkname{Non-Abelian theories}{NonAbelianTheories}; Algebraic 
Geometry and \PMlinkname{Noncommutative Geometry}{NoncommutativeGeometry}}.
 
\begin{thebibliography}{199}

\bibitem{BHR2}
Brown, R., Higgins, P. J. and R. Sivera,: 2007, \emph{Non-Abelian Algebraic Topology}, 
\PMlinkexternal{vol.I pdf doc.}{http://www.bangor.ac.uk/~mas010/nonab-t/partI010604.pdf};
\PMlinkexternal{Review of Part I and full contents PDF doc.}{http://planetmath.org/?op=getobj&from=lec&id=75}

\bibitem{RB2k8}
R. Brown. 2008. {\em Higher Dimensional Algebra Preprint as pdf and ps docs. at arXiv:math/0212274v6 [math.AT]}

\bibitem{Br-Hardy76}
Brown, R., and Hardy, J.P.L.:1976, ``Topological groupoids I: universal constructions.'', \emph{Math. Nachr.}, \textbf{71}: 273-286.

\bibitem{CH-ES56}
Cartan, H. and Eilenberg, S. 1956. {\em Homological Algebra}, Princeton Univ. Press: Pinceton.

\bibitem{CC46}
Chevalley, C. 1946. {\em The theory of Lie groups}. Princeton University Press, Princeton NJ.

\bibitem{Chaician}
M. Chaician and A. Demichev. 1996. {\em Introduction to Quantum Groups}, World Scientific .

\bibitem{CPM65}
Cohen, P.M. 1965. {\em Universal Algebra}, Harper and Row: New York, London and Tokyo.

\bibitem{CA94}
Connes A 1994. \emph{Noncommutative geometry}. Academic Press: New York.

\bibitem{CR-LL63}
Croisot, R. and Lesieur, L. 1963. \emph{Alg\`ebre noeth\'erienne non-commutative.},
Gauthier-Villard: Paris.

\bibitem{CRL94}
Crole, R.L., 1994, {\em Categories for Types}, Cambridge: Cambridge University Press.  

\bibitem{DJ-ALEX60-71}
Dieudonn\'e, J. and Grothendieck, A., 1960, [1971], {\em \'El\'ements de G\'eom\'etrie Alg\'ebrique}, Berlin: Springer-Verlag.  

\bibitem{Dixmier}
Dixmier, J., 1981, Von Neumann Algebras, Amsterdam: North-Holland Publishing Company. [First published in French in 1957: Les Algebres d'Operateurs dans l'Espace Hilbertien, Paris: Gauthier--Villars.]

\bibitem{Durdevich1}
M. Durdevich : Geometry of quantum principal bundles I, Commun. Math. Phys. \textbf{175} (3) (1996), 457--521.

\bibitem{Durdevich2}
M. Durdevich : Geometry of quantum principal bundles II, Rev.Math. Phys. \textbf{9} (5) (1997), 531-607.

\bibitem{Eh-pseudo}
Ehresmann, C.: 1952, Structures locales et structures infinit\'esimales, \emph{C.R.A.S.} Paris \textbf{274}: 587-589.

\bibitem{Eh}
Ehresmann, C.: 1959, Cat\'egories topologiques et cat\'egories diff\'erentiables, \emph{Coll. G\'eom. Diff. Glob.} Bruxelles, pp.137-150.

\bibitem{Eh-quintettes}
Ehresmann, C.:1963, Cat\'egories doubles des quintettes: applications covariantes , \emph{C.R.A.S. Paris}, \textbf{256}: 1891--1894.

\bibitem{Eh-Oe}
Ehresmann, C.: 1984, \emph{Oeuvres compl\`etes et  comment\'ees: Amiens, 1980-84}, edited and commented by Andr\'ee Ehresmann.

\bibitem{EML1}
Eilenberg, S. and S. Mac Lane.: 1942, Natural Isomorphisms in Group Theory., \emph{American Mathematical Society 43}: 757-831.

\bibitem{EL}
Eilenberg, S. and S. Mac Lane: 1945, The General Theory of Natural Equivalences, \emph{Transactions of the American Mathematical Society} \textbf{58}: 231-294.

\bibitem{ES-CH56}
Eilenberg, S. \& Cartan, H., 1956, {\em Homological Algebra}, Princeton: Princeton University Press. 

\bibitem{ES-MCLS42}
Eilenberg, S. \& MacLane, S., 1942, Group Extensions and Homology, {\em Annals of Mathematics}, 43, 757--831. 

\bibitem{ES-SN52}
Eilenberg, S. \& Steenrod, N., 1952, {\em Foundations of Algebraic Topology}, Princeton: Princeton University Press. 

\bibitem{ES60}
Eilenberg, S.: 1960. Abstract description of some basic functors., J. Indian Math.Soc., \textbf{24} :221-234.

\bibitem{S.Eilenberg}
S.Eilenberg. Relations between Homology and Homotopy Groups. {\em Proc.Natl.Acad.Sci.USA} (1966),v:10--14.  

\bibitem{ED88}
Ellerman, D., 1988, Category Theory and Concrete Universals, {\em Synthese}, 28, 409--429. 

\bibitem{ETH}
Ezawa,Z.F.,  G. Tsitsishvilli and K. Hasebe : Noncommutative geometry, extended $W_{\infty}$ algebra and Grassmannian solitons in multicomponent Hall systems, (at arXiv:hep--th/0209198).

\bibitem{FP2k2}
Freyd, P., 2002, Cartesian Logic, {\em Theoretical Computer Science}, 278, no. 1--2, 3--21.  

\bibitem{FP-FH-SA87}
Freyd, P., Friedman, H. \& Scedrov, A., 1987, Lindembaum Algebras of Intuitionistic Theories and Free Categories, {\em Annals of Pure and Applied Logic}, 35, 2, 167--172.

\bibitem{Gablot}
Gablot, R. 1971. Sur deux classes de cat\'{e}gories de Grothendieck. {\em Thesis},  Univ. de Lille.

\bibitem{Gabriel1}
Gabriel, P.: 1962, Des cat\'egories ab\'eliennes, \emph{Bull. Soc. Math. France} \textbf{90}: 323-448.

\bibitem{Gabriel2}
Gabriel, P. and M.Zisman:. 1967: \emph{Category of fractions and homotopy theory}, \emph{Ergebnesse der math.} Springer: Berlin.

\bibitem{GabrielNP}
Gabriel, P. and N. Popescu: 1964, Caract\'{e}risation des cat\'egories ab\'eliennes
avec g\'{e}n\'{e}rateurs et limites inductives. , \emph{CRAS Paris} \textbf{258}: 4188-4191.

\bibitem{GA-RG-SM2k}
Galli, A. \& Reyes, G. \& Sagastume, M., 2000, Completeness Theorems via the Double Dual Functor, 
{\em Studia Logica}, \textbf{64}, no. 1: 61--81. 

\bibitem{GN}
Gelfan'd, I. and Naimark, M., 1943, On the Imbedding of Normed Rings into the Ring of Operators in Hilbert Space, Recueil Math\'ematique [Matematicheskii Sbornik] Nouvelle S\'erie, 12 [54]: 197--213. [Reprinted in C*--algebras: 
1943--1993, in the series Contemporary Mathematics, 167,  Providence, R.I.: American Mathematical Society, 1994.] 

\bibitem{GS-ZM2K2}
Ghilardi, S. \& Zawadowski, M., 2002, {\em Sheaves, Games \& Model Completions: A Categorical Approach to Nonclassical Propositional Logics}, Dordrecht: Kluwer.  

\bibitem{gs89}
Ghilardi, S., 1989, Presheaf Semantics and Independence Results for some Non-classical first-order logics, 
{\em Archive for Mathematical Logic}, 29, no. 2, 125--136. 

\bibitem{Gob68}
Goblot, R., 1968, Cat\'egories modulaires , {\em C. R. Acad. Sci. Paris, S\'erie A.}, \textbf{267}: 381--383.

\bibitem{Gob71}
Goblot, R., 1971, Sur deux classes de cat\'egories de Grothendieck, {\em Th\`ese.}, Univ. Lille, 1971.

\bibitem{GR79}
Goldblatt, R., 1979, Topoi: The Categorical Analysis of Logic, Studies in logic and the foundations of mathematics, Amsterdam: Elsevier North-Holland Publ. Comp. 

\bibitem{Goldie}
Goldie, A. W., 1964, Localization in non-commutative noetherian rings, {\em J.Algebra}, \textbf{1}: 286-297.

\bibitem{Godement}
Godement,R. 1958. Th\'{e}orie des faisceaux. Hermann: Paris.

\bibitem{GRAY65}
Gray, C. W.: 1965. Sheaves with values in a category.,\emph {Topology}, 3: 1-18.

\bibitem{Alex71}
Grothendieck, A.: 1971, Rev\^{e}tements \'Etales et Groupe Fondamental (SGA1),
chapter VI: Cat\'egories fibr\'ees et descente, \emph{Lecture Notes in Math.}
\textbf{224}, Springer--Verlag: Berlin.

\bibitem{Alex57}
Grothendieck, A.: 1957, Sur quelque point d-alg\`{e}bre homologique. , \emph{Tohoku Math. J.}, \textbf{9:} 119-121.

\bibitem{Alex3}
Grothendieck, A. and J. Dieudon\'{e}.: 1960, El\'{e}ments de geometrie alg\'{e}brique., \emph{Publ. Inst. des Hautes Etudes de Science}, \textbf{4}.

\bibitem{ALEXsem}
Grothendieck, A. et al., S\'eminaire de G\'eom\'etrie Alg\'ebrique, Vol. 1--7, Berlin: Springer-Verlag.

\bibitem{HKK}
Hardie, K.A. K.H. Kamps and R.W. Kieboom, A homotopy 2-groupoid of a Hausdorff space, {\em Applied Cat. Structures} 8 (2000), 209-234.

\bibitem{HWS82}
Hatcher, W. S., 1982, {\em The Logical Foundations of Mathematics}, Oxford: Pergamon Press. 
  
\bibitem{Heller58}
Heller, A. :1958, Homological algebra in Abelian categories., \emph{Ann. of Math.}
\textbf{68}: 484-525.

\bibitem{HellerRowe62}
Heller, A.  and K. A. Rowe.:1962, On the category of sheaves., \emph{Amer J. Math.}
\textbf{84}: 205-216.

\bibitem{HG2k3}
Hellman, G., 2003, "Does Category Theory Provide a Framework for Mathematical Structuralism?", Philosophia Mathematica, 11, 2, 129--157. 

\bibitem{HC-MM-PJ2K}
Hermida, C. \& Makkai, M. \& Power, J., 2000, On Weak Higher-dimensional Categories I, Journal of Pure and Applied Algebra, 154, no. 1-3, 221--246. 

\bibitem{HC-MM-PI2K1}
Hermida, C. \& Makkai, M. \& Power, J., 2001, On Weak Higher-dimensional Categories II, Journal of Pure and Applied Algebra, 157, no. 2-3, 247--277.  

\bibitem{HC-MM-PI2K2}
Hermida, C. \& Makkai, M. \& Power, J., 2002, On Weak Higher-dimensional Categories III, Journal of Pure and Applied Algebra, 166, no. 1-2, 83--104.  

\bibitem{HPJbook}
Higgins, P. J.: 2005, \emph{Categories and groupoids}, Van Nostrand Mathematical Studies: 32, (1971); \emph{Reprints in
Theory and Applications of Categories}, No. 7: 1-195.

\bibitem{HPJ2k5}
Higgins, Philip J. Thin elements and commutative shells in cubical $\omega$-categories. Theory Appl. Categ. 14 (2005), No. 4, 60--74 (electronic). msc: 18D05.

\bibitem{HJ-RE-RG90}
Hyland,  J.M.E. \& Robinson,  E.P. \& Rosolini, G., 1990, The Discrete Objects in the Effective Topos, 
{\em Proceedings of the London Mathematical Society} (3), 60, no. 1, 1--36. 

\bibitem{HJME82}
Hyland,  J.M.E., 1982, The Effective Topos, {\em Studies in Logic and the Foundations of Mathematics}, 110, Amsterdam: North Holland, 165--216.  

\bibitem{HJME88}
Hyland, J. M..E., 1988, A Small Complete Category, {\em Annals of Pure and Applied Logic}, 40, no. 2, 135--165. 

\bibitem{HJME91}
Hyland,  J. M .E., 1991, First Steps in Synthetic Domain Theory, {\em Category Theory (Como 1990)}, Lecture Notes in Mathematics, 1488, Berlin: Springer, 131-156.  

\bibitem{HJME2K2}
Hyland, J. M.E., 2002, Proof Theory in the Abstract, {\em Annals of Pure and Applied Logic}, 114, no. 1--3, 43--78. 

\bibitem{E.Hurewicz}
E.Hurewicz. CW Complexes., {\em Trans AMS}.1955.

\bibitem{JB99}
Jacobs, B., 1999, Categorical Logic and Type Theory, Amsterdam: North Holland.  

\bibitem{JPT77}
Johnstone, P. T., 1977, Topos Theory, New York: Academic Press. 

\bibitem{JPT79A}
Johnstone, P. T., 1979a, {\em Conditions Related to De Morgan's Law, Applications of Sheaves}, Lecture Notes in Mathematics, 753, Berlin: Springer, 479--491. 

\bibitem{JPT81}
Johnstone, P. T., 1981, Tychonoff's Theorem without the Axiom of Choice, 
{\em Fundamenta Mathematicae}, 113, no. 1, 21--35. 

\bibitem{JPT52}
Johnstone, P. T., 1982, {\em Stone Spaces}, Cambridge:Cambridge University Press.  

\bibitem{JPT85}
Johnstone, P. T., 1985, How General is a Generalized Space?, {\em Aspects of Topology}, Cambridge: Cambridge University Press, 77--111. 

\bibitem{JAMI95}
Joyal, A. \& Moerdijk, I., 1995, {\em Algebraic Set Theory}, Cambridge: Cambridge University Press.  

\bibitem{kampen1-1933}
Van Kampen, E. H.: 1933, On the Connection Between the Fundamental
Groups of some Related Spaces, \emph{Amer. J. Math.} \textbf{55}: 261-267

\bibitem{KDM58}
Kan, D. M., 1958, Adjoint Functors, {\em Transactions of the American Mathematical Society} 87, 294-329.  

\bibitem{Kleisli62}
Kleisli, H.: 1962, Homotopy theory in Abelian categories.,{\em Can. J. Math.}, \textbf{14}: 139-169.

\bibitem{KJT70}
Knight, J.T., 1970, On epimorphisms of non-commutative rings., {\em Proc. Cambridge Phil. Soc.},
\textbf{25}: 266-271.

\bibitem{KA81}
Kock, A., 1981, {\em Synthetic Differential Geometry}, London Mathematical Society Lecture Note Series, 51, Cambridge: Cambridge University Press. 

\bibitem{KN1}
S. Kobayashi and K. Nomizu : {\em Foundations of Differential Geometry}, Vol I., Wiley Interscience, New York--London 1963.

\bibitem{Krips}
H. Krips : Measurement in Quantum Theory, \emph{The Stanford Encyclopedia of Philosophy } (Winter 1999 Edition), 
Edward N. Zalta (ed.), 

\bibitem{LTY}
Lam, T. Y., 1966, The category of noetherian modules, {\em Proc. Natl. Acad. Sci. USA}, \textbf{55}: 1038-104.

\bibitem{LJ-SPJ86}
Lambek, J. \& Scott, P.J., 1986, {\em Introduction to Higher Order Categorical Logic}, Cambridge: Cambridge University Press. 

\bibitem{LJ68}
Lambek, J., 1968, Deductive Systems and Categories I. Syntactic Calculus and Residuated Categories, 
{\em Mathematical Systems Theory}, 2, 287--318. 

\bibitem{LJ69}
Lambek, J., 1969, {\em Deductive Systems and Categories II. Standard Constructions and Closed Categories, Category Theory, Homology Theory and their Applications I}, Berlin: Springer, 76--122. 

\bibitem{LJ72}
Lambek, J., 1972, {\em Deductive Systems and Categories III. Cartesian Closed Categories, Intuitionistic Propositional Calculus, and Combinatory Logic, Toposes, Algebraic Geometry and Logic}, Lecture Notes in Mathematics, 274, Berlin: Springer, 57--82.  

\bibitem{LT89A}
Lambek, J., 1989A, On Some Connections Between Logic and Category Theory, {\em Studia Logica}, 48, 3, 269--278. 

\bibitem{LJ89B}
Lambek, J., 1989B, On the Sheaf of Possible Worlds, {\em Categorical Topology and its relation to Analysis, Algebra and Combinatorics}, Teaneck: World Scientific Publishing, 36--53. 

\bibitem{LJ94a}
Lambek, J., 1994a, Some Aspects of Categorical Logic, in {\em Logic, Methodology and Philosophy of Science IX, Studies in Logic and the Foundations of Mathematics} \textbf{134}, Amsterdam: North Holland, 69--89. 
 
\bibitem{LJ2k4}
Lambek, J., 2004, What is the world of Mathematics? Provinces of Logic Determined, {\em Annals of Pure and Applied Logic}, \textbf{126}(1-3), 149--158. 

\bibitem{LaSc}
Lambek, J. and P.~J.~Scott. {\em Introduction to higher order categorical logic}. Cambridge University Press, 1986.

\bibitem{Lance}
E. C. Lance: Hilbert C*--Modules. \emph{London Math. Soc. Lect. Notes} \textbf{210}, \emph{Cambridge Univ. Press.} 1995.

\bibitem{LE-MJP2k5}
Landry, E. \& Marquis, J.-P., 2005, Categories in Context: Historical, Foundational and philosophical, 
{\em Philosophia Mathematica}, 13, 1--43.  

\bibitem{LE99}
Landry, E., 1999, Category Theory: the Language of Mathematics, {\em Philosophy of Science}, 66, 3: supplement, S14--S27. 

\bibitem{LandNP98}
Landsman, N. P.: 1998, \emph{Mathematical Topics between Classical and Quantum Mechanics}, Springer Verlag: New York.

\bibitem{Land1}
Landsman, N. P. : Compact quantum groupoids, (at arXiv:math--ph/9912006).

\bibitem{LPRM94}
La Palme Reyes, M., et. al., 1994, The non-Boolean Logic of Natural Language Negation, 
{\em Philosophia Mathematica}, \textbf{2}, no. 1, 45--68.

\bibitem{LPRM99} 
La Palme Reyes, M., et. al., 1999, {\em Count Nouns, Mass Nouns, and their Transformations: a Unified Category-theoretic Semantics, Language, Logic and Concepts}, Cambridge: MIT Press, 427--452.  
 
\bibitem{LFW64}
Lawvere, F. W., 1964, An Elementary Theory of the Category of Sets, {\em Proceedings of the National Academy of Sciences U.S.A.}, 52, 1506--1511. 

\bibitem{LFW65}
Lawvere, F. W., 1965, Algebraic Theories, Algebraic Categories, and Algebraic Functors, {\em Theory of Models}, Amsterdam: North Holland, 413--418.  

\bibitem{LFW66}
Lawvere, F. W.: 1966, The Category of Categories as a Foundation for Mathematics., in
\emph{Proc. Conf. Categorical Algebra- La Jolla}., Eilenberg, S. et al., eds. Springer--Verlag:
Berlin, Heidelberg and New York., pp. 1-20.

\bibitem{LFW69a}
Lawvere, F. W., 1969a, Diagonal Arguments and Cartesian Closed Categories, in {\em Category Theory, Homology Theory, and their Applications II}, Berlin: Springer, 134--145.  

\bibitem{LFW69b}
Lawvere, F. W., 1969b, Adjointness in Foundations, {\em Dialectica}, \textbf{23}: 281--295.  

\bibitem{LFW70}
Lawvere, F. W., 1970, Equality in Hyper doctrines and Comprehension Schema as an Adjoint Functor, 
{\em Applications of Categorical Algebra}, Providence: AMS, 1-14.  

\bibitem{LT271}
Lawvere, F. W., 1971, Quantifiers and Sheaves, {\em Actes du Congr\'es International des Math\'ematiciens}, Tome 1, Paris: Gauthier-Villars, 329--334. 

\bibitem{LFW72}
Lawvere, F. W., 1972, Introduction, in {\em Toposes, Algebraic Geometry and Logic}, Lecture Notes in Mathematics, 274, Springer-Verlag, 1-12.  

\bibitem{LFW75}
Lawvere, F. W., 1975, Continuously Variable Sets: Algebraic Geometry = Geometric Logic, {\em Proceedings of the Logic Colloquium}, Bristol 1973, Amsterdam: North Holland, 135-153. 

\bibitem{LFW76}
Lawvere, F. W., 1976, Variable Quantities and Variable Structures, in {\em Topoi, Algebra, Topology, and Category Theory}, New York: Academic Press, 101--131. 

\bibitem{LFW63}
Lawvere, F. W.: 1963, Functorial Semantics of Algebraic Theories, \emph{Proc. Natl. Acad. Sci. USA, Mathematics}, \textbf{50}: 869-872.

\bibitem{LFW92}
Lawvere, F. W., 1992, Categories of Space and of Quantity, {\em The Space of Mathematics, Foundations of Communication and Cognition}, Berlin: De Gruyter, 14-30.  

\bibitem{LFW94b}
Lawvere, F. W., 1994b, Tools for the Advancement of Objective Logic: Closed Categories and Toposes, 
{\em The Logical Foundations of Cognition}, Vancouver Studies in Cognitive Science, 4, Oxford: Oxford University Press, 43--56.  

\bibitem{LFW2k}
Lawvere, F. W., 2000, Comments on the Development of Topos Theory, {\em Development of Mathematics 1950-2000}, Basel: Birkh\''auser, 715--734. 

\bibitem{LFW2k2}
Lawvere, F. W., 2002, Categorical Algebra for Continuum Micro-Physics, {\em Journal of Pure and Applied Algebra}, 175, no. 1--3, 267--287. 

\bibitem{LFWk3}
Lawvere, F. W., 2003, Foundations and Applications: Axiomatization and Education. New Programs and Open Problems in the Foundation of Mathematics, {\em Bulletin of Symbolic Logic}, 9, 2, 213--224. 

\bibitem{LT2k2}
Leinster, T., 2002, A Survey of Definitions of n-categories, in {Theory and Applications of Categories}, (electronic), \textbf{10}, 1--70. 

\bibitem{LiM-PV97}
Li, M. and P. Vitanyi: 1997, \emph{An introduction to Kolmogorov Complexity and its Applications}, Springer Verlag: New York.

\bibitem{Lofgren68}
L\''{o}fgren,  L.: 1968, An Axiomatic Explanation of Complete Self-Reproduction, \emph{Bulletin of Mathematical Biophysics}, \textbf{30}: 317-348

\bibitem{LS60}
Lubkin, S., 1960. Imbedding of abelian categories.,  {\em Trans. Amer. Math. Soc.}, \textbf{97}: 410-417.


\bibitem{Mack1}
K. C. H. Mackenzie : {\em Lie Groupoids and Lie Algebroids in Differential Geometry}, LMS Lect. Notes \textbf{124}, Cambridge University Press, 1987

\bibitem{MCLSS48}
MacLane, S.: 1948. Groups, categories, and duality., {\em Proc. Natl. Acad. Sci.U.S.A}, \textbf{34}: 263-267.

\bibitem{MCLSS69}
MacLane, S., 1969, Foundations for Categories and Sets, in {\em Category Theory, Homology Theory and their Applications II}, Berlin: Springer, 146--164. 

\bibitem{MCLS71}MacLane, S., 1971, Categorical algebra and Set-Theoretic Foundations, in {\em Axiomatic Set Theory}, Providence: AMS, 231--240. 

\bibitem{MCLS75}
MacLane, S., 1975, Sets, Topoi, and Internal Logic in Categories, {\em Studies in Logic and the Foundations of Mathematics}, 80, Amsterdam: North Holland, 119--134. 

\bibitem{MCLS86}
MacLane, S., 1986, {\em Mathematics, Form and Function}, New York: Springer. 

\bibitem{MCLS88}
MacLane, S., 1988, Concepts and Categories in Perspective, in {\em A Century of Mathematics in America, Part I}, Providence: AMS, 323--365. 

\bibitem{MCLS89}
MacLane, S., 1989, The Development of Mathematical Ideas by Collision: the Case of Categories and Topos Theory, in {\em Categorical Topology and its Relation to Analysis, Algebra and Combinatorics}, Teaneck: World Scientific, 1--9.

\bibitem{MS-IM92}
Maclane, S. and I. Moerdijk : {\em Sheaves in Geometry and Logic -- A first Introduction to Topos Theory}, Springer Verlag, New York
1992. 

\bibitem{MLS50}
MacLane, S., 1950, Dualities for Groups, {\em Bulletin of the American Mathematical Society}, 56, 485-516. 

\bibitem{MCLS96} 
MacLane, S., 1996, Structure in Mathematics. Mathematical Structuralism., {\em Philosophia Mathematica}, 4, 2, 174-183. 

\bibitem{MCLS98}
MacLane, S., 1997, Categories for the Working Mathematician, 2nd edition, New York: Springer-Verlag. 

\bibitem{Majid1}
Majid, S.: 1995, \emph{Foundations of Quantum Group Theory}, Cambridge Univ. Press: Cambridge, UK.

\bibitem{Majid2}
Majid, S.: 2002, \emph{A Quantum Groups Primer}, Cambridge Univ.Press: Cambridge, UK.

\bibitem{MM-RG95} 
Makkai, M. \& Par\'e, R., 1989, {\em Accessible Categories: the Foundations of Categorical Model Theory, Contemporary Mathematics}, 104, Providence: AMS. 

\bibitem{MM99}
Makkai, M., 1999, On Structuralism in Mathematics, in {\em Language, Logic and Concepts}, Cambridge: MIT Press, 43--66. 

\bibitem{MM-RG77}
Makkai, M. \& Reyes, G., 1977, {\em First-Order Categorical Logic}, Springer Lecture Notes in Mathematics 611, New York: Springer. 

\bibitem{MM-RG95}
Makkei, M. \& Reyes, G., 1995, Completeness Results for Intuitionistic and Modal Logic in a Categorical Setting, 
{\em Annals of Pure and Applied Logic}, 72, 1, 25--101. 

\bibitem{MJP93}
Marquis, J.-P., 1993, Russell's Logicism and Categorical Logicisms, in {\em Russell and Analytic Philosophy}, A. D. Irvine \& G. A. Wedekind, (eds.), Toronto, University of Toronto Press, 293--324.

\bibitem{MJP95}
Marquis, J.-P., 1995, Category Theory and the Foundations of Mathematics: Philosophical Excavations., 
{\em Synthese}, 103, 421--447. 

\bibitem{MJP2k6} 
Marquis, J.-P., 2006, Categories, Sets and the Nature of Mathematical Entities, in {\em The Age of Alternative Logics. Assessing philosophy of logic and mathematics today}, J. van Benthem, G. Heinzmann, Ph. Nabonnand, M. Rebuschi, H.Visser, eds., Springer,181-192. 

\bibitem{MJP1999}
May, J.P. 1999, \emph{A Concise Course in Algebraic Topology}, The University of Chicago Press: Chicago.

\bibitem{MCWP43}
McCulloch, W. and W. Pitt.: 1943, A logical Calculus of Ideas Immanent in Nervous Activity., \emph{Bull. Math. Biophysics,} \textbf{5}: 115-133.

\bibitem{MLC86}
Mc Larty, C., 1986, Left Exact Logic, {\em Journal of Pure and Applied Algebra}, 41, no. 1, 63-66.

\bibitem{MLC91}
Mc Larty, C., 1991, Axiomatizing a Category of Categories, {\em Journal of Symbolic Logic}, 56, no. 4, 1243-1260. 

\bibitem{MLC92} 
Mc Larty, C., 1992, {\em Elementary Categories, Elementary Toposes}, Oxford: Oxford University Press.

\bibitem{MLC94}
Mc Larty, C., 1994, Category Theory in Real Time, {\em Philosophia Mathematica}, \textbf{2}, no. 1, 36-44.

\bibitem{MLC2k5}
Mc Larty, C., 2005, Learning from Questions on Categorical Foundations, {\em Philosophia Mathematica}, \textbf{13}, 1, 44--60.

\bibitem{Mitchell1}
Mitchell, B.: 1965, \emph{Theory of Categories}, Academic Press:London.

\bibitem{Mitchell2}
Mitchell, B.: 1964, The full imbedding theorem. \emph{Amer. J. Math}. \textbf{86}: 619-637.

\bibitem{MI-P2k2}
Moerdijk, I. \& Palmgren, E., 2002, Type Theories, Toposes and Constructive Set Theory: Predicative Aspects of AST., Annals of Pure and Applied Logic, 114, no. 1--3, 155--201. 

\bibitem{MO98}
Moerdijk, I., 1998, Sets, Topoi and Intuitionism., Philosophia Mathematica, 6, no. 2, 169-177.

\bibitem{Moer1}
I. Moerdijk : Classifying toposes and foliations, {\it Ann. Inst. Fourier, Grenoble} \textbf{41}, 1 (1991) 189-209.

\bibitem{Moer2}
I. Moerdijk : Introduction to the language of stacks and gerbes, (preprint at arXiv:math.AT/0212266) (2002).

\bibitem{MK62}
Morita, K. 1962. Category isomorphism and endomorphism rings of modules,
{\em Trans. Amer. Math. Soc.}, \textbf{103}: 451-469.

\bibitem{MK70}
Morita, K. , 1970. Localization in categories of modules. I., {\em Math. Z.}, 
\textbf{114}: 121-144.

\bibitem{Mostow}
M. A. Mostow : The differentiable space structure of Milnor classifying spaces, simplicial complexes, and geometric
realizations, \emph{J. Diff. Geom.} \textbf{14} (1979) 255-293.

\bibitem{OB69}
Oberst, U.: 1969, Duality theory for Grothendieck categories., \emph{Bull. Amer. Math. Soc.} \textbf{75}: 1401-1408.

\bibitem{ORT70}
Oort, F.: 1970. On the definition of an abelian category. \emph{Proc. Roy. Neth. Acad. Sci}. \textbf{70}: 13-02.

\bibitem{OO31}
Ore, O., 1931, Linear equations on non-commutative fields, {\em Ann. Math.} \textbf{32}: 463-477.

\bibitem{PLR}
Plymen, R.J. and P. L. Robinson: 1994,  \emph{Spinors in Hilbert Space}, Cambridge Tracts in Math. 
\textbf{114}, Cambridge Univ. Press, Cambridge.

\bibitem{NPop1}
Popescu, N.: 1973, \emph{Abelian Categories with Applications to Rings and Modules.} New York and London: Academic Press., 2nd edn. 1975. \emph{(English translation by I.C. Baianu)}.

\bibitem{PB70}
Pareigis, B., 1970, Categories and Functors, New York: Academic Press. 

\bibitem{PMC2k4}
Pedicchio, M. C. and Tholen, W., 2004, Categorical Foundations, Cambridge: Cambridge University Press. 

\bibitem{PB91}
Peirce, B., 1991, Basic Category Theory for Computer Scientists, Cambridge: MIT Press. 

\bibitem{PAM90}
Pitts, A. M., 1989, Conceptual Completeness for First-order Intuitionistic Logic: an Application of Categorical Logic,
{\em Annals of Pure and Applied Logic}, 41, no. 1, 33--81. 

\bibitem{PAM2k}
Pitts, A. M., 2000, {\em Categorical Logic}, in {\em Handbook of Logic in Computer Science}, Vol.5, Oxford: Oxford Unversity Press, 39-128.

\bibitem{PB2k} 
Plotkin, B., 2000, Algebra, Categories and Databases, in {\em Handbook of Algebra}, Vol. 2, Amsterdam: Elsevier, 79--148. 

\bibitem{Pradines1966}
Pradines, J.: 1966, Th\'eorie de Lie pour les groupoides diff\'erentiable, relation entre propri\'etes locales et globales, \emph{C. R. Acad Sci. Paris S\'er. A} \textbf{268}: 907-910.

\bibitem{Raptis1-2k3}
Raptis, I.: 2003, Algebraic quantisation of causal sets, \emph{Int. Jour. Theor. Phys.} \textbf{39}: 1233.

\bibitem{RGZH91}
Reyes, G. \& Zolfaghari, H., 1991, Topos-theoretic Approaches to Modality,in {\em Category Theory (Como 1990), Lecture Notes in Mathematics}, 1488, Berlin: Springer, 359--378. 

\bibitem{RGZH96}
Reyes, G. \& Zolfaghari, H., 1996, Bi-Heyting Algebras, Toposes and Modalities, in {\em Journal of Philosophical Logic}, 25, no. 1, 25-43. 

\bibitem{RG74}
Reyes, G., 1974, From Sheaves to Logic, in {\em Studies in Algebraic Logic}, A. Daigneault, ed., Providence: AMS. 

\bibitem{Rieffel}
M. A. Rieffel : Group C*-algebras as compact quantum metric spaces, \emph{Documenta Math.} \textbf{7} (2002), 605-651.

\bibitem{Roberts}
Roberts, J. E.: 2004, More lectures on algebraic quantum field theory, in A. Connes, et al. \emph{Noncommutative Geometry}, Springer: Berlin and New York.

\bibitem{RSE-KEP94} 
Rodabaugh, S. E. \& Klement, E. P., eds., Topological and Algebraic Structures in Fuzzy Sets: A Handbook of Recent Developments in the Mathematics of Fuzzy Sets, Trends in Logic, 20, Dordrecht: Kluwer. 


\bibitem{RRosen2}
Rosen, R.: 1958b, The Representation of Biological Systems from the Standpoint of the Theory of Categories., \emph{ Bulletin of Mathematical Biophysics} \textbf{20}: 317-341.

\bibitem{Rota} G. C. Rota : On the foundation of combinatorial theory, I. The theory of M\"obius functions, \emph{Zetschrif f\''ur Wahrscheinlichkeitstheorie} \textbf{2} (1968), 340.

\bibitem{SPJ2k} 
Scott, P. J., 2000, {\em Some Aspects of Categories in Computer Science}, Handbook of Algebra, Vol. 2, Amsterdam: North Holland, 3--77. 

\bibitem{SHS2k5}
Shapiro, S., 2005, Categories, Structures and the Frege-Hilbert Controversy: the Status of Metamathematics, 
{\em Philosophia Mathematica}, 13, 1, 61--77.

\bibitem{Sorkin}
Sorkin, R.D.: 1991, Finitary substitute for continuous topology, \emph{Int. J. Theor. Phys.} \textbf{30} No. 7.: 923--947.

\bibitem{Spanier}
Spanier, E. H.: 1966, \emph{Algebraic Topology}, McGraw Hill: New York.

\bibitem{Szabo}
Szabo, R. J.: 2003, Quantum field theory on non-commutative spaces, \emph{Phys. Rep.} \textbf{378}: 207--209.

\bibitem{Thom80}
Thom, R.: 1980, \emph{Mod\`eles math\'ematiques de la morphog\'en\`ese}, Paris, Bourgeois.

\bibitem{TP96} 
Taylor, P., 1996, Intuitionistic sets and Ordinals, {\em Journal of Symbolic Logic}, 61, 705--744.
 
\bibitem{TP99}
Taylor, P., 1999, Practical Foundations of Mathematics, Cambridge: Cambridge University Press. 

\bibitem{TP72}
Tierney, M., 1972, Sheaf Theory and the Continuum Hypothesis, {\em Toposes, Algebraic Geometry and Logic}.

\bibitem{VdHG-MI84a}
Van der Hoeven, G. and  Moerdijk, I., 1984a, Sheaf Models for Choice Sequences, 
{\em Annals of Pure and Applied Logic}, 27, no. 1, 63--107. 

\bibitem{Varilly}
V\'arilly, J. C.: 1997, An introduction to noncommutative geometry. (at arXiv:physics/9709045)

\bibitem{Weinstein96}
Weinstein, A.: 1996, Groupoids : unifying internal and external symmetry, \emph{Notices of the Amer. Math. Soc.} \textbf{43}: 744--752.

\bibitem{WB83}
Wess J. and J. Bagger: 1983, \emph{Supersymmetry and Supergravity}, Princeton University Press: Princeton, NJ.

\bibitem{Whitehead1941}
Whitehead, J. H. C.: 1941, On adding relations to homotopy groups, \emph{Annals of Math.} \textbf{42} (2): 409--428.

\bibitem{WRJ2k4}
Wood, R.J., 2004, Ordered Sets via Adjunctions, in {\em Categorical Foundations}, M. C. Pedicchio and W. Tholen, eds., Cambridge: Cambridge University Press. 

\end{thebibliography}
%%%%%
%%%%%
\end{document}
