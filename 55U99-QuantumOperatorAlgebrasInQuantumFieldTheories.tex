\documentclass[12pt]{article}
\usepackage{pmmeta}
\pmcanonicalname{QuantumOperatorAlgebrasInQuantumFieldTheories}
\pmcreated{2013-03-22 18:10:46}
\pmmodified{2013-03-22 18:10:46}
\pmowner{bci1}{20947}
\pmmodifier{bci1}{20947}
\pmtitle{quantum operator algebras in quantum field theories}
\pmrecord{66}{40752}
\pmprivacy{1}
\pmauthor{bci1}{20947}
\pmtype{Topic}
\pmcomment{trigger rebuild}
\pmclassification{msc}{55U99}
\pmclassification{msc}{81T99}
\pmclassification{msc}{81T80}
\pmclassification{msc}{81T75}
\pmclassification{msc}{81T05}
\pmsynonym{QOA}{QuantumOperatorAlgebrasInQuantumFieldTheories}
\pmsynonym{QAT}{QuantumOperatorAlgebrasInQuantumFieldTheories}
\pmsynonym{QFT}{QuantumOperatorAlgebrasInQuantumFieldTheories}
\pmsynonym{quantum theories}{QuantumOperatorAlgebrasInQuantumFieldTheories}
%\pmkeywords{fundamentals of quantum theories}
%\pmkeywords{quantum operator algebras}
%\pmkeywords{von Neumann algebras}
%\pmkeywords{C*-algebras}
\pmrelated{CAlgebra3}
\pmrelated{Groupoids}
\pmrelated{HamiltonianOperator}
\pmrelated{QuantumChromodynamicsQCD}
\pmrelated{QEDInTheoreticalAndMathematicalPhysics}
\pmrelated{NuclearCAlgebra}
\pmrelated{QuantumSuperOperators}
\pmrelated{QuantumLogicsTopoi}
\pmrelated{SchrodingerOperator}
\pmrelated{Distribution4}
\pmrelated{QuantumChromodynamicsQCD}
\pmrelated{CompactQuantumGroup}
\pmrelated{Equiv}
\pmdefines{bialgebra}
\pmdefines{antihomomorphism}
\pmdefines{QOA}
\pmdefines{groupoids}
\pmdefines{von Neumann algebra}
\pmdefines{Hopf algebra}
\pmdefines{Haar systems for locally compact topological groupoids}

\endmetadata

% this is the default PlanetMath preamble.  as your knowledge
% of TeX increases, you will probably want to edit this, but
% it should be fine as is for beginners.

% almost certainly you want these
\usepackage{amssymb}
\usepackage{amsmath}
\usepackage{amsfonts}

% used for TeXing text within eps files
%\usepackage{psfrag}
% need this for including graphics (\includegraphics)
%\usepackage{graphicx}
% for neatly defining theorems and propositions
%\usepackage{amsthm}
% making logically defined graphics
%%%\usepackage{xypic}

% there are many more packages, add them here as you need them

% define commands here
\usepackage{amsmath, amssymb, amsfonts, amsthm, amscd, latexsym, enumerate}
\usepackage{xypic, xspace}
\usepackage[mathscr]{eucal}
\usepackage[dvips]{graphicx}
\usepackage[curve]{xy}

\setlength{\textwidth}{6.5in}
%\setlength{\textwidth}{16cm}
\setlength{\textheight}{9.0in}
%\setlength{\textheight}{24cm}

\hoffset=-.75in     %%ps format
%\hoffset=-1.0in     %%hp format
\voffset=-.4in


\theoremstyle{plain}
\newtheorem{lemma}{Lemma}[section]
\newtheorem{proposition}{Proposition}[section]
\newtheorem{theorem}{Theorem}[section]
\newtheorem{corollary}{Corollary}[section]

\theoremstyle{definition}
\newtheorem{definition}{Definition}[section]
\newtheorem{example}{Example}[section]
%\theoremstyle{remark}
\newtheorem{remark}{Remark}[section]
\newtheorem*{notation}{Notation}
\newtheorem*{claim}{Claim}

\renewcommand{\thefootnote}{\ensuremath{\fnsymbol{footnote}}}
\numberwithin{equation}{section}

\newcommand{\Ad}{{\rm Ad}}
\newcommand{\Aut}{{\rm Aut}}
\newcommand{\Cl}{{\rm Cl}}
\newcommand{\Co}{{\rm Co}}
\newcommand{\DES}{{\rm DES}}
\newcommand{\Diff}{{\rm Diff}}
\newcommand{\Dom}{{\rm Dom}}
\newcommand{\Hol}{{\rm Hol}}
\newcommand{\Mon}{{\rm Mon}}
\newcommand{\Hom}{{\rm Hom}}
\newcommand{\Ker}{{\rm Ker}}
\newcommand{\Ind}{{\rm Ind}}
\newcommand{\IM}{{\rm Im}}
\newcommand{\Is}{{\rm Is}}
\newcommand{\ID}{{\rm id}}
\newcommand{\grpL}{{\rm GL}}
\newcommand{\Iso}{{\rm Iso}}
\newcommand{\rO}{{\rm O}}
\newcommand{\Sem}{{\rm Sem}}
\newcommand{\SL}{{\rm Sl}}
\newcommand{\St}{{\rm St}}
\newcommand{\Sym}{{\rm Sym}}
\newcommand{\Symb}{{\rm Symb}}
\newcommand{\SU}{{\rm SU}}
\newcommand{\Tor}{{\rm Tor}}
\newcommand{\U}{{\rm U}}

\newcommand{\A}{\mathcal A}
\newcommand{\Ce}{\mathcal C}
\newcommand{\D}{\mathcal D}
\newcommand{\E}{\mathcal E}
\newcommand{\F}{\mathcal F}
%\newcommand{\grp}{\mathcal G}
\renewcommand{\H}{\mathcal H}
\renewcommand{\cL}{\mathcal L}
\newcommand{\Q}{\mathcal Q}
\newcommand{\R}{\mathcal R}
\newcommand{\cS}{\mathcal S}
\newcommand{\cU}{\mathcal U}
\newcommand{\W}{\mathcal W}

\newcommand{\bA}{\mathbb{A}}
\newcommand{\bB}{\mathbb{B}}
\newcommand{\bC}{\mathbb{C}}
\newcommand{\bD}{\mathbb{D}}
\newcommand{\bE}{\mathbb{E}}
\newcommand{\bF}{\mathbb{F}}
\newcommand{\bG}{\mathbb{G}}
\newcommand{\bK}{\mathbb{K}}
\newcommand{\bM}{\mathbb{M}}
\newcommand{\bN}{\mathbb{N}}
\newcommand{\bO}{\mathbb{O}}
\newcommand{\bP}{\mathbb{P}}
\newcommand{\bR}{\mathbb{R}}
\newcommand{\bV}{\mathbb{V}}
\newcommand{\bZ}{\mathbb{Z}}

\newcommand{\bfE}{\mathbf{E}}
\newcommand{\bfX}{\mathbf{X}}
\newcommand{\bfY}{\mathbf{Y}}
\newcommand{\bfZ}{\mathbf{Z}}

\renewcommand{\O}{\Omega}
\renewcommand{\o}{\omega}
\newcommand{\vp}{\varphi}
\newcommand{\vep}{\varepsilon}

\newcommand{\diag}{{\rm diag}}
\newcommand{\grp}{{\mathsf{G}}}
\newcommand{\dgrp}{{\mathsf{D}}}
\newcommand{\desp}{{\mathsf{D}^{\rm{es}}}}
\newcommand{\grpeod}{{\rm Geod}}
%\newcommand{\grpeod}{{\rm geod}}
\newcommand{\hgr}{{\mathsf{H}}}
\newcommand{\mgr}{{\mathsf{M}}}
\newcommand{\ob}{{\rm Ob}}
\newcommand{\obg}{{\rm Ob(\mathsf{G)}}}
\newcommand{\obgp}{{\rm Ob(\mathsf{G}')}}
\newcommand{\obh}{{\rm Ob(\mathsf{H})}}
\newcommand{\Osmooth}{{\Omega^{\infty}(X,*)}}
\newcommand{\grphomotop}{{\rho_2^{\square}}}
\newcommand{\grpcalp}{{\mathsf{G}(\mathcal P)}}

\newcommand{\rf}{{R_{\mathcal F}}}
\newcommand{\grplob}{{\rm glob}}
\newcommand{\loc}{{\rm loc}}
\newcommand{\TOP}{{\rm TOP}}

\newcommand{\wti}{\widetilde}
\newcommand{\what}{\widehat}

\renewcommand{\a}{\alpha}
\newcommand{\be}{\beta}
\newcommand{\grpa}{\grpamma}
%\newcommand{\grpa}{\grpamma}
\newcommand{\de}{\delta}
\newcommand{\del}{\partial}
\newcommand{\ka}{\kappa}
\newcommand{\si}{\sigma}
\newcommand{\ta}{\tau}

\newcommand{\med}{\medbreak}
\newcommand{\medn}{\medbreak \noindent}
\newcommand{\bign}{\bigbreak \noindent}

\newcommand{\lra}{{\longrightarrow}}
\newcommand{\ra}{{\rightarrow}}
\newcommand{\rat}{{\rightarrowtail}}
\newcommand{\ovset}[1]{\overset {#1}{\ra}}
\newcommand{\ovsetl}[1]{\overset {#1}{\lra}}
\newcommand{\hr}{{\hookrightarrow}}

\newcommand{\<}{{\langle}}

%\newcommand{\>}{{\rangle}}
%\usepackage{geometry, amsmath,amssymb,latexsym,enumerate}
%%%\usepackage{xypic}

\def\baselinestretch{1.1}


\hyphenation{prod-ucts}

%\grpeometry{textwidth= 16 cm, textheight=21 cm}

\newcommand{\sqdiagram}[9]{$$ \diagram  #1  \rto^{#2} \dto_{#4}&
#3  \dto^{#5} \\ #6    \rto_{#7}  &  #8   \enddiagram
\eqno{\mbox{#9}}$$ }

\def\C{C^{\ast}}

\newcommand{\labto}[1]{\stackrel{#1}{\longrightarrow}}

%\newenvironment{proof}{\noindent {\bf Proof} }{ \hfill $\Box$
%{\mbox{}}

\newcommand{\quadr}[4]
{\begin{pmatrix} & #1& \\[-1.1ex] #2 & & #3\\[-1.1ex]& #4&
 \end{pmatrix}}
\def\D{\mathsf{D}}
\begin{document}
\subsection{Introduction}
This is a topic entry that introduces quantum operator algebras and presents concisely the important
roles they play in quantum field theories.

\begin{definition} {\em Quantum operator algebras} (QOA) in quantum field theories are
defined as the algebras of observable operators, and as such, they are also related to the von Neumann algebra; 
quantum operators are usually defined on Hilbert spaces, or in some QFTs on Hilbert space bundles
or other similar families of spaces. 
\end{definition}

\begin{remark}
 Representations of Banach $*$-algebras (that are defined on Hilbert spaces) are closely related
to C* -algebra representations which provide a useful approach to defining quantum space-times. 
\end{remark}

\subsection{Quantum operator algebras in quantum field theories: QOA Role in QFTs}

 Important examples of quantum operators are: the Hamiltonian operator (or Schr\"odinger operator), the position and momentum operators, Casimir operators, unitary operators and spin operators. The observable operators are also {\em self-adjoint}. More general operators were recently defined, such as Prigogine's superoperators. 

 Another development in quantum theories was the introduction of Frech\'et nuclear spaces or `rigged' Hilbert spaces (Hilbert {\em bundles}). The following sections define several types of quantum operator algebras that provide the foundation of modern quantum field theories in mathematical physics. 


\subsubsection{Quantum groups; quantum operator algebras and related symmetries.}

 Quantum theories adopted a new lease of life post 1955 when von Neumann beautifully re-formulated quantum mechanics (QM) and quantum theories (QT) in the mathematically rigorous context of Hilbert spaces and operator
algebras defined over such spaces. From a current physics perspective, 
von Neumann' s approach to quantum mechanics has however done much more: it has
not only paved the way to expanding the role of symmetry in physics, as for example with the Wigner-Eckhart theorem and its applications, but also revealed the fundamental importance in quantum physics of the state space geometry of quantum operator algebras.


\subsection{Basic mathematical definitions in QOA: }

\begin{itemize} 
\item {\em Von Neumann algebra}

\item {\em Hopf algebra}

\item {\em Groupoids}

\item {\em Haar systems associated to measured groupoids or locally compact groupoids.}
\item C*-algebras and quantum groupoids entry (attached). 
\end{itemize}

\subsubsection{Von Neumann algebra}

 Let $\H$ denote a complex (separable) Hilbert space. A \emph{von Neumann algebra} $\A$ acting on $\H$ is a subset of the algebra of all bounded operators $\cL(\H)$ such that:

\begin{itemize}

\item  (i) $\A$ is closed under the adjoint operation (with the
adjoint of an element $T$ denoted by $T^*$).

\item  (ii) $\A$ equals its bicommutant, namely:

\begin{equation}
\A= \{A \in \cL(\H) : \forall B \in \cL(\H), \forall C\in \A,~
(BC=CB)\Rightarrow (AB=BA)\}~.
\end{equation}
\end{itemize}

 If one calls a \emph{commutant} of a set $\A$ the special set of
bounded operators on $\cL(\H)$ which commute with all elements in
$\A$, then this second condition implies that the commutant of the
commutant of $\A$ is again the set $\A$.

 On the other hand, a von Neumann algebra $\A$ inherits a
\emph{unital} subalgebra from $\cL(\H)$, and according to the
first condition in its definition $\A$, it does indeed inherit a
$*$-subalgebra structure as further explained in the next
section on C* -algebras. Furthermore, one also has available a notable
\emph{`bicommutant theorem'} which states that: ``{\em $\A$ is a von
Neumann algebra if and only if $\A$ is a $*$-subalgebra of
$\cL(\H)$, closed for the smallest topology defined by continuous
maps $(\xi,\eta)\longmapsto (A\xi,\eta)$ for all $<A\xi,\eta)>$
where $<.,.>$ denotes the inner product defined on $\H$}~''. 

 For a well-presented treatment of the geometry of the state spaces of quantum operator algebras, 
the reader is referred to Aflsen and Schultz (2003; \cite{AS2k3}).


\subsubsection{Hopf algebra}
 First, a unital associative algebra consists of a linear space
$A$ together with two linear maps:

\begin{equation}
\begin{aligned} m &: A \otimes A \lra A~,~(multiplication) \\
\eta &: \bC \lra A~,~ (unity)
\end{aligned}
\end{equation}
satisfying the conditions
\begin{equation}
\begin{aligned}
m(m \otimes \mathbf 1) &= m (\mathbf 1 \otimes m)  \\  m(\mathbf 1
\otimes \eta) &= m (\eta \otimes \mathbf 1) = \ID~.
\end{aligned}
\end{equation}
This first condition can be seen in terms of a commuting diagram~:
\begin{equation}
\begin{CD}
A \otimes A \otimes A @> m \otimes \ID>> A \otimes A
\\ @V \ID \otimes mVV   @VV m V
 \\ A \otimes A  @ > m >> A
\end{CD}
\end{equation}

 Next suppose we consider `reversing the arrows', and take an
algebra $A$ equipped with a linear homorphisms $\Delta : A \lra A
\otimes A$, satisfying, for $a,b \in A$ :

\begin{equation}
\begin{aligned} \Delta(ab) &= \Delta(a) \Delta(b)
\\ (\Delta \otimes \ID) \Delta &= (\ID \otimes \Delta) \Delta~.
\end{aligned}
\end{equation}

  We call $\Delta$ a \emph{comultiplication}, which is said to be
\emph{coasociative} in so far that the following diagram commutes
\begin{equation}
\begin{CD}
A \otimes A \otimes A @< \Delta\otimes \ID<< A \otimes A
\\ @A \ID \otimes \Delta AA  @AA \Delta A
 \\ A \otimes A  @ < \Delta << A
\end{CD}
\end{equation}

 There is also a counterpart to $\eta$, the \emph{counity} map
$\vep : A \lra \bC$ satisfying
\begin{equation}
(\ID \otimes \vep) \circ \Delta = (\vep \otimes \ID) \circ \Delta
= \ID~.
\end{equation}
  
 A \emph{bialgebra} $(A, m, \Delta, \eta,\vep)$ is a linear space $A$ with maps $m, \Delta, \eta, \vep$
satisfying the above properties.

  Now to recover anything resembling a group structure, we must
append such a bialgebra with an antihomomorphism $S : A \lra A$,
satisfying $S(ab) = S(b) S(a)$, for $a,b \in A$~. This map is
defined implicitly via the property~:
\begin{equation} m(S \otimes
\ID) \circ \Delta = m(\ID \otimes S) \circ \Delta = \eta \circ
\vep~~.
\end{equation}

 We call $S$ the \emph{antipode map}. 

 A \emph{Hopf algebra} is then a bialgebra $(A,m, \eta, \Delta, \vep)$ equipped with an antipode
map $S$~.

 Commutative and non-commutative Hopf algebras form the backbone of
quantum `groups' and are essential to the generalizations of
symmetry. Indeed, in most respects a quantum `group' is closely related to its dual
Hopf algebra; in the case of a finite, commutative quantum group its dual Hopf algebra is 
obtained via Fourier transformation of the group elements.  When Hopf algebras are actually associated with their dual, proper groups of matrices, there is considerable scope for their representations on both finite and infinite dimensional Hilbert spaces.


\subsubsection{Groupoids}

 Recall that a \emph{groupoid} $\grp$ is, loosely speaking, a small
category with inverses over its set of objects $X = Ob(\grp)$~. One
often writes $\grp^y_x$ for the set of morphisms in $\grp$ from
$x$ to $y$~. \emph{A topological groupoid} consists of a space
$\grp$, a distinguished subspace $\grp^{(0)} = \obg \subset \grp$,
called {\it the space of objects} of $\grp$, together with maps
\begin{equation}
r,s~:~ \xymatrix{ \grp \ar@<1ex>[r]^r \ar[r]_s & \grp^{(0)} }
\end{equation}
called the {\it range} and {\it source maps} respectively,
together with a law of composition
\begin{equation}
\circ~:~ \grp^{(2)}: = \grp \times_{\grp^{(0)}} \grp = \{
~(\gamma_1, \gamma_2) \in \grp \times \grp ~:~ s(\gamma_1) =
r(\gamma_2)~ \}~ \lra ~\grp~,
\end{equation}
such that the following hold~:~
\begin{enumerate}
\item[(1)]
$s(\gamma_1 \circ \gamma_2) = r(\gamma_2)~,~ r(\gamma_1 \circ
\gamma_2) = r(\gamma_1)$~, for all $(\gamma_1, \gamma_2) \in
\grp^{(2)}$~.

\item[(2)]
$s(x) = r(x) = x$~, for all $x \in \grp^{(0)}$~.

\item[(3)]
$\gamma \circ s(\gamma) = \gamma~,~ r(\gamma) \circ \gamma =
\gamma$~, for all $\gamma \in \grp$~.

\item[(4)]
$(\gamma_1 \circ \gamma_2) \circ \gamma_3 = \gamma_1 \circ
(\gamma_2 \circ \gamma_3)$~.

\item[(5)]
Each $\gamma$ has a two--sided inverse $\gamma^{-1}$ with $\gamma
\gamma^{-1} = r(\gamma)~,~ \gamma^{-1} \gamma = s (\gamma)$~.
Furthermore, only for topological groupoids the inverse map needs be continuous.
It is usual to call $\grp^{(0)} = Ob(\grp)$ {\it the set of objects}
of $\grp$~. For $u \in Ob(\grp)$, the set of arrows $u \lra u$ forms a
group $\grp_u$, called the \emph{isotropy group of $\grp$ at $u$}.
\end{enumerate}

  Thus, as it is well kown, a topological groupoid is just a groupoid internal to the category of topological spaces and continuous maps. The notion of internal groupoid has proved significant in a number of fields, since groupoids generalise bundles of groups, group actions, and equivalence relations. For a further study of groupoids we refer the reader to Brown (2006).


  Several examples of groupoids are:
\begin{itemize}
\item (a) locally compact groups,  transformation groups , and any group in general (e.g. [59]
\item (b) equivalence relations
\item (c) tangent bundles
\item (d) the tangent groupoid 
\item (e) holonomy groupoids for foliations 
\item (f) Poisson groupoids 
\item (g) graph groupoids.
\end{itemize}

 As a simple, helpful example of a groupoid, consider (b) above. Thus, let \textit{R} be an \textit{equivalence relation} on a set X. Then \textit{R} is a groupoid under the following operations:
$(x, y)(y, z) = (x, z), (x, y)^{-1} = (y, x)$. Here, $\grp^0 = X $, (the diagonal of $X \times X$ ) and $r((x, y)) = x,  s((x, y)) = y$.

 Therefore, $ R^2$ = $\left\{((x, y), (y, z)) : (x, y), (y, z) \in R \right\} $.
When $R = X \times X $,  \textit{R} is called a \textit{trivial} groupoid. A special case of a trivial groupoid is
$R = R_n = \left\{ 1, 2, . . . , n \right\}$  $\times $ $\left\{ 1, 2, . . . , n \right\} $. (So every \textit{i} is equivalent to every \textit{j}). Identify $(i,j) \in R_n$ with the matrix unit $e_{ij}$. Then the groupoid $R_n$ is just matrix multiplication except that we only multiply $e_{ij}, e_{kl}$ when $k = j$, and $(e_{ij} )^{-1} = e_{ji}$. We do not really lose anything by restricting the multiplication, since the pairs $e_{ij}, {e_{kl}}$ excluded from groupoid multiplication just give the 0 product in normal algebra anyway.
For a groupoid $\grp_{lc}$ to be a locally compact groupoid means that $\grp_{lc}$ is required to be a (second countable) locally compact Hausdorff space, and the product and also inversion maps are required to be continuous. Each $\grp_{lc}^u$ as well as the unit space $\grp_{lc}^0$ is closed in $\grp_{lc}$. What replaces the left Haar measure on $\grp_{lc}$ is a system of measures $\lambda^u$ ($u \in \grp_{lc}^0$), where $\lambda^u$ is a positive regular Borel measure on $\grp_{lc}^u$ with dense support. In addition, the $\lambda^u~$ 's are required to vary continuously (when integrated against $f \in C_c(\grp_{lc}))$ and to form an invariant family in the sense that for each x, the map $y \mapsto xy$ is a measure preserving homeomorphism from $\grp_{lc}^s(x)$ onto $\grp_{lc}^r(x)$. Such a system $\left\{ \lambda^u \right\}$ is called a \textit{left Haar system} for the locally compact groupoid $\grp_{lc}$.

 This is defined more precisely in the next subsection next.

\subsubsection{Haar systems for locally compact topological groupoids}

 Let
\begin{equation}
\xymatrix{ \grp \ar@<1ex>[r]^r \ar[r]_s & \grp^{(0)}}=X
\end{equation}
be a locally compact, locally trivial topological groupoid with
its transposition into transitive (connected) components. Recall
that for $x \in X$, the \emph{costar of $x$} denoted
$\rm{CO}^*(x)$ is defined as the closed set $\bigcup\{ \grp(y,x) :
y \in \grp \}$, whereby
\begin{equation}
\grp(x_0, y_0) \hookrightarrow \rm{CO}^*(x) \lra X~,
\end{equation}
is a principal $\grp(x_0, y_0)$--bundle relative to
fixed base points $(x_0, y_0)$~. Assuming all relevant sets are
locally compact, then following Seda (1976), a \emph{(left) Haar
system on $\grp$} denoted $(\grp, \tau)$ (for later purposes), is
defined to comprise of i) a measure $\kappa$ on $\grp$, ii) a
measure $\mu$ on $X$ and iii) a measure $\mu_x$ on $\rm{CO}^*(x)$
such that for every Baire set $E$ of $\grp$, the following hold on
setting $E_x = E \cap \rm{CO}^*(x)$~:
\begin{itemize}
\item[(1)] $x \mapsto \mu_x(E_x)$ is measurable.

\med
\item[(2)]
$\kappa(E) = \int_x \mu_x(E_x)~d\mu_x$ ~.

\med
\item[(3)]
$\mu_z(t E_x) = \mu_x(E_x)$, for all $t \in \grp(x,z)$ and $x, z
\in \grp$~.
\end{itemize}
\med

The presence of a left Haar system on $\grp_{lc}$ has important
topological implications: it requires that the range map $r :
\grp_{lc} \rightarrow \grp_{lc}^0$ is open. For such a $\grp_{lc}$
with a left Haar system, the vector space $C_c(\grp_{lc})$ is a
\textit{convolution} \textit{*--algebra}, where for $f, g \in
C_c(\grp_{lc})$: \\
\med
$f * g(x) = \int f(t)g(t^{-1} x) d \lambda^{r(x)} (t)$,  with
f*(x) $ = \overline{f(x^{-1})}$.
\med
One has $C^*(\grp_{lc})$ to be the \textit{enveloping C*--algebra}
of $C_c(\grp_{lc})$ (and also representations are required to be
continuous in the inductive limit topology). Equivalently, it is
the completion of $\pi_{univ}(C_c(\grp_{lc}))$ where $\pi_{univ}$
is the \textit{universal representation} of $\grp_{lc}$. For
example, if $ \grp_{lc} = R_n$ , then $C^*(\grp_{lc})$ is just the
finite dimensional algebra $C_c(\grp_{lc}) = M_n$, the span of the
$e_{ij}$'s.


There exists (e.g.[63, p.91]) a \textit{measurable Hilbert bundle}
$(\grp_{lc}^0, \H, \mu)$ with $\H  = \left\{ \H^u_{u \in
\grp_{lc}^0} \right\}$ and a G-representation L on $\H$.  Then,
for every pair $\xi, \eta$ of square integrable sections of $\H$,
it is required that the function $x \mapsto (L(x)\xi (s(x)), \eta
(r(x)))$ be $\nu$--measurable. The representation $\Phi$ of
$C_c(\grp_{lc})$ is then given by:\\ $\left\langle \Phi(f) \xi
\vert,\eta \right\rangle = \int f(x)(L(x) \xi (s(x)), \eta (r(x)))
d \nu_0(x)$.


The triple $(\mu, \H, L)$ is called a \textit{measurable
$\grp_{lc}$--Hilbert bundle}.



\begin{thebibliography}{9}

\bibitem{AS}
E. M. Alfsen and F. W. Schultz: \emph{Geometry of State Spaces of
Operator Algebras}, Birkh\''auser, Boston--Basel--Berlin (2003).

\bibitem{ICB71}
I. Baianu : Categories, Functors and Automata Theory: A Novel Approach to Quantum Automata through Algebraic--Topological Quantum Computations., \emph{Proceed. 4th Intl. Congress LMPS}, (August-Sept. 1971).

\bibitem{ICB8}
I.C. Baianu, N. Boden and D. Lightowlers.1981. NMR Spin--Echo Responses of Dipolar--Coupled Spin--1/2 Triads in Solids., \emph{J. Magnetic Resonance}, \textbf{43}:101--111.

\bibitem{BGB07}
I. C. Baianu, J. F. Glazebrook and R. Brown.: A Non--Abelian, Categorical Ontology of Spacetimes and Quantum Gravity., \emph{Axiomathes} \textbf{17},(3-4): 353-408(2007).

\bibitem{BSS}
F.A. Bais, B. J. Schroers and J. K. Slingerland: Broken quantum symmetry and confinement phases in planar physics, \emph{Phys. Rev. Lett.} \textbf{89} No. 18 (1--4): 181-201 (2002).

\bibitem{BJW}
J.W. Barrett.: Geometrical measurements in three-dimensional quantum gravity.
Proceedings of the Tenth Oporto Meeting on Geometry, Topology and Physics (2001).
\textit{Intl. J. Modern Phys.} \textbf{A 18} , October, suppl., 97-113 (2003)


\bibitem{Chaician}
M. Chaician and A. Demichev: \emph{Introduction to Quantum Groups}, World Scientific (1996).

\bibitem{Coleman}
Coleman and De Luccia: Gravitational effects on and of vacuum decay., \emph{Phys. Rev. D} \textbf{21}: 3305 (1980).

\bibitem{Connesbook}
A. Connes: \emph{Noncommutative Geometry}, Academic Press 1994.

\bibitem{CF}
L. Crane and I.B. Frenkel. Four-dimensional topological quantum field theory, Hopf categories, and the canonical bases. Topology and physics. \textit{J. Math. Phys}. \textbf{35} (no. 10): 5136-5154 (1994).

\bibitem{DT96}
W. Drechsler and P. A. Tuckey:  On quantum and parallel transport in a Hilbert bundle over spacetime., Classical and Quantum Gravity, \textbf{13}:611--632 (1996).
doi: 10.1088/0264--9381/13/4/004


\bibitem{Drinfeld}
V. G. Drinfel'd: Quantum groups, In \emph{Proc. Int. Cong. of
Mathematicians, Berkeley 1986}, (ed. A. Gleason), Berkeley, 798--820 (1987).

\bibitem{Ellis}
G. J. Ellis: Higher dimensional crossed modules of algebras,
\emph{J. of Pure Appl. Algebra} \textbf{52} (1988), 277--282.

\bibitem{Etingof1}
P.. I. Etingof and A. N. Varchenko, Solutions of the Quantum Dynamical Yang-Baxter Equation and Dynamical Quantum Groups, Comm.Math.Phys., 196:  591-640 (1998)

\bibitem{Etingof2}
P. I. Etingof and A. N. Varchenko: Exchange dynamical quantum
groups, \emph{Commun. Math. Phys.} \textbf{205} (1): 19--52 (1999)

\bibitem{Etingof3}
P. I. Etingof and O. Schiffmann: Lectures on the dynamical Yang--Baxter equations, in \emph{Quantum Groups and Lie Theory (Durham, 1999)}, pp. 89--129, Cambridge University Press, Cambridge, 2001.

\bibitem{Fauser2002}
B. Fauser: A treatise on quantum Clifford Algebras. Konstanz,
Habilitationsschrift.  $arXiv.math.QA/0202059$ (2002).

\bibitem{Fauser2004}
B. Fauser: Grade Free product Formulae from Grassman--Hopf Gebras.
Ch. 18 in R. Ablamowicz, Ed., \emph{Clifford Algebras: Applications to Mathematics, Physics and Engineering}, Birkh\"{a}user: Boston, Basel and Berlin, (2004).


\bibitem{Fell}
J. M. G. Fell. 1960. ``The Dual Spaces of  C*--Algebras.'', {\em Transactions of the American
Mathematical Society}, \textbf{94}: 365--403 (1960).

\bibitem{FernCastro}
F.M. Fernandez and E. A. Castro.:  \textit{(Lie) Algebraic Methods in Quantum Chemistry and Physics.}, Boca Raton: CRC Press, Inc  (1996).

\bibitem{Feynman}
 R. P. Feynman: Space--Time Approach to Non--Relativistic Quantum Mechanics, {\em Reviews 
of Modern Physics}, 20: 367-387 (1948). [It is also reprinted in (Schwinger 1958).]


\bibitem{frohlich:nonab}
A.~Fr{\"o}hlich, {\em Non-Abelian Homological Algebra. {I}.
{D}erived functors and satellites.\/}, Proc. London Math. Soc. (3), 11: 239--252 (1961).


\bibitem{GN}
Gel'fand, I. and Naimark, M., 1943, On the Imbedding of Normed Rings into the Ring of
Operators in Hilbert Space, {\em Recueil Math\'ematique [Matematicheskii Sbornik] Nouvelle S\'erie}, 
\textbf{12} [54]: 197-213. [Reprinted in C*-algebras: 1943--1993, in the series Contemporary 
Mathematics, 167,  Providence, R.I. : American Mathematical Society, 1994.]

\bibitem{GR02}
R. Gilmore: \textit{``Lie Groups, Lie Algebras and Some of Their Applications.''},
Dover Publs., Inc.: Mineola and New York, 2005.

\bibitem{Hahn1}
P. Hahn: Haar measure for measure groupoids., \textit{Trans. Amer. Math. Soc}. \textbf{242}: 1-33(1978).

\bibitem{Hahn2}
P. Hahn: The regular representations of measure groupoids., \textit{Trans. Amer. Math. Soc}. \textbf{242}:34-72(1978).

\bibitem{HeynLifsctz}
R. Heynman and S. Lifschitz. 1958. \emph{``Lie Groups and Lie Algebras''}., New York and London: Nelson Press.

\bibitem{HLS2k8}
C. Heunen, N. P. Landsman, B. Spitters.: A topos for algebraic quantum theory, (2008) $arXiv:0709.4364v2 [quant-ph]$

\end{thebibliography}
%%%%%
%%%%%
\end{document}
