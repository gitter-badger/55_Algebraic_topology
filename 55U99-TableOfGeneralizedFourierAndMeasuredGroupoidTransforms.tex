\documentclass[12pt]{article}
\usepackage{pmmeta}
\pmcanonicalname{TableOfGeneralizedFourierAndMeasuredGroupoidTransforms}
\pmcreated{2013-03-22 18:10:27}
\pmmodified{2013-03-22 18:10:27}
\pmowner{bci1}{20947}
\pmmodifier{bci1}{20947}
\pmtitle{table of generalized Fourier and measured groupoid transforms}
\pmrecord{46}{40739}
\pmprivacy{1}
\pmauthor{bci1}{20947}
\pmtype{Topic}
\pmcomment{trigger rebuild}
\pmclassification{msc}{55U99}
\pmsynonym{Fourier-Stieltjes transforms}{TableOfGeneralizedFourierAndMeasuredGroupoidTransforms}
%\pmkeywords{generalized Fourier transforms}
\pmrelated{FourierTransform}
\pmrelated{TwoDimensionalFourierTransforms}
\pmdefines{Fourier-Stieltjes and measured groupoid transforms}

\endmetadata

% this is the default PlanetMath preamble.  as your knowledge
% of TeX increases, you will probably want to edit this, but
% it should be fine as is for beginners.

% almost certainly you want these
\usepackage{amssymb}
\usepackage{amsmath}
\usepackage{amsfonts}
\usepackage{tabls}

% used for TeXing text within eps files
%\usepackage{psfrag}
% need this for including graphics (\includegraphics)
%\usepackage{graphicx}
% for neatly defining theorems and propositions
%\usepackage{amsthm}
% making logically defined graphics
%%%\usepackage{xypic}

% there are many more packages, add them here as you need them

% define commands here
\usepackage{amsmath, amssymb, amsfonts, amsthm, amscd, latexsym, enumerate}
\usepackage{xypic, xspace}
\usepackage[mathscr]{eucal}
\usepackage[dvips]{graphicx}
\usepackage[curve]{xy}

\setlength{\textwidth}{6.5in}
%\setlength{\textwidth}{16cm}
\setlength{\textheight}{9.0in}
%\setlength{\textheight}{24cm}

\hoffset=-.75in     %%ps format
%\hoffset=-1.0in     %%hp format
\voffset=-.4in


\theoremstyle{plain}
\newtheorem{lemma}{Lemma}[section]
\newtheorem{proposition}{Proposition}[section]
\newtheorem{theorem}{Theorem}[section]
\newtheorem{corollary}{Corollary}[section]

\theoremstyle{definition}
\newtheorem{definition}{Definition}[section]
\newtheorem{example}{Example}[section]
%\theoremstyle{remark}
\newtheorem{remark}{Remark}[section]
\newtheorem*{notation}{Notation}
\newtheorem*{claim}{Claim}

\renewcommand{\thefootnote}{\ensuremath{\fnsymbol{footnote}}}
\numberwithin{equation}{section}

\newcommand{\Ad}{{\rm Ad}}
\newcommand{\Aut}{{\rm Aut}}
\newcommand{\Cl}{{\rm Cl}}
\newcommand{\Co}{{\rm Co}}
\newcommand{\DES}{{\rm DES}}
\newcommand{\Diff}{{\rm Diff}}
\newcommand{\Dom}{{\rm Dom}}
\newcommand{\Hol}{{\rm Hol}}
\newcommand{\Mon}{{\rm Mon}}
\newcommand{\Hom}{{\rm Hom}}
\newcommand{\Ker}{{\rm Ker}}
\newcommand{\Ind}{{\rm Ind}}
\newcommand{\IM}{{\rm Im}}
\newcommand{\Is}{{\rm Is}}
\newcommand{\ID}{{\rm id}}
\newcommand{\grpL}{{\rm GL}}
\newcommand{\Iso}{{\rm Iso}}
\newcommand{\rO}{{\rm O}}
\newcommand{\Sem}{{\rm Sem}}
\newcommand{\SL}{{\rm Sl}}
\newcommand{\St}{{\rm St}}
\newcommand{\Sym}{{\rm Sym}}
\newcommand{\Symb}{{\rm Symb}}
\newcommand{\SU}{{\rm SU}}
\newcommand{\Tor}{{\rm Tor}}
\newcommand{\U}{{\rm U}}

\newcommand{\A}{\mathcal A}
\newcommand{\Ce}{\mathcal C}
\newcommand{\D}{\mathcal D}
\newcommand{\E}{\mathcal E}
\newcommand{\F}{\mathcal F}
%\newcommand{\grp}{\mathcal G}
\renewcommand{\H}{\mathcal H}
\renewcommand{\cL}{\mathcal L}
\newcommand{\Q}{\mathcal Q}
\newcommand{\R}{\mathcal R}
\newcommand{\cS}{\mathcal S}
\newcommand{\cU}{\mathcal U}
\newcommand{\W}{\mathcal W}

\newcommand{\bA}{\mathbb{A}}
\newcommand{\bB}{\mathbb{B}}
\newcommand{\bC}{\mathbb{C}}
\newcommand{\bD}{\mathbb{D}}
\newcommand{\bE}{\mathbb{E}}
\newcommand{\bF}{\mathbb{F}}
\newcommand{\bG}{\mathbb{G}}
\newcommand{\bK}{\mathbb{K}}
\newcommand{\bM}{\mathbb{M}}
\newcommand{\bN}{\mathbb{N}}
\newcommand{\bO}{\mathbb{O}}
\newcommand{\bP}{\mathbb{P}}
\newcommand{\bR}{\mathbb{R}}
\newcommand{\bV}{\mathbb{V}}
\newcommand{\bZ}{\mathbb{Z}}

\newcommand{\bfE}{\mathbf{E}}
\newcommand{\bfX}{\mathbf{X}}
\newcommand{\bfY}{\mathbf{Y}}
\newcommand{\bfZ}{\mathbf{Z}}

\renewcommand{\O}{\Omega}
\renewcommand{\o}{\omega}
\newcommand{\vp}{\varphi}
\newcommand{\vep}{\varepsilon}

\newcommand{\diag}{{\rm diag}}
\newcommand{\grp}{\mathcal G}
\newcommand{\dgrp}{{\mathsf{D}}}
\newcommand{\desp}{{\mathsf{D}^{\rm{es}}}}
\newcommand{\grpeod}{{\rm Geod}}
%\newcommand{\grpeod}{{\rm geod}}
\newcommand{\hgr}{{\mathsf{H}}}
\newcommand{\mgr}{{\mathsf{M}}}
\newcommand{\ob}{{\rm Ob}}
\newcommand{\obg}{{\rm Ob(\mathsf{G)}}}
\newcommand{\obgp}{{\rm Ob(\mathsf{G}')}}
\newcommand{\obh}{{\rm Ob(\mathsf{H})}}
\newcommand{\Osmooth}{{\Omega^{\infty}(X,*)}}
\newcommand{\grphomotop}{{\rho_2^{\square}}}
\newcommand{\grpcalp}{{\mathsf{G}(\mathcal P)}}

\newcommand{\rf}{{R_{\mathcal F}}}
\newcommand{\grplob}{{\rm glob}}
\newcommand{\loc}{{\rm loc}}
\newcommand{\TOP}{{\rm TOP}}

\newcommand{\wti}{\widetilde}
\newcommand{\what}{\widehat}

\renewcommand{\a}{\alpha}
\newcommand{\be}{\beta}
\newcommand{\grpa}{\grpamma}
%\newcommand{\grpa}{\grpamma}
\newcommand{\de}{\delta}
\newcommand{\del}{\partial}
\newcommand{\ka}{\kappa}
\newcommand{\si}{\sigma}
\newcommand{\ta}{\tau}

\newcommand{\med}{\medbreak}
\newcommand{\medn}{\medbreak \noindent}
\newcommand{\bign}{\bigbreak \noindent}

\newcommand{\lra}{{\longrightarrow}}
\newcommand{\ra}{{\rightarrow}}
\newcommand{\rat}{{\rightarrowtail}}
\newcommand{\ovset}[1]{\overset {#1}{\ra}}
\newcommand{\ovsetl}[1]{\overset {#1}{\lra}}
\newcommand{\hr}{{\hookrightarrow}}

\newcommand{\<}{{\langle}}

%\newcommand{\>}{{\rangle}}
%\usepackage{geometry, amsmath,amssymb,latexsym,enumerate}
%%%\usepackage{xypic}

\def\baselinestretch{1.1}


\hyphenation{prod-ucts}

%\grpeometry{textwidth= 16 cm, textheight=21 cm}

\newcommand{\sqdiagram}[9]{$$ \diagram  #1  \rto^{#2} \dto_{#4}&
#3  \dto^{#5} \\ #6    \rto_{#7}  &  #8   \enddiagram
\eqno{\mbox{#9}}$$ }

\def\C{C^{\ast}}

\newcommand{\labto}[1]{\stackrel{#1}{\longrightarrow}}

%\newenvironment{proof}{\noindent {\bf Proof} }{ \hfill $\Box$
%{\mbox{}}

\newcommand{\quadr}[4]
{\begin{pmatrix} & #1& \\[-1.1ex] #2 & & #3\\[-1.1ex]& #4&
 \end{pmatrix}}
\def\D{\mathsf{D}}
\begin{document}
\subsection{Generalized Fourier transforms}

\textbf{Fourier-Stieltjes} transforms and \textbf{measured groupoid} transforms are useful generalizations of the (much simpler) Fourier transform, as concisely shown in the following table- 
with the same format as C. Woo's Feature on \PMlinkname{Fourier transforms}{TableOfFourierTransforms} 
- for the purpose of direct comparison with the latter transform. Unlike the more general Fourier-Stieltjes
transform, the Fourier transform exists if and only if the function to be transformed is Lebesgue integrable over the whole real axis for $t \in{\mathbb{R}}$, or over the entire ${\mathbb{C}}$ domain when $\check{m}(t)$ is a complex function.

\begin{definition} \textbf{Fourier-Stieltjes transform}. 

Given a \emph{positive definite, measurable function} $f(x)$ on the interval 
$(-\infty ,\infty)$ there exists a monotone increasing, real-valued bounded 
function $ \alpha (t)$ such that:

\begin{equation}
f(x)=\int_\mathbb{R}e^{itx}d(\alpha (t), 
\end{equation}

for all $x \in{\mathbb{R}}$ except a small set. When $f(x)$ is defined as above and if $\alpha(t)$ is nondecreasing and bounded then the measurable function defined by the above integral is called \emph{the Fourier-Stieltjes transform of} $\alpha(t)$, and it is continuous in addition to being positive definite.

\end{definition} 

\subsubsection*{FT Generalizations}
\begin{center}
\begin{tabular}{|c|c|c|c|c|}
\hline\hline
$f(t)$ & $\F{f(t)} = \hat{f}(x)$ & Conditions* & Explanation & Description \\
\hline
$c$ & $(\sqrt{2 \pi})^{-1}c$ & Notice on the next line the & & \\ 
& & overline bar placed above $t(x)$ & & \\
\hline
$f(t)$ & $\int \hat{f}(x) \overline{t(x)}dx$ & $f(t)\in{L^1(G_l)}$, with $G_l$ a & Fourier-Stieltjes transform & $\hat{f}(x)\in{C_0(\hat{G_l})}$ \\ 
& & locally compact groupoid \cite{RW97}; & & \\
& & $\int $ is defined \emph{via} & & \\
& & a left Haar measure on $G_l$ & & \\
\hline
$\hat{m}(x)$ & $\check{m}(t)= \int e^{itx}d\hat{m}(x)$ & as above & Inverse Fourier-Stieltjes & $\check{m}(t) \in{L^1(G_l)}$, \\
& & & transform & (\cite{PALT2k1}, \cite{PALT2k3}). \\
\hline
$\hat{m}(x)$ & $\check{m}(t) = \int e^{itx}d\hat{m}(x)$ & When $G_l=\mathbb{R}$, and it exists & This is the usual & $\check{m}(t) \in{\mathbb{R}}$ \\ 
& & only when $\hat{m}(x)$ is & Inverse Fourier transform  & \\ 
& & \emph{Lebesgue integrable} on & &  \\
& & the entire real axis & &  \\
\hline\hline


\end{tabular}
\end{center}
*Note the `slash hat' on $\hat{f}(x)$ and $\hat{G_l}$.

\begin{thebibliography}{9}
\bibitem{RW97}
A. Ramsay and M. E. Walter, Fourier-Stieltjes algebras of locally compact groupoids,
\emph{J. Functional Anal}. \textbf{148}: 314-367 (1997).

\bibitem{PALT2k1}
A. L. T. Paterson, The Fourier algebra for locally compact groupoids., Preprint, (2001).

\bibitem{PALT2k3}
A. L. T. Paterson, The Fourier-Stieltjes and Fourier algebras for locally
compact groupoids., (2003) \PMlinkexternal{Free PDF file download}{http://aux.planetmath.org/files/objects/10739/AFourierStjelties_LocallyCompactsGds_Harmonic0310138v1.pdf}.

\end{thebibliography}
%%%%%
%%%%%
\end{document}
